%%%%%%%%%%%%%%%%%%%%%%%%%%%%%%%%%%%%%%%%%%%%%%%%%%%%%%%%%%%%%%%%%%
%%%  MTLPEs                                                    %%%
%%%%%%%%%%%%%%%%%%%%%%%%%%%%%%%%%%%%%%%%%%%%%%%%%%%%%%%%%%%%%%%%%%
\documentclass[twoside,fleqn,a4paper,dvips]{article}
\usepackage{textcomp,amsmath,amssymb,amsthm,stmaryrd}
\usepackage{geometry}
\usepackage[ps2pdf]{hyperref} % remove for printing
%\usepackage[active]{srcltx}
\usepackage{mymath,mythm,mcrl}
  
\setlength{\mathindent}{0.5em}

\def\YSU{\texttt{ YSU: TODO :USY\marginpar{!!!} }}
\def\JFG{\texttt{ JFG: TODO :GFJ\marginpar{!!!} }}
\newcommand{\afunc}[3]{\mathsf{#1}:#2\rightarrow#3}

\title{LPE definitions and an ATerm representation format for \mcrl\ 
  with multiactions and time}
  
\author{Jan Friso Groote \and Yaroslav S. Usenko}

\begin{document}
\maketitle

\section{LPE definitions}

The equation below represents a Linear Process Equation for \mcrl\ 
with multiactions and time (MTLPE).
\begin{gather}
\begin{split}
\pr{X}(\vect{d\ap D})=&\sum_{i\in I}\sum_{\vect{e_i\ap E_i}} c_i(\vect{d,e_i})\to\act{a}^0_i(\vect{f_{i,0}}(\vect{d,e_i}))\comm\dots\comm\act{a}^{n(i)}_i(\vect{f_{i,n(i)}}(\vect{d,e_i}))\at t_i(\vect{d,e_i})\seqc
\pr{X}_i(\vect{g_i}(\vect{d,e_i}))\notag\\
&+\sum_{j\in J}\sum_{\vect{e_j\ap E_j}} c_j(\vect{d,e_j})\to\act{a}^0_j(\vect{f_{j,0}}(\vect{d,e_j}))\comm\dots\comm\act{a}^{n(j)}_j(\vect{f_{j,n(j)}}(\vect{d,e_j}))\at t_j(\vect{d,e_j})\\
&+\sum_{\vect{e_\delta\ap E_\delta}} c_\delta(\vect{d,e_\delta})\to\delta\at t_\delta(\vect{d,e_\delta})
\end{split}
\end{gather}
where $I$ and $J$ are disjoint.

It is possible to translate multiactions to regular \mcrl\ actions
(with longer parameter lists).  In this way a MTLPE can be translated
to a TLPE, preserving equivalence.  The TLPE that corresponds to the
above MTLPE is defined in the following way.
\begin{gather}
\begin{split}
\pr{X}(\vect{d\ap D})=&\sum_{i\in I}\sum_{\vect{e_i\ap E_i}} c_i(\vect{d,e_i})\to\act{a}^0_i\_\act{a}^1_i\_\dots\_\act{a}^{n(i)}_i(\vect{f_{i,0}}(\vect{d,e_i}),\dots,\vect{f_{i,n(i)}}(\vect{d,e_i}))\at t_i(\vect{d,e_i})\seqc
\pr{X}_i(\vect{g_i}(\vect{d,e_i}))\notag\\
&+\sum_{j\in J}\sum_{\vect{e_j\ap E_j}} c_j(\vect{d,e_j})\to\act{a}^0_j\_\act{a}^1_j\_\dots\_\act{a}^{n(j)}_j(\vect{f_{j,0}}(\vect{d,e_j}),\dots,\vect{f_{j,n(j)}}(\vect{d,e_j}))\at t_j(\vect{d,e_j})\\
&+\sum_{\vect{e_\delta\ap E_\delta}} c_\delta(\vect{d,e_\delta})\to\delta\at t_\delta(\vect{d,e_\delta})
\end{split}
\end{gather}
where $I$ and $J$ are disjoint, and
$\act{a}^0_i\_\act{a}^1_i\_\dots\_\act{a}^{n(i)}_i$ and
$\act{a}^0_j\_\act{a}^1_j\_\dots\_\act{a}^{n(j)}_j$ are new actions
(for each $i$ and $j$), parameterized by the concatenation of the
parameter lists of the contained actions.

\YSU formalize below

\begin{thm}
  Given 
  $\mathrm{MTLPE1}=(\mathrm{genspect})=\mathrm{MTLPE2}$,\\
  $\mathrm{TLPE(MTLPE1)}=(\mathrm{timed\ mcrl})=\mathrm{TLPE(MTLPE2)}$.
\end{thm}

Time can be eliminated from TLPEs in a similar way (see page 106 of
the thesis).

\appendix
\newpage
\section{Static Semantics and Well-formedness}
\label{section:SSC} 
In this section it is defined when a specification is correctly defined.
We use the syntactical categories from the previous section (in teletype
font) to refer to items in a specification. If we denote a
concrete part of a specification, we prefer using the latex symbols, to
increase readability. The definitions below are an adapted copy from those in
\cite{GroPo94a}.

In
essence the static semantics says that functions and terms are well
typed, and some sorts and functions are present in the specification. The
validity of all static semantic requirements can efficiently be decided for any
specification.

A specification is well formed, if it satisfies the static semantic requirements,
the symmetric closure of the communication function is associative, there are no
empty sorts and the sorts $\Bool$ and $\Time$ are appropriately defined. We
only give an operational semantics to well formed specifications.
\subsection{The signature of a specification}
\begin{defn}
The signature $\Sig(E)$ of a {\tt Specification} $E$ consists of a
seven-tuple \[(\Sort,\Fun,\Map,\Act,\Comm,\Proc,\Init)\] where
each component is a set containing all elements of a main syntactical category
declared in $E$.
The signature $\Sig(E)$ of $E$ is inductively defined as follows:
\newcommand{\fd}{{\it fd}}
\newcommand{\md}{{\it md}}
\newcommand{\pe}{{\it pe}}
\newcommand{\ad}{{\it ad}}
\begin{itemize}
\item
If $E\equiv \sort~n_1\cdots n_m$ with $m\geq 1$,
then
$\Sig(E)\wor(\{n_1,\ldots,n_m\},\emptyset,\emptyset,\emptyset,
\emptyset,\emptyset,\emptyset).$
\item
If $E\equiv
\func~\fd_1\cdots \fd_m$ with $m\geq 1$,
then
$\Sig(E)\wor(\emptyset,\Fun,
\emptyset,\emptyset,
\emptyset,\emptyset,\emptyset),$
where \[\begin{array}{lll}
\Fun &\wor&
\{n_{ij}\ap \rightarrow S_i\mid \fd_i\equiv
n_{i1},\ldots,n_{il_i}\ap \rightarrow S_i, 1\leq i\leq m, 1\leq j\leq
l_i\}\\
&\cup&\{n_{ij}\ap S_{i1}\times\cdots\times S_{ik_i}\rightarrow S_i\mid\\
&&~~~~~\fd_i\equiv
n_{i1},\ldots,n_{il_i}\ap S_{i1}\times\cdots\times S_{ik_i}\rightarrow
S_i, 1\leq i\leq
m, 1\leq j\leq l_i\}.
\end{array}
\]
\item
If $E\equiv
\map~\md_1\cdots \md_m$ with $m\geq 1$,
then
$\Sig(E)\wor(\emptyset,\emptyset,\Map, \emptyset,\emptyset,
\emptyset,\emptyset),$
where \[\begin{array}{lll}
\Map &\wor&
\{n_{ij}\ap \rightarrow S_i\mid \md_i\equiv
n_{i1},\ldots,n_{il_i}\ap \rightarrow S_i, 1\leq i\leq m, 1\leq j\leq
l_i\}\\
&\cup&\{n_{ij}\ap S_{i1}\times\cdots\times S_{ik_i}\rightarrow S_i\mid\\
&&~~~~~\md_i\equiv
n_{i1},\ldots,n_{il_i}\ap S_{i1}\times\cdots\times S_{ik_i}\rightarrow
S_i, 1\leq i\leq
m, 1\leq j\leq l_i\}.
\end{array}
\]
\item
If $E$ is a {\tt Equation-specification},
then
$\Sig(E)\wor(\emptyset,\emptyset,\emptyset,\emptyset,\emptyset,\emptyset,\emptyset)$.
\item
If $E\equiv
\act~\ad_1\cdots \ad_m$ with $m\geq 1$,
then
$\Sig(E)\wor(\emptyset,\emptyset,\emptyset,\Act,
\emptyset,\emptyset,
\emptyset)$, where
\[\begin{array}{lll}
\Act&\wor&
\{n_{i}\mid ad_i\equiv
n_{i}, 1\leq i\leq m \}\\
&\cup&\{n_{ij}\ap S_{i1}\times\cdots\times S_{ik_i}\mid\\
&&~~~~~ad_i\equiv
n_{i1},\ldots,n_{il_i}\ap S_{i1}\times\cdots\times S_{ik_i}, 1\leq i\leq
m, 1\leq j\leq l_i\}.
\end{array}
\]
\item
If $E\equiv
\comm~cd_1\cdots cd_m$ with $m\geq 1$,
then
$\Sig(E)\wor(\emptyset,\emptyset,\emptyset,\emptyset,
\{cd_i\mid 1\leq
i\leq m\}, \emptyset,\emptyset).$
\item
If $E\equiv
\proc~pd_1\cdots pd_m$ with $m\geq 1$,
then
$\Sig(E)\wor(\emptyset,\emptyset,\emptyset,\emptyset,\emptyset,
\{pd_i\mid 1\leq i\leq m\},\emptyset).$
\item
If $E\equiv
\init~\pe$ then $\Sig(E)\wor(\emptyset,\emptyset,\emptyset,\emptyset,\emptyset,
\emptyset,\{\pe\}).$
\item
If $E\equiv E_1~E_2$
with $\Sig(E_i)=(Sort_{i},\Fun_{i},\Map_i,
\Act_{i},\Comm_{i},\Proc_{i},\Init_i)$ for $i=1,2$, then
\[\begin{array}{l}
\Sig(E)\wor(Sort_{1}\cup Sort_{2},\Fun_{1}\cup \Fun_{2},
\Map_1\cup\Map_2,\\
\hspace{8em}\Act_{1}\cup \Act_{2},\Comm_{1}\cup \Comm_{2},\Proc_{1}\cup
\Proc_{2},
\Init_1\cup\Init_2).
\end{array}\]
\end{itemize}
\end{defn}
\begin{defn}
Let $Sig=(Sort,\Fun,\Map,\Act,\Comm,\Proc,\Init)$ be a signature. We write
\[\begin{array}{llll}
\Sig.Sort~\mbox{for}~Sort,&
\Sig.\Fun~\mbox{for}~\Fun,&
\Sig.\Map~{\rm for}~\Map,&
\Sig.\Act~\mbox{for}~\Act,\\
\Sig.\Comm~\mbox{for}~\Comm,&
\Sig.\Proc~\mbox{for}~\Proc,&
\Sig.\Init~\mbox{for}~\Init.
\end{array}\]
\end{defn}
\subsection{Variables}
Variables play an important role in specifications. The next definition
says given a specification $E$, which elements from {\tt Name} can play the
role of a variable without confusion with
defined constants. Moreover, variables must have an unambiguous and
declared sort.
\begin{Dn}
Let $Sig$ be a signature. A set $\NV$ containing 
pairs
$\langle x\ap S\rangle$ with $x$ and $S$ from {\tt Name}, 
is called a {\it set of variables} over $Sig$ iff for each $\langle
x\ap S\rangle\in \NV$:
\begin{itemize}
\item
for each {\tt Name} $S'$ and {\tt Process-term} $p$ it holds that
$x\ap \rightarrow S'\notin \Sig.\Fun\cup\Sig.\Map$, 
$x\notin\Sig.\Act$ and $x=p\notin\Sig.\Proc$,
\item
$S\in \Sig.Sort$,
\item
for each {\tt Name} $S'$ such that $S'\not\equiv S$ it holds
that $\langle x\ap S'\rangle\not\in \NV$.
\end{itemize}
\end{Dn}
\begin{Dn}
\newcommand{\vd}{{\it vd}}
Let $\vd$ be a {\tt Variable-declaration}.
The function $\Vars$ is defined by:
\[\Vars(\vd)\wor\left\{
\begin{array}{ll}
\emptyset&\mbox{if $\vd$ is empty},\\
\{\langle x_{ij}\ap S_i\rangle\mid 1\leq i\leq m,\\
\phantom{\{\langle x_i\ap S_{ij}\rangle\mid }\hspace*{1mm}1\leq j\leq
l_i\}&
\mbox{if}~\vd\equiv
\var~x_{11},\ldots,x_{1l_1}\ap  S_1~\ldots~x_{m1},\ldots,x_{ml_m}\ap S_m.
\end{array}\right.\]
\end{Dn}
In the following definitions we give functions yielding the sort of and the 
variables in a
{\tt Data-term} $t$. 
\begin{Dn}
Let $t$ be a {\tt data-term} and $Sig$ a signature. Let $\NV$ be a set
of variables over $Sig$. We define:
\[sort_{Sig,\NV}(t)\wor\left\{
\begin{array}{ll}
S&\mbox{if~}t\equiv x\mbox{ and }\langle x\ap S\rangle \in \NV,\\
S&\mbox{if $t\equiv n$, $n\ap \rightarrow S\in
\Sig.\Fun\cup\Sig.\Map$}\\
&\hspace*{0.5cm}\mbox{and for no $S'\not\equiv S$}~n\ap \rightarrow
S'\in\Sig.\Fun\cup\Sig.\Map,\\
S&\mbox{if $t\equiv n(t_1,\ldots,t_m)$,}\\
&\mbox{$\hspace{0.5cm}n\ap sort_{Sig,\NV}(t_1)\times \cdots
\times sort_{Sig,\NV}(t_m)\rightarrow S \in \Sig.\Fun\cup\Sig.\Map$}\\
&\hspace{0.5cm}\mbox{and for
no}~S'\not\equiv S~n\ap sort_{Sig,\NV}(t_1)\times \cdots
\times sort_{Sig,\NV}(t_m)\rightarrow\\
&\hspace*{0.5cm}S' \in \Sig.\Fun\cup\Sig.\Map,\\
\perp&\mbox{otherwise}.
\end{array}\right.\]
\end{Dn}
If a variable or a function is not or inappropriately declared 
no answer can be obtained. In this case $\perp$ results.
\begin{Dn}
Let $Sig$ be a signature, $\NV$ a set of
variables over $Sig$ and let $t$ be a {\tt Data-term}.
\[\Var_{Sig,\NV}(t)\wor
\left\{
\begin{array}{ll}
\{\langle x\ap S\rangle\}&\mbox{if }t\equiv x\mbox{ and }\langle
x\ap S\rangle\in \NV,\\
\emptyset&\mbox{if~}t\equiv n\mbox{ and }n\ap \rightarrow S\in
\Sig.\Fun\cup\Sig.\Map,\\
\bigcup_{1\leq i\leq m} \Var_{Sig,\NV}(t_i)&\mbox{if~}t\equiv 
n(t_1,\ldots,t_m),\\
\{\perp\}&\mbox{otherwise}.
\end{array}\right.\]
\end{Dn}
We call a {\tt Data-term} $t$ {\em closed} wrt.\ a signature
$Sig$ and a set of variables $\NV$ iff $\Var_{Sig,\NV}(t)=
\emptyset$. Note that $\Var_{Sig,\NV}(t)\subseteq\NV\cup\{\perp\}$
for any {\tt data-term} $t$. If $\perp\in \Var_{\Sig,\NV}(t)$, then
due  to some missing or inappropriate declaration 
it can not be determined what the variables of $t$ are on basis of
$\Sig$ and $\NV$.
\subsection{Static semantics}
A {\tt Specification} must be internally consistent. This means
that all objects that are used must be declared exactly once and 
are used such that the sorts are correct. It also means that action,
process, constant and variable names cannot be confused.
Furthermore, it means that communications are specified in a functional
way and that it is guaranteed that the terms used in an equation are well-typed.
Because all these properties can be statically decided, a
specification that is internally consistent is called SSC
({\em Statically Semantically Correct}). All next definitions
culminate in Definition \ref{SSC-mCRL}.

\begin{Dn-arg}{{\tt Data-term}}
Let $\Sig$ be a signature, and let $\NV$ be a set of variables over
$\Sig$.
A {\tt Data-term} $t$ is called SSC wrt.\ $\Sig$ and $\NV$ iff one of
the following holds
\begin{itemize}
\item
$t\equiv n$ with $n$ a {\tt Name} and
$\langle n\ap S\rangle\in \NV$ for some $S$, or
$n\ap  \rightarrow sort_{Sig,\NV}(n)\in \Sig.\Fun\cup\Sig.\Map$.
\item
$t\equiv n(t_1,\ldots,t_m)$ ($m\geq 1$) and
$n\ap sort_{Sig,\NV}(t_1)\times\cdots\times sort_{Sig,\NV}(t_m)
\rightarrow sort_{Sig,\NV}(n(t_1,\ldots,t_m))\in \Sig.\Fun\cup\Sig.\Map$ and
all
$t_i$ ($1\leq i\leq m$) are SSC wrt.\ $Sig$ and $\NV$.
\end{itemize}
\end{Dn-arg}

\begin{Dn-arg}{{\tt Equation-section}}
Let $\Sig$ be a signature and $\NV$ be a set of variables over $\Sig$. An {\tt
Equation-section}
$\rew~rw_1\cdots rw_m$
with $m\geq 1$ is SSC wrt.\ $\Sig$ and $\NV$ iff for all
$1\leq i\leq m $ if $rw_i\equiv t_1=t_2$,
both $t_1$ and $t_2$ are SSC wrt.\ $\Sig$ and $\NV$ and
$sort_{\Sig,\NV}(t_1)=sort_{Sig,\NV}(t_2)$.
\end{Dn-arg}

\begin{Dn-arg}{{\tt Variable-declaration}}
A {\tt Variable-declaration} {\it vd} is SSC wrt.\ a signature
$Sig$ iff one of the following holds.
\begin{itemize}
\item
{\it vd} is empty.
\item
${\it vd}\equiv\var$
\begin{minipage}[t]{7cm}
$n_{11},\ldots,n_{1k_1}\ap  S_1$\\
$\vdots$\\
$n_{m1},\ldots,n_{mk_m}\ap  S_m$\vspace{2ex}
\end{minipage}\\
with $m\geq 1$, $k_i\geq 1$ for $1\leq i\leq m$ and
\begin{itemize}
\item
$n_{ij}\not\equiv n_{i'j'}$ whenever $i\neq i'$ or 
$j\neq j'$ for $1\leq i\leq m$, $1\leq i'\leq m$,
$1\leq j\leq k_i$ and $1\leq j'\leq k_{i'}$,
\item
the set $\Vars(\var~n_{11},\ldots,n_{1k_1}: 
S_1\cdots n_{m1},\ldots,n_{mk_m}:
S_m)$ is a set of variables
over $Sig$.
\end{itemize}
\end{itemize}
\end{Dn-arg}

\begin{Dn-arg}{{\tt Process-term}}
Let $\Sig$ be a signature and $\NV$ be a set of variables over $\Sig$.
We say that a {\tt Process-term} $p$ is {\it SSC wrt.\ to $\Sig$ and
$E$} iff one of the following hold:
\begin{itemize}
\item
$p\equiv p_1+p_2$, 
$p\equiv p_1\parallel p_2$, $p\equiv p_1\leftmerge p_2$, $p\equiv p_1\mid
p_2$,
$p\equiv p_1 {\cdot} p_2$ or $p\equiv p_1\before p_2$ and
\begin{itemize}
\item
$p_1$ is SSC wrt.\ $Sig$ and $\NV$,
\item
$p_2$ is SSC wrt.\ $Sig$ and $\NV$.
\end{itemize}
\item
$p\equiv p_1\triangleleft t \triangleright p_2$ and
\begin{itemize}
\item
$p_1$ is SSC wrt.\ $Sig$ and $\NV$,
\item
$p_2$ is SSC wrt.\ $Sig$ and $\NV$,
\item
$t$ is SSC wrt.\ $Sig$ and $\NV$ and $sort_{Sig,\NV}(t)=\Bool$.
\end{itemize}
\item
$p\equiv p_1\at{t}$ and 
\begin{itemize}
\item
$p_1$ is SSC wrt.\ $Sig$ and $\NV$
\item
$t$ is SSC wrt.\ $Sig$ and $\NV$ and $sort_{Sig,\NV}(t)=\Time$.
\end{itemize}
\item
$p\equiv \delta$ or $p\equiv\tau$.
\item
$p\equiv\partial_{\{n_1,\ldots,n_m\}}p_1$ or
$p\equiv \tau_{\{n_1,\ldots,n_m\}}p_1$ with $m\geq 1$ and
\begin{itemize}
\item
for all $1\leq i< j \leq m$ $n_i\not\equiv n_j$,
\item
for $1\leq i\leq m$ either $n_i\in \Sig.\Act$ or
$n_i\ap S_1\times \cdots\times S_k\in
\Sig.\Act$ for some $k\geq 1$ and {\tt Name}s $S_1,\ldots,S_k$,
\item
$p_1$ is SSC wrt.\ $Sig$ and $\NV$.
\end{itemize}
\item
$p\equiv\rho_{\{n_1\rightarrow n_1',\ldots,n_m\rightarrow
n_m'\}}p_1$ and
\begin{itemize}
\item
for $1\leq i\leq m$ either $n_i\in \Sig.\Act$ or
$n_i\ap S_1\times \cdots\times S_k\in
\Sig.\Act$ for some $k\geq 1$ and {\tt Name}s $S_1,\ldots,S_k$,
\item
for each $1\leq i<j\leq m$ it holds that $n_i\not\equiv n_j$,
\item
for $1\leq i\leq m$, $k\geq 1$ and {\tt Name}s $S_1,..,S_k$ it holds
that if $n_i\ap S_1\times\cdots\times S_k\in \Sig.\Act$, then also
$n_i'\ap S_1\times\cdots\times S_k\in \Sig.\Act$,
\item
for $1\leq i\leq m$ it holds that if $n_i\in \Sig.\Act$, then also
$n_i'\in \Sig.\Act$,
\item
$p_1$ is SSC wrt.\ $Sig$ and $\NV$.
\end{itemize}
\item
$p\equiv \Sigma_{x\ap S}p_1$ and
iff
\begin{itemize}
\item
$(\NV\backslash\{\langle x\ap S'\rangle \mid S'
\mbox{ is a {\tt Name}}\})\cup\{\langle x\ap S\rangle\}$
is a set of variables over $Sig$,
\item
$p_1$ is SSC wrt.\ $Sig$ and
$(\NV\backslash\{\langle x\ap S'\rangle \mid S'\mbox{ is a
{\tt Name}}\})\cup\{\langle x\ap S\rangle\}$.
\end{itemize}
\item
$p\equiv n$ and 
$n=p'\in \Sig.\Proc$ for some {\tt Process-term} $p'$, or $n\in
\Sig.\Act$.
\item
$p\equiv n(t_1,\ldots,t_m)$
with $m\geq 1$ and
\begin{itemize}
\item
$n(x_1\ap sort_{Sig,\NV}(t_1),\ldots,x_m\ap sort_{Sig,\NV}(t_m))=p'\in
\Sig.\Proc$ for {\tt Name}s $x_1,\ldots,x_m$ and\\ {\tt Process-term} $p'$,
or
$n\ap sort_{Sig,\NV}(t_1)\times \cdots\times sort_{Sig,\NV}(t_m)\in
\Sig.\Act$,
\item
for $1\leq i\leq m$ the {\tt Data-term}
$t_i$ is SSC wrt.\ $Sig$ and $\NV$.
\end{itemize}
\end{itemize}
\end{Dn-arg}

\begin{Dn-arg}{{\tt Action-declaration}}
Let $\Sig$ be a signature.
An {\tt Action-declaration} ${\it ad}$ is SSC wrt.\ $Sig$ iff one of the following
hold:
\begin{itemize}
\item
${\it ad}\equiv n$ and
for each {\tt Name} $S'$ it holds that $n\ap \rightarrow S' \notin
\Sig.\Fun\cup\Sig.\Map$.
\item
An {\tt Action-declaration} $n_1,\ldots,n_m\ap S_1\times \cdots \times
S_k$ with $k,m\geq 1$ is SSC wrt. $Sig$ iff
\begin{itemize}
\item
for all $1\leq i<j\leq m$ it holds that $n_i\not\equiv n_j$,
\item
for all $1\leq i\leq k$ it holds that $S_i\in \Sig.Sort$,
\item for all $1\leq i\leq m$ and for each
{\tt Name} $S'$
it holds that $n_i\ap S_1\times\cdots\times S_k\rightarrow S'\notin
\Sig.\Fun\cup\Sig.\Map$.
\end{itemize}
\end{itemize}
\end{Dn-arg}

\begin{Dn-arg}{{\tt Specification}}
Let $Sig$ be a signature and $\NV$ be a set of variables over $Sig$. We
define the predicate `is SSC wrt.\ $Sig$'
inductively over the syntax of a {\tt Specification}.
\begin{itemize}
\item
A {\tt Specification}
$\sort~n_1\cdots n_m$ with $m\geq 1$
is SSC wrt.\ $Sig$ iff all $n_1,\ldots,n_m$ are pairwise
different.
\item
A {\tt Specification}
$\func$
\begin{minipage}[t]{7cm}
$n_{11},\ldots,n_{1l_1}\ap S_{11}\times \cdots\times S_{1k_1}\rightarrow
S_1$\\
$\vdots$\\
$n_{m1},\ldots,n_{ml_m}\ap S_{m1}\times \cdots\times S_{mk_m}\rightarrow
S_m$\vspace{2ex}
\end{minipage}\\
with $m\geq 1$, $l_i\geq 1$, $k_i\geq 0$ for $1\leq i\leq m$
is SSC wrt.\ $Sig$ iff 
\begin{itemize}
\item
for all $1\leq i\leq m$  $n_{i1},\ldots,n_{il_i}$ are
pairwise different,
\item
for all $1\leq i<j\leq m$ it holds that
if $n_{ik}\equiv n_{jk'}$ for some $1\leq k\leq l_i$ and $1\leq
k'\leq l_j$, then either $k_i\neq k_j$, or $S_{il}\not\equiv
S_{jl}$ for some $1\leq l\leq k_i$,
\item
for all $1\leq i\leq m$ and $1\leq j\leq k_i$ it holds that
$S_{ij}\in \Sig.Sort$ and $S_i\in \Sig.Sort$.
\end{itemize}
\item
A {\tt Specification}
$\map$
\begin{minipage}[t]{7cm}
$n_{11},\ldots,n_{1l_1}\ap S_{11}\times \cdots\times S_{1k_1}\rightarrow
S_1$\\
$\vdots$\\
$n_{m1},\ldots,n_{ml_m}\ap S_{m1}\times \cdots\times S_{mk_m}\rightarrow
S_m$\vspace{2ex}
\end{minipage}\\
with $m\geq 1$, $l_i\geq 1$, $k_i\geq 0$ for $1\leq i\leq m$
is SSC wrt.\ $Sig$ iff 
\begin{itemize}
\item
for all $1\leq i\leq m$  $n_{i1},\ldots,n_{il_i}$ are
pairwise different,
\item
for all $1\leq i<j\leq m$ it holds that
if $n_{ik}\equiv n_{jk'}$ for some $1\leq k\leq l_i$ and $1\leq
k'\leq l_j$, then either $k_i\neq k_j$, or $S_{il}\not\equiv
S_{jl}$ for some $1\leq l\leq k_i$,
\item
for all $1\leq i\leq m$ and $1\leq j\leq k_i$ it holds that
$S_{ij}\in \Sig.Sort$ and $S_i\in \Sig.Sort$.
\end{itemize}
\item
A {\tt Specification} of the form:
$\begin{minipage}[t]{7cm}
{\it var-dec}\\
{\it rew-rul}\vspace{2ex}
\end{minipage}$\\
where {\it var-dec} is a {\tt Variable-declaration}
and {\it rew-rul} is a {\it Equation-section}
is SSC wrt.\ $Sig$ iff
\begin{itemize}
\item
{\it var-dec} is SSC wrt.\ $Sig$,
\item
{\it rew-rul} is SSC wrt.\ $Sig$ and $\Vars(\mbox{\it var-dec})$.
\end{itemize}
\item
A {\tt Specification} $\act~ad_1\cdots ad_m$
with $m\geq 1$ is SSC wrt. $Sig$ iff
\begin{itemize}
\item for all $1\leq i\leq m$ the {\tt Action-declaration}
$ad_i$ is SSC wrt.  $\Sig$,
\item for all $1\leq i<j\leq m$ it holds that
$\Sig(\act~ad_i).\Act\cap \Sig(\act~ad_j).\Act=\emptyset$.
\end{itemize}
\item
A {\tt Specification}
$\comm~n_{11}| n_{12}= n_{13}\cdots n_{m1}|
n_{m2}= n_{m3}$ with $m\geq 1$ is SSC wrt.\ $Sig$ iff
\begin{itemize}
\item
for each $1\leq i<j\leq m$ it is not the case that $n_{i1}\equiv n_{j1}$
and $n_{i2}\equiv n_{j2}$, or $n_{i1}\equiv n_{j2}$ and $n_{i2}\equiv
n_{j1}$,
\item
for each $1\leq i\leq m$ either $n_{i1}\in \Sig.\Act$ or there is a 
$k\geq 1$ such that $n_{i1}\ap S_1\times \cdots \times S_k\in \Sig.\Act$,
\item
for each $1\leq i\leq m$, $k\geq 1$ and {\tt Name}s $S_1,\ldots,S_k$ it
holds
that if $n_{i1}\ap S_1\times\cdots\times S_k\in \Sig.\Act$ then
$n_{i2}\ap S_1\times\cdots\times
S_k\in \Sig.\Act$ and $n_{i3}\ap S_1\times\cdots\times S_k\in \Sig.\Act$,
\item
for each $1\leq i\leq m$, $k\geq 1$ and {\tt Name}s $S_1,\ldots,S_k$ it
holds
that if $n_{i2}\ap S_1\times\cdots\times S_k\in \Sig.\Act$ then
$n_{i1}\ap S_1\times\cdots\times
S_k\in \Sig.\Act$ and $n_{i3}\ap S_1\times\cdots\times S_k\in \Sig.\Act$,
\item
for each $1\leq i\leq m$ it holds
that if $n_{i1}\in \Sig.\Act$ then
$n_{i2}\in \Sig.\Act$ and $n_{i3}\in \Sig.\Act$,
\item
for each $1\leq i\leq m$ it holds
that if $n_{i2}\in \Sig.\Act$ then
$n_{i1}\in \Sig.\Act$ and $n_{i3}\in \Sig.\Act$.
\end{itemize}
\item
A {\tt specification} 
$\proc~pd_1\cdots pd_m$
with $m\geq 1$ is SSC wrt.\ $Sig$ iff
\begin{itemize}
\item
for each $1\leq i<j\leq m$:
\begin{itemize}
\item
if $pd_i\equiv n=p$ and $pd_j\equiv n'=p'$ then $n\not\equiv
n'$,
\item
if for some $k\geq1$ it holds that $pd_i\equiv
n(x_1\ap S_1,\ldots,x_k\ap S_k)=p$
and\\ $pd_j\equiv n'(x_1'\ap S_1,\ldots,x_k'\ap S_k)=p'$ then $n\not\equiv
n'$,
\end{itemize}
\item
if $pd_i\equiv n=p$ ($1\leq i\leq m$), then 
\begin{itemize}
\item
$n\not\in
\Sig.\Act$ and $p$ is SSC wrt.\ $Sig$ and $\emptyset$, and
\item
for each {\tt Name} $S'$ it holds that $n\ap \rightarrow S'\notin
\Sig.\Fun\cup\Sig.\Map$,
\end{itemize}
\item
if $pd_i\equiv n(x_{1}\ap S_{1},\ldots,x_{k}\ap S_{k})=p$
($1\leq i\leq m$), then
\begin{itemize}
\item $n\ap S_{1}\times\cdots\times S_{k}\not\in \Sig.\Act$,
\item
for each {\tt Name} $S'$ it holds that $n\ap S_{1}\times\cdots\times
S_{k}\rightarrow S'\notin \Sig.\Fun\cup\Sig.\Map$,
\item the {\tt Name}s $x_{1},\ldots,x_{k}$ are pairwise different
and $\{\langle x_{j}\ap S_{j}\rangle\mid 1\leq j\leq k\}$ is a set
of variables over $Sig$,
\item $p$ is SSC wrt.\ $Sig$ and $\{\langle
x_{j}\ap S_{j}\rangle\mid 1\leq j\leq k\}$.
\end{itemize}
\end{itemize}
\item
A {\tt Specification} of the form $\init~p$ is SSC wrt.\ $\Sig$ 
iff $p$ SSC wrt.\ to $\Sig$ and $\emptyset$.
\item
A {\tt specification} $E_1~E_2$ is SSC wrt.\ $Sig$ iff 
\begin{itemize}
\item
$E_1$ and $E_2$ are SSC wrt.\ $Sig$,
\item
$\Sig(E_1).Sort \cap \Sig(E_2).Sort=\emptyset$,
\item
if $n\ap S_1\times \cdots\times S_m\rightarrow S \in \Sig(E_1).\Fun\cup
\Sig(E_1).\Map$ for some
$m\geq 0$ then
$n\ap S_1\times\cdots\times S_m\rightarrow S'\notin \Sig(E_2).\Fun\cup
\Sig(E_2).\Map$ for any {\tt Name} $S'$,
\item
$\Sig(E_1).\Act \cap \Sig(E_2).\Act=\emptyset$,
\item
if $n_1| n_2=n_3\in \Sig(E_1).\Comm$ then for any {\tt Name}s $n_3'$
and $n_3''$ $n_1| n_2=n_3'\notin
\Sig(E_2).\Comm$ and $n_2| n_1=n_3''\notin \Sig(E_2).\Comm$, 
\item
if $pd_1\in \Sig(E_1).\Proc$ and $pd_2\in \Sig(E_2).\Proc$, then
\begin{itemize}
\item
if $pd_1\equiv n_1=p_1$ and $pd_2\equiv n_2=p_2$, then $n_1\not\equiv
n_2$,
\item
if for some $m\geq 1$ $pd_1\equiv n_1(x_1\ap S_1,\ldots,x_m\ap S_m)=p_1$
and 
$pd_2\equiv n_2(x_1'\ap S_1,\ldots,x_m'\ap S_m)=p_2$, then $n_1\not\equiv
n_2$,
\end{itemize}
\item
$\Sig(E_1).\Init=\emptyset$ or $\Sig(E_2).\Init=\emptyset$.
\end{itemize}
\end{itemize}
\end{Dn-arg}
\begin{Dn}\llabel{SSC-mCRL}
Let $E$ be a {\tt Specification}. We say that $E$ is SSC iff $E$ is SSC
wrt.\ $\Sig(E)$.
\end{Dn}
\subsection{The communication function}
The following definition helps us in guaranteeing that the
communication function is commutative and associative. This implies
that the merge is also commutative and associative which allows us to
write parallel processes without brackets. 
\begin{Dn}
Let $Sig$ be a signature. The set $\Sig.\Comm^*$ is defined by:
\[\begin{array}{rcl}
\Sig.\Comm^*&\wor&\{n_1| n_2=n_3,~ n_2| n_1=n_3 \mid
n_1| n_2=n_3\in \Sig.\Comm\}.
\end{array}\]
So, in $\Sig.\Comm^*$ communication is always commutative. A
{\it specification} $E$ is {\it communication-associative} iff
\[
\begin{array}{l}
n_1| n_2= n,~ n| n_3=n'\in
\Sig(E).\Comm^*~\Rightarrow\\
\hspace*{3cm}\exists n''\ap ~n_2| n_3=n'',
n_1| n''= n' \in \Sig(E).\Comm^*.
\end{array}
\]
\end{Dn}
With the condition that $E$ is SSC
this exactly implies that communication is associative.
\subsection{Well-formed $\mCRL$ specifications}
We define what well-formed specifications are. We only provide
well-formed {\tt Specification}s with a semantics. Well-formedness
is a decidable property.
\begin{Dn}
Let $\Sig$ be a signature. We call a {\tt Name} $S$ a {\it constructor sort}
iff $S\in\Sig.\Sort$ and there exists {\tt Name}s $S_1,\ldots, S_k, f$ ($k\geq 0$)
such that $f\ap S_1\times \cdots\times S_k\rightarrow S\in \Sig.\Fun$.
\end{Dn}
\begin{Dn}
Let $E$ be a {\tt Specification} that is SSC. 
We inductively define which sorts are {\it non empty constructor sorts} in $E$.
A constructor sort $S$ is called {\it non empty} iff there is a function
$f\ap S_1\times\cdots\times S_k\rightarrow S\in \Sig.\Fun$ ($k\geq 0$) such
that for all $1\leq i\leq k$ if $S_i$ is a constructor sort, it is non empty.
We say
that $E$ has {\em no empty constructor sorts} iff each constructor sort is non
empty.
\end{Dn}
\begin{Dn}
Let $E$ be a {\tt Specification}. $E$ is called {\it well-formed} iff 
\begin{itemize}
\item
$E$ is SSC, 
\item
$E$ is communication-associative,
\item
$E$ has no empty constructor sorts,
\item
$\Bool\in \Sig(E).Sort$, $T\ap \rightarrow\Bool\in \Sig(E).\Fun$ and
$F\ap \rightarrow\Bool\in \Sig(E).\Fun$,
\item
If $\Time\in\Sig(E).\Sort$, then $\nul\ap \rightarrow\Time\in \Sig(E).\Fun\cup
\Sig(E).\Map$ and $\leq\ap \Time\times\Time\rightarrow\Bool\in
\Sig(E).\Map$. 
\end{itemize}
\end{Dn}

\newpage
\section{ATerm representation format for MTLPEs}
A MTLPE is stored as an ATerm with the following functions. The sort
of stored MTLPE is $\nm{MTLPE}$.

\begin{gather*}
\afunc{spec2gen}{\nm{DataTypes}\x\nm{ActionSpec}^*\x\nm{InitProcSpec}}{\nm{MTLPE}}\\
\afunc{actspec}{\nm{String}\x\nm{String}^*}{\nm{ActionSpec}}\\
\afunc{initprocspec}{\nm{Term}^*\x\nm{Variable}^*\x\nm{Summand}^*}{\nm{InitProcSpec}}\\
\afunc{smd}{\nm{Variable}^*\x\nm{Action}^*\x\nm{Term}\x\nm{NextState}\x\nm{Term}}{\nm{Summand}}\\
\afunc{act}{\nm{String}\x\nm{Term}^*}{\nm{Action}}\\
\afunc{terminated}{}{\nm{NextState}}\\
\afunc{i}{\nm{IndexedTerm}^*}{\nm{NextState}}\\
\afunc{d}{\nm{Signature}\x\nm{Equation}^*}{\nm{DataTypes}}\\
\afunc{e}{\nm{Variable}^*\x\nm{Term}\x\nm{Term}}{\nm{Equation}}\\
\afunc{v}{\nm{String}\x\nm{String}}{\nm{Variable}}\\
\afunc{it}{\Nat\x\nm{Term}}{\nm{IndexedTerm}}\\
\afunc{dc}{\Nat}{\nm{IndexedTerm}}\\
\afunc{s}{\nm{String}^*\x\nm{Function}^*\x\nm{Function}^*}{\nm{Signature}}\\
\afunc{f}{\nm{String}\x\nm{String}^{*}\x\nm{String}}{\nm{Function}}
\end{gather*}
The sort $\nm{Term}$ consists of arbitrary ATerm terms where all function
symbols must be quoted. The sort $\nm{String}$ consists of quoted constants,
i.e.\ function symbols of arity 0. The sort $\Nat$ is the built in
sort of natural numbers in the ATerm library. The list
of elements of sort $D$ is denoted by $D^{*}$.

The constructor of sort $\nm{InitProcSpec}$ contains the actual LPE
parameters (from \texttt{init}) as the first parameter, the formal LPE
parameters as the second argument, and the list of summands as the
third parameter. The third parameter of $\mathsf{cmd}$ is the term of
sort $\nm{Time}$ representing the time at which the multiaction
happens.  The last parameter of $\mathsf{cmd}$ is the boolean term
representing the condition.

The first parameter of $\mathsf{v}$ is the variable name, appended with '\#'.
The first parameter of $\mathsf{f}$ is the function name, appended 
with its parameter types list, separated by '\#' (for constants only '\#' is appended).

If the delta summand of the TLPE is present, $\delta$ has to be
represented by the ATerm string \texttt{"Delta"}, and actions with
this name should not be allowed.  An alternative is in using a special
summand construction.

\newpage
\section{ATerm representation format for LPEs (for \mcrl\ v1)}
An LPE is stored as an ATerm with the following functions. The sort
of stored LPE is $\nm{LPE}$.

\begin{gather*}
\afunc{spec2gen}{\nm{DataTypes}\x\nm{InitProcSpec}}{\nm{LPE}}\\
\afunc{initprocspec}{\nm{Term}^*\x\nm{Variable}^*\x\nm{Summand}^*}{\nm{InitProcSpec}}\\
\afunc{smd}{\nm{Variable}^*\x\nm{String}\x\nm{Term}^*\x\nm{NextState}\x\nm{Term}}{\nm{Summand}}\\
\afunc{terminated}{}{\nm{NextState}}\\
\afunc{i}{\nm{Term}^*}{\nm{NextState}}\\
\afunc{d}{\nm{Signature}\x\nm{Equation}^*}{\nm{DataTypes}}\\
\afunc{e}{\nm{Variable}^*\x\nm{Term}\x\nm{Term}}{\nm{Equation}}\\
\afunc{v}{\nm{String}\x\nm{String}}{\nm{Variable}}\\
\afunc{s}{\nm{String}^*\x\nm{Function}^*\x\nm{Function}^*}{\nm{Signature}}\\
\afunc{f}{\nm{String}\x\nm{String}^{*}\x\nm{String}}{\nm{Function}}
\end{gather*}
The sort $\nm{Term}$ consists of arbitrary ATerm terms where all function
symbols must be quoted. The sort $\nm{String}$ consists of quoted constants,
i.e.\ function symbols of arity 0. The list
of elements of sort $D$ is denoted by $D^{*}$.

The first parameter of $\mathsf{v}$ is the variable name, appended with '\#'.
The first parameter of $\mathsf{f}$ is the function name, appended 
with its parameter types list, separated by '\#' (for constants only '\#' is appended).

The constructor of sort $\nm{InitProcSpec}$ contains the actual LPE
parameters (from \texttt{init}) as the first parameter, the formal LPE
parameters as the second argument, and the list of summands as the
third parameter. The last parameter of $\mathsf{cmd}$ is the boolean term
representing the condition.

\newpage
\section{ATerm representation format for input muCRL (for \mcrl\ v1)}
An LPE is stored as an ATerm with the following functions. The sort
of stored LPE is $\nm{LPE}$.

\begin{gather*}
\afunc{spec2gen}{\nm{DataTypes}\x\nm{InitProcSpec}}{\nm{LPE}}\\
\afunc{initprocspec}{\nm{Term}^*\x\nm{Variable}^*\x\nm{Summand}^*}{\nm{InitProcSpec}}\\
\afunc{smd}{\nm{Variable}^*\x\nm{String}\x\nm{Term}^*\x\nm{NextState}\x\nm{Term}}{\nm{Summand}}\\
\afunc{terminated}{}{\nm{NextState}}\\
\afunc{i}{\nm{Term}^*}{\nm{NextState}}\\
\afunc{d}{\nm{Signature}\x\nm{Equation}^*}{\nm{DataTypes}}\\
\afunc{e}{\nm{Variable}^*\x\nm{Term}\x\nm{Term}}{\nm{Equation}}\\
\afunc{v}{\nm{String}\x\nm{String}}{\nm{Variable}}\\
\afunc{s}{\nm{String}^*\x\nm{Function}^*\x\nm{Function}^*}{\nm{Signature}}\\
\afunc{f}{\nm{String}\x\nm{String}^{*}\x\nm{String}}{\nm{Function}}
\end{gather*}
The sort $\nm{Term}$ consists of arbitrary ATerm terms where all function
symbols must be quoted. The sort $\nm{String}$ consists of quoted constants,
i.e.\ function symbols of arity 0. The list
of elements of sort $D$ is denoted by $D^{*}$.

The first parameter of $\mathsf{v}$ is the variable name, appended with '\#'.
The first parameter of $\mathsf{f}$ is the function name, appended 
with its parameter types list, separated by '\#' (for constants only '\#' is appended).

The constructor of sort $\nm{InitProcSpec}$ contains the actual LPE
parameters (from \texttt{init}) as the first parameter, the formal LPE
parameters as the second argument, and the list of summands as the
third parameter. The last parameter of $\mathsf{cmd}$ is the boolean term
representing the condition.

%\bibliographystyle{plain}
%\bibliography{genspect}

\end{document}
