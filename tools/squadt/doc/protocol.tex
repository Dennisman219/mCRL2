\documentclass{article}
\usepackage[UKenglish]{babel}
\usepackage{pslatex}
\usepackage{graphicx}
\usepackage{calc}
\usepackage{latexsym,amssymb,amsfonts,amsthm,amsmath}
\usepackage{float}
\usepackage{xspace}
\usepackage[top=3cm,left=3cm,bottom=3cm,right=3cm]{geometry}

\title{A Protocol for Interactively Controlling Software Tools\\DRAFT}
\author{J. van der Wulp}

\newtheorem{example}{Example}

\newcommand{\msg}[1]{\texttt{#1}}
\newcommand{\squadt}{SQuADT\xspace}

\makeatletter
\renewcommand{\paragraph}{\@startsection
  {paragraph}%
  {4}%
  {-1.3em}%
  {-\baselineskip}%
  {0.1\baselineskip}%
  {\normalfont\normalsize\textbf}}
\makeatother

\bibliographystyle{alpha}

\begin{document}

\maketitle
\thispagestyle{empty}

 \section{Introduction}

  In September 2005 work started on an integration platform for the tools in
  the mCRL2 toolset, and tools like it. The mCRL2 toolset (see
  \cite{groote_et_al:DSP:2007:862}) is a collection of tools around the formal
  modelling language mCRL2 that can be used for formal verification and analysis.
  Most of the tools have a traditional command line interface and today not
  everyone is comfortable with this way of working.  The goal of the platform
  is to unlock the toolset for a broader audience by making it easier to use
  without having too much knowledge about the specifics of every tool. In
  addition there is other integration-level functionality like automation of
  common tasks often involving multiple tools.
  
  Conceptually this integration framework is a communication layer between
  tools and the user. It provides means to execute and terminate tools and feed
  output of tools as input to other tools. The \squadt desktop application
  represents the interface to the user built on top of the framework. The name
  \squadt, stands for (Systems Quality, Analysis and Design Toolset), which
  refers to the kind of tasks that can be performed with connected tools.
  Communication with the user occurs my means of a graphical user interface, in
  terms of files and tools operating on those files.
  %  Ever since the first version, \squadt communicates with tools using the
  %  communication protocol described in this document.
  
  The current level of integration between tools, as implemented in \squadt, is
  as follows. Tools can be started and stopped on demand and all communication
  between tools is uni-directional and occurs through files. In particular if
  one tool produces a file and another reads it and those tools are never
  running at the same moment. This behaviour is easily explained by looking at
  the mCRL2 tools that only communicate in such a fashion.

  Integration between tools means communication and co-operation between those
  tools. This also means that the tools involved must have functionality that
  can be combined in some useful way. The user is assumed to provide one or
  more files that contain data that serve as a starting point from which to
  apply a combination of tools in order to achieve certain goals. These
  together form a project context from within which \squadt operates.
  In such a context the following are integration level functionality provided by \squadt:
   \begin{itemize}
    \item visual (and interactive) overview of data interdependencies,
    \item guards data consistency,
    \item (semi-)automated task execution.
   \end{itemize}
  The core of the graphical user interface of \squadt is a visualisation of all
  files in a project and the dependencies created by tool execution. The data
  in files is static and tools create the dependency relations between the
  files. The last two points require a more elaborate explanation.
  
  Data is considered consistent when the tool that produced it communicates
  that it has produced a file that contains this data. So establishing data
  consistency is all up to the tool that produces the data. Note that
  consistency can be a relative notion, file $b$ is consistent if it respects a
  certain relation $\gamma$ with file $a$: $a \gamma b$. The functionality that
  \squadt offers in this respect is two-fold. First it prevents that two tools
  write to the same file within the project. Meaning that as long as a tool is
  running that is writing a file this file cannot be read by another tool.
  Second, using file interdependencies, all possible inconsistencies across
  files are taken into account.  If a file $a$ changes, this fact is
  recursively propagated to all files that depend on $a$, and the user is
  notified of the possibly arisen inconsistencies.

  Tools are meant to help perform tasks and must often be configured to do
  those tasks. Most of the time this means reading data from file(s) and
  producing other file(s). The point (semi-)automated task execution means that
  tasks can be repeated without or with minimal user interaction. Once a task
  has been configured, \squadt stores this configuration and it can be used to
  start a tool on-demand and redo the task.

  All of the functionality described above can be obtained by using existing
  tools and their command line interfaces. A non-interactive command line
  interface essentially makes it possible to store task configuration. Some
  problems with this approach are:
  \begin{itemize}
   \item Command line interfaces are not standardised, especially not across
         platforms
   \item Not every tool has a command line interface and does not provide other
         means of automated task configuration
   \item It is unclear what constitutes the output of a tool when it is run,
         for successful integration this fact must be communicated to its
         environment (the integration framework).
  \end{itemize}
  An abstract communication interface can solve all these problems in a more
  elegant fashion. This text introduces such an interface: a communication
  protocol used for controlling tools.

%  In abstract, the platform provides a number of facilities that can be used by
%  tools to communicate with the user. The tools provide services that are made
%  accessible to the user of the platform. The platform acts as a controller
%  (possibly for multiple tools at the same time) and therefore the system will
%  mostly be referred to as (the) controller in the remainder of this document.
%  Similarly, a tool is the communication partner of the controller.

  The remainder of this text is structured as follows. Starting with the next
  section the important concepts are introduced. This give the reader an
  overview of things involved.  This is followed by a high level overview of
  the protocol. It specifies the structure of communication: what is
  communicated and in what way. This is followed by a more detailed description
  of the protocol consisting of design choices along with motivation and
  concrete syntax and semantics of messages by means of examples.

% IDE comparison
% note control aspect: user asserts control over a tool by means of the system

 \section{Concepts}

   This document describes the design of a communication protocol that will be
   used by a system that targets integration of software tools. There are lots
   of examples of software tools that all have different functionality. Not
   every tool can be used in combination with every other tool. For any kind of
   integration some form of communication between tools is required. The system
   has the task to enable and direct the communication between tools.  This
   section functions as introduction to the important concepts on top of which
   the protocol is designed.

  \subsection{System}

   The system is an anonymous software application for interactive software tool
   integration. A key concept is that the system is an intermediary between the
   user and a set of tools that can be controlled. A concrete example of such a
   system is the deskSQuADT application.

   As main task the system is responsible for running tools and directing
   communication between tools. Other tasks are:
   
    \begin{itemize}
     \item help guide users to find tools that offer the desired functionality
     \item and automate frequently (re)occurring tasks that involve running one or more tools
     \item provide a uniform manner for users to interact with tools
    \end{itemize}

   There are some more but the ones mentioned directly affect the design of the
   communication protocol.

  \subsection{Tool}

   A (software) tool is a program that processes input to produce output. The
   input can be anything for example a file in the local filesystem or a
   potentially unbounded stream of data across a network. The output can be
   anything that can be input as well as output on a display or another human
   interface device. An example of the last kind is a program that creates an
   interactive visualisation of the input.
   
   Modern operating systems allow programs to be executed as processes either
   concurrently or in a time-sharing fashion. The consequence of is that a tool
   can be running multiple times at the same moment on the same machine. Notice
   that a tool's input and output function as communication channels between
   processes. % where to put unidirectional requirement?

   Any tool (in the ordinary sense) is always used with a particular purpose in
   mind. Whether the use actually satisfies the purpose depends on the
   capabilities of that tool. A software tool is no exception, it offers a
   limited set of functionality each of which specifies how input will be
   related to output.  The choice on how to use a software tool is the choice
   between available relations between input and output.  The process of making
   a choice is called configuration and that of actually using the tool in the
   selected fashion corresponds to performing a task.

  \subsection{Task \& Task Configuration} \label{concepts::task_configuration}

   A \textit{task} is an indivisible amount of work with some result, that can
   be performed by a single tool. The work consists of processing input and
   producing output according to functionality specified by a task \textit{
   configuration}.  A (task) \textit{configuration} is a specification that
   uniquely defines a task up to user-interaction. Naturally a task
   specification for a tool $t$ is restricted to the set of functionality
   offered by $t$.

   % creation and manipulation (tool and system)
   A task configuration is both created and modified by the tool itself. Both
   input and output of the tool are part of the task configuration. Other than
   that the configuration specifies the relation between input and output.  In
   the case of concrete inputs or outputs that can serve as input, such as
   files the system can change the configuration by renaming files. The system
   must have this capability to be able to perform its main task.
 
   % note on failure
   Since tools are programs, and programs can fail, task execution may fail.
   Nothing can be concluded about a task that failed. If output exists it must
   be assumed corrupted. If tool input and output are always disjoint (i.e. a
   task cannot change input), then it is as if the task was not started at all.

   % note on user interaction
   Some tools require interaction with the user. In such a case a task cannot
   be unambiguously specified. The task configuration only specifies a starting
   point from where to start further configuration. The actions of the user can
   completely change the configuration in any way.

  \subsection{Interaction Display}

   Obtaining a task configuration for a tool is a process that often requires
   user interaction. The system has no knowledge about any particular tool
   other than what may be communicated by means of the protocol. So obviously
   the tool is better equipped to direct the interaction with the user than the
   system is. For communicating with the user the system provides a number of
   facilities. Currently the only available facility is an \textit{interaction
   display} which is a per-tool interactive user interface.

   % added value
   The interaction display can be used for other purposes than configuration.
   The tool controls the contents of the display as long as it runs. One
   example of its use may be showing progress during task execution.

   % added value
   The interaction display is a facility of which the use is optional. A tool
   can very well provide its own graphical user interface for the purpose of
   obtaining a suitable task configuration. The added value of using this
   facility is that it the improves interface uniformity across tools and that
   it also works without effort for tools that run on a different machine and
   communicate across a network.

 \section{Protocol (high level)} \label{s:message_definitions}

  The protocol is defined in terms of a number of typed messages. The semantics
  of a message may depend on the type, the direction in which the message
  travels (controller-to-tool or tool-to-controller) and the messaging context.
  The context of a message $m$ is the ordered list of all sent and received
  messages (assuming strict interleaving) up to the point that $m$ is sent or
  received.

  The protocol behaves asynchronous at any moment a request can arrive for any
  of the communication partners. In particular non of the parties is allowed to
  block. After communication initiation, of which the initiative is at the side
  of a tool, the initiative is determined by which side sends a request.

  Typical patterns are request-response sequences. When one side sends a
  request, say a message with a type $t$, the other side \emph{must} respond
  after it has received with message with an acknowledgement, which is also a
  message of type $t$. After sending the request the sender can wait for the
  response provided that in the meanwhile it sends responds for all incoming
  requests. This avoids deadlock, i.e. both sides waiting on response for
  requests to one another.

  \subsection{Instance identification} \label{s:instance_identification}

   When a tool is started it will initiate communication with a controller.
   The details on how the communication can be initiated is somehow
   communicated outside this protocol. On new incoming communication the
   controller has no way of knowing what lives on the other side of the
   connection. The tool should identify itself by presenting a pre-communicated
   token. The first message after a connection has been established must be
   structured as follows.
   
   \begin{figure}[H]
    \begin{center}
     \begin{tabular}{|ll|}
      \hline
      message type:    & \msg{identification} \\
      \hline
      direction:       & tool to controller \\
      data:            & token \\
      \hline
     \end{tabular}
    \end{center}
   \end{figure}

   \noindent When started the tool is given a token which it should sent as
   part of this message. The controller must sever the connection if the first
   message is not an identification message, or if the token was not among
   those that were expected. It is assumed that the controller has knowledge
   about what tools are expected to connect and each of them can be identified
   by their token.  Messages of this type must be ignored if they are not the
   first message after the connection has been established.

  \subsection{Capabilities}

   Exchange of capabilities represents a manner of communicating available
   functionality between the communication partners.  Such functionality can be
   used to offer implementation dependent functionality that is not (or not
   yet) prescribed by the protocol. Only the basic structure of the message is
   prescribed, extensions are determined by anyone who implements it. This is the
   recommended way of adding implementation dependent extensions to the
   protocol functionality without changing the protocol specification.

   A useful example of an implementation dependent extension is adding a new
   control for the tool display. The tool can check the capabilities of the
   controller to see whether the control is available. If it is available it
   can use it just like any other control in the tool display (see section
   \ref{ss:user_interaction}).

   The extension mechanism works the same in both ways (controller-to-tool or
   vice versa). The whole thing basically is a request-response sequence,
   where the request is fixed and the response can be arbitrarily complex. The
   request represents a query for the other side's capabilities. The response
   is the combined set of capabilities. A request can be sent from any context
   and has the following structure.

   \begin{figure}[H]
    \begin{center}
     \begin{tabular}{|ll|}
      \hline
      message type:    & \msg{capabilities} \\
      \hline
      data:            & none \\
      \hline
     \end{tabular}
    \end{center}
   \end{figure}

   The controller, when it receives a capabilities request must respond with a
   message that contains at least the following information.

   \begin{figure}[H]
    \begin{center}
     \begin{tabular}{|ll|}
      \hline
       message type:   & \msg{capabilities} \\
      \hline
       direction:      & controller to tool \\
       data:           & Version = struct major : Nat $|$ minor : Nat \\
      \hline
     \end{tabular}
    \end{center}
   \end{figure}

   \noindent The tool, when it receives a request for capabilities must also respond with
   a similar message. The difference with the response from the controller is
   in the fact that the tool must advertise its functionality. To operate, the
   tool requires input. Based on what the input represents (characterised by a
   format), it offers functionality (characterised by a category).
   
   \begin{figure}[H]
    \begin{center}
     \begin{tabular}{|ll|}
      \hline
       message type:   & \msg{capabilities} \\
      \hline
       direction:      & tool to controller \\
       data:           & version $\times$ List (input-configuration) \\
                       &  version             = struct major : Nat $|$ minor : Nat \\
                       &  input-configuration = category $\times$ List (id $\times$ format) \\
                       &  category            = String,   a name that characterises the function of the tool \\
                       &  format              = storage format specifier for the tools main input \\
      \hline
     \end{tabular}
    \end{center}
   \end{figure}

   \noindent The combination of a category and a list of pairs of a unique name and a
   format is called an \textit{input configuration} of the tool.  The available
   input configurations, in a rather abstract fashion represent the range of
   functionality offered by the tool.  An input configuration can also be
   considered as a template for a configuration (see section
   \ref{ss::tool_configuration}). The controller may use the category names
   in an input configuration to categorise tools based on functionality. The
   pairing of a name and a format represents a name for a typed input slot. The
   idea is that to make use of functionality associated with a particular input
   configuration all of the inputs need to be assigned a valid data source
   (e.g. files).

  \subsection{Tool Configuration} \label{ss::tool_configuration}

   The controller makes use of the services provided by a tool through task
   configuration. After that the task that is performed makes use of a
   selection of the functionality that the tool provides.  Configuration and
   task execution are separated, the latter is discussed in section
   \ref{ss::task_execution}.

   Tool configuration is a two step process. First the controller sends a
   configuration. The tool inspects this configuration and responds with a
   verdict on whether or not the configuration is valid. A valid configuration
   has the property that it can be used to successfully configure the tool. A
   configuration message may be sent from any context and has the following
   structure:

   \begin{figure}[H]
    \begin{center}
     \begin{tabular}{|ll|}
      \hline
       message type:   & \msg{configuration} \\
      \hline
       direction       & controller to tool \\
       data:           & fresh $\times$ Configuration \\
                       & \ Configuration = category $\times$ List (option) $\times$ List(object) \\
                       & \ \ fresh   = Bool \\
                       & \ \ option  = id $\times$ List(data-type $\times$ String) \\
                       & \ \ object  = id $\times$ io-type $\times$ format $\times$ URI \\
                       & \ \ io-type = struct input $|$ output \\
      \hline
     \end{tabular}
    \end{center}
   \end{figure}

   \noindent The message describes a freshness marking, and a configuration. A
   fresh configuration is a configuration that is generated by the controller
   using information from capability messages. The configuration itself
   consists of a category name, a list of options and a list of input/output
   objects. The category name has a cosmetic function only; it describes the
   functionality the tool will use to perform the task. The list of options,
   represents the total of options as well as values for possible arguments to
   those options.

   Before sending any response the tool may initiate user interaction (see
   subsection \ref{ss:user_interaction}) in order to refine the configuration.
   The second step of configuration is the response message. This response
   signals whether a configured state can be created from the configuration.
   The response message has the same structure structure as the request:

   \begin{figure}[H]
    \begin{center}
     \begin{tabular}{|ll|}
      \hline
       message type:   & \msg{configuration} \\
      \hline
       direction       & tool to controller \\
       data:           & valid $\times$ Configuration \\
                       & \ valid = Bool \\
      \hline
     \end{tabular}
    \end{center}
   \end{figure}

   \noindent The only difference between the request and the response is
   the validity marking. The message indicates by means of the validity marking
   whether the embedded configuration is valid. A configuration is called
   accepted if a response marks it as valid.

   It is assumed that the embedded configuration has a strong resemblance to
   the configuration embedded in the request. There is only one restriction,
   the category embedded in both configurations must be the same in both the
   request and the response. The underlying idea is that the tool would start
   with the configuration from the request, determine its validity and possibly
   refine it by consulting the user. How much the configurations in the request
   and response messages resemble each other depends on the creativity of the
   tool developer.

   The controller needs a configuration tailored for that specific tool for a
   configuration request message.  The controller can obtain configurations in
   two ways.  It can generate a configuration from a single input configuration
   (as received from a previous \msg{tool\_capabilities} message). In this case
   the configuration must be marked as fresh. Alternatively the controller can
   use any configuration for a tool that was previously considered valid by
   that tool.

%   \noindent About the configuration specifications: every option is uniquely
%   identified, and so is every object. The reason is to make it easier for a
%   tool developer to test for availability of options/objects.  An option
%   represents an atomic unit in the configurable behaviour of a tool. For
%   validation purposes a data type can be specified against which the values
%   for the option are matched. An object is a file associated with a format and
%   a location.

  \subsection{Task Execution} \label{ss::task_execution}

   \noindent Tool configuration is complete when the controller receives a
   configuration response message with an accepted configuration and it has not
   sent a new configuration request.  When configuration is complete the
   controller may start task execution with message with the following
   structure: 

   \begin{figure}[H]
    \begin{center}
     \begin{tabular}{|ll|}
      \hline
       message type:   & \msg{task\_start} \\
      \hline
       direction       & controller to tool \\
      \hline
     \end{tabular}
    \end{center}
   \end{figure}

   \noindent On receipt, a tool must start executing a task based on the last
   accepted configuration. The controller has the responsibility to ensure that
   the input objects in a configuration exist prior to starting a task that
   depends on them, and ensure that they remain unchanged (by the environment)
   during task execution. Likewise, the output objects must not exist or be
   modifiable by the tool prior to starting a task, and it must ensure that the
   environment leaves the outputs unchanged during task execution.

   When a task has been completed the tool must send a response to the
   controller that signals this fact. From a context where a \msg{task\_start}
   has been received and no subsequent \msg{task\_done} has been sent a tool
   may send a message with the following structure:

   \begin{figure}[H]
    \begin{center}
     \begin{tabular}{|ll|}
      \hline
       message type:   & \msg{task\_done} \\
      \hline
       direction       & tool to controller \\
       data            & result : Bool \\
      \hline
     \end{tabular}
    \end{center}
   \end{figure}

   \noindent which signifies that execution of the task has finished. The
   result represents success or failure in task execution.

  \subsection{Tool Termination}

   \noindent The controller can request a tool to terminate. This facility is
   present to allow tools to free resources and remove outputs or leave
   them in a consistent state.

   \begin{figure}[H]
    \begin{center}
     \begin{tabular}{|ll|}
      \hline
       message type:    & \msg{termination} \\
      \hline
       direction:       & controller to tool \\
       data:            & none \\
      \hline
     \end{tabular}
    \end{center}
   \end{figure}

   \noindent As a response the tool should sent a message with the following
   structure:

   \begin{figure}[H]
    \begin{center}
     \begin{tabular}{|ll|}
      \hline
       message type:   & \msg{termination} \\
      \hline
       direction:      & tool to controller \\
       data:           & none \\
      \hline
     \end{tabular}
    \end{center}
   \end{figure}

   If, after a reasonable amount of time has past without a response, the
   controller may terminate the tool through other means.

  \pagebreak

  \subsection{User interaction} \label{ss:user_interaction}

   For user interaction a tool depends on the facilities provided by the
   controller. The controller shows a display to the user on behalf of each
   tool.  A tool can communicate with a user by listening and responding to
   user-interaction with the display. The controller captures and relays all
   user interaction with the display to the tool. A tool can either alter the
   content of the display entirely or update the internal state of some
   controls it contains.

   To use the display, a tool must send a valid layout specification first.
   The controller can use it to fill the display. The layout specification
   describes the controls and layout constraints that must be observed for
   laying them out on the display. Every time a layout specification is sent in
   this fashion it replaces the contents of the display. From any context the
   a message structured as follows may be used communicates a layout specification.
%\footnote{It would be nice if these controls are not part of the protocol but added
%   through the extension mechanism (because they are highly implementation
%   specific)}. 
   
   \begin{table}[H]
    \begin{center}
     \begin{tabular}{|ll|}
      \hline
       message type:   & \msg{display\_layout} \\
      \hline
       direction:      & tool to controller \\
       data:           & element = struct box(direction $\times$ List(properties $\times$ id $\times$ element)) $|$ control \\
                       & \ direction = struct horizontal $|$ vertical \\
                       & \ properties = struct visibility $|$ input $|$ margin $|$ alignment \\
                       & \ \ visibility = struct visible $|$ hidden \\
                       & \ \ input = struct enabled $|$ disabled \\
                       & \ \ margin = struct left $|$ top $|$ right $|$ bottom \\
                       & \ \ alignment = struct horizontal(struct left $|$ center $|$ right) $|$ \\
                       & \ \                    vertical(struct bottom $|$ middle $|$ top) \\
                       & \ control = progress\_bar $|$ radio\_button $|$ button $|$ label $|$ text\_field $|$ checkbox \\
                       & \ \ progress\_bar = Nat $\times$ Nat $\times$ Nat \\
                       & \ \ radio\_button = Bool \\
                       & \ \ button        = String \\
                       & \ \ label         = String \\
                       & \ \ text\_field   = String \\
      \hline
     \end{tabular}
    \end{center}
   \end{table}
   \vspace{-0.4cm}
   \noindent Every layout element has a unique identifier that can be used in
   subsequent messages to signal or change the state of a control. A layout is
   specified as a nested box elements that contain controls. Each control is
   associated with a set of layout properties that affect the way in which the
   control is placed on the display. The horizontal or vertical direction of a
   box determines the way it lays out the elements it contains above or beside
   each other respectively.

   The layout properties further affect layout of elements relative to each
   other or the containing box. Visibility determines whether a layout element
   is visible or not. Input determines whether a control is active, i.e.
   whether it responds to user interaction. Margins control the distance
   between the directly adjacent elements. For first and last elements this
   means the distance to the borders of the containing box. A box equally
   divides the amount of available space over the available controls. When
   there is plenty of space after deduction of margins the alignment can be
   used to control either the vertical or horizontal position of a control
   within the available space.

   The following message is used to communicate changes to a control on the
   display.

   \begin{figure}[H]
    \begin{center}
     \begin{tabular}{|ll|}
      \hline
       message type:   & \msg{display\_data} \\
      \hline
       direction:      & both \\
       data:           & id $\rightarrow$ state \\
      \hline
     \end{tabular}
    \end{center}
   \end{figure}
   \vspace{-0.4cm}
   \noindent The interpretation of the message depends on the direction of the
   message. When a tool receives the message it is interpreted as a change in
   the state of a control on the display. When the controller receives such a
   message it is interpreted as a request to change the state of a control on
   the display. If the id is unknown at the receiving side then the message
   must be ignored.
%    A tool can change the state of controls on a display by means of their
%    identifier. The following message must be used for this purpose.

  \subsection{Tool Report}

   From any context a tool can send a report that signifies a warning, error or
   just some information.

   \begin{figure}[H]
    \begin{center}
     \begin{tabular}{|ll|}
      \hline
       message type:   & \msg{report} \\
      \hline
       direction:      & tool to controller \\
       data:           & struct notice $|$ warning $|$ error $\times$ description \\
      \hline
     \end{tabular}
    \end{center}
   \end{figure}
   \vspace{-0.7cm}
   \noindent The controller will probably pass it through to the user but
   it has no effect on protocol state. Before termination a tool can report to
   the controller about what has been done. This report can contain anything
   from general information to errors. A controller can display this in some
   way to the user.

 \section{Implementation} \label{s:protocol_implementation}

   Section \ref{s:message_definitions}, presented a high level view of the
   protocol. The main characteristics are that all messages are typed, and that
   the origin of the message (or the direction in which it travels) determines
   how it should be interpreted. Also significant, but not explicitly mentioned
   is that message order is preserved. Other important functionality is
   authentication, which was already made a part of the protocol.
   Alternatively it is possible to rely on another protocol with such
   functionality.

   In terms of the seven layers of the OSI model \cite{Day1983}, our protocol
   only covers the 6th (presentation) layer. All functionality below the
   presentation layer will be provided through the use of standard protocols
   (see section \ref{ss:transport}).
   
  \subsection{Transport} \label{ss:transport}

   For the actual transport of data the TCP/IP protocol suite is the best
   contender. It satisfies all requirements discussed previously. More
   specifically, it provides: a state-full connection which allows
   authentication at initiation, and data is delivered in the order in which it
   was offered.  Furthermore TCP/IP is available on many platforms and usable
   from many different programming languages.

   A possible alternative is to use standard input/output streams (or piping).
   But the main disadvantage of this facility is that there is not built-in
   support in many programming languages to do non-blocking communication. This
   is a strict requirement because of the asynchronous nature of the protocol.
   A further disadvantage over TCP/IP is that all communication is limited to
   the same machine.

%   The TCP/IP protocol suite is chosen as the recommended means of
%   transport. It is well-known and supported by a lot of programming languages.
%   The protocol requires connection-state, a tool is authenticated once and on
%   failure the connection is terminated. In addition, because meaning is
%   assigned to the order of messages, the protocol requires that messages are
%   delivered in the same order as which they were offered to the sender. All
%   requirements are all supported by TCP/IP, note that for UDP/IP this is not
%   the case.

  \subsection{Messaging}

   In the previous subsection TCP/IP was established as a good candidate for
   data transport. All that is needed now is a messaging mechanism for
   communication of arbitrary content between the communication partners. We
   have been searching for XML-based protocols. The use of XML is widespread
   with an enormous variety of tools and libraries for many existing
   programming languages. This should make it easy to add the tool-side
   functionality of the protocol to tools written any popular programming
   language.
   
   An existing protocol that offers a messaging mechanism and as added bonus
   also offers authentication is XMPP core (Extensible Messaging and Presence
   Protocol, \cite{rfc3920}).  It is a simple communication protocol that that
   relies on TCP/IP to transport two XML (see \cite{Sperberg-McQueen:06:EML})
   streams (one for each direction).

   The only disadvantage of XMPP core is that no suitable standalone
   implementation seems to be available that suits all of our needs. Creating
   our own XMPP core implementation would take too much time, especially the
   authentication would require a substantial amount of time. We are more
   interested in the results obtained from using this protocol in the context
   of the interactive tool integration system. Instead of using XMPP core
   directly a very similar looking messaging mechanism will be developed in the
   next section.


%   XMPP has the notion of a message that can be sent from client to client, on
%   top of it our protocol can be implemented.   Unfortunately no single
%   client/server side implementations exist at the time that are usable on all
%   target platforms without also introducing quite a number of other
%   dependencies. Creating our own implementation would have taken to much time,
%   especially when we would have implemented all of XMPP.  So we choose a lightweight
%   custom implementation, instead of using XMPP with 3rd-party client and
%   server implementations.

%   The choice for XML is obvious we aim for extensibility and XML to some
%   extend allows changes to the format extensions to the format there is wide
%   support in many programming languages. With XML it is possible to create a
%   good parser that is to some extent resistant to extensions, meaning it will
%   also work on future versions of the format that may contain additional
%   information.  Another benefit of using XML is it makes the messages
%   readable, making it easy to print and manually verify the structure and
%   contents.

   \subsubsection{Structure of a Message} \label{ss:structure}

    \noindent Messages are wrapped in the \textbf{message} element. A mandatory
    attribute is \textbf{type} that specifies the type of the message. The type
    attribute can occur only once. For example:
 
    \begin{verbatim}
     <message type="termination"><![CDATA[message content]]></message>\end{verbatim}

    The content of a message is wrapped as so-called CDATA section (XML
    construct).It means that the data inside is not interpreted when parsing a
    message. The valid message types are: \textit{identification},
    \textit{capabilities}, \textit{configuration}, \textit{display\_layout},
    \textit{display\_data}, \textit{termination}, \textit{task\_start},
    \textit{task\_stop}, \textit{report}.

  \subsection{Message Content}

   The structure of an XML document is usually specified using XML Document
   Type Declaration (\cite{Sperberg-McQueen:06:EML}) or the XML Schema standard
   \cite{Malhotra:06:XSP}. Many people find both difficult to read and understand.
   So the translation of the functional specification of the previous section to
   XML is sketched by means of examples.

  \subsubsection{Authentication}

   Instead of a full-featured authentication scheme we chose a simple instance
   identification scheme. Of course security is something that should be part
   of the design of the protocol. Security is not a main concern, so it is
   postponed until it can be added as an extension somewhere in the future.
   
   The only authentication that is of importance to the functioning of the
   system that uses the protocol is identifying the peer as one of the tools
   that was started. For this purpose the instance identification message,
   described in section \ref{s:instance_identification} was devised. When
   starting a tool the system must somehow pass the tool a token that can be
   used later to uniquely identify it.

  \subsubsection{Capabilities}

   Messages for communicating the capabilities of each of the communication
   partners contain at the very least the version of the protocol. For example:

    \begin{verbatim}
     <capabilities>
      <protocol-version major="1" minor="0"/>
     </capabilities>\end{verbatim}

   A capabilities message sent in reply to a request for capabilities by the
   controller contains the input-configurations in addition to the version
   number. So in addition the \textbf{capabilities} element may contain any
   number of \textbf{input-configuration} elements as follows:

    \begin{verbatim}
      <input-configuration category="" format="" id=""/>\end{verbatim}

   All attributes: \textbf{category}, \textbf{format} and \textbf{id} are
   mandatory.  The \textbf{category} attribute is an arbitrary string which
   represents a name that characterises functionality targeted by the input
   configuration. The \textbf{id} attribute represents a unique identifier for
   an input object in a future configuration. The \textbf{format} attribute
   specifies the format of this object as a MIME type (Multipurpose Internet
   Mail Extensions, \cite{rfc2822}).

   A valid (initial) configuration can be obtained from an input configuration
   as follows.  Create a \textbf{configuration} element, add a Boolean
   attribute \textbf{fresh} set to true and a as child element add a single
   object (see section \ref{ss:implementation_configuration}) of type input
   with the same values for the \textbf{id} and \textbf{format} attributes as
   the \textbf{input-configuration} element.

  \subsubsection{Configuration} \label{ss:implementation_configuration}

   The purpose of a configuration specification is to differentiate different
   behaviours of a tool and to be able to make a selection of the desired
   behaviour. Of course it depends on the developer of a tool, whether
   different behaviour can be observed from the outside. It also depends on the
   developer to what extend this behaviour can be selected before task
   execution is started.

   Programs with command line interfaces take arguments that are used to
   (re)produce this state non-interactively by means of a single string. The
   model behind this method of expressing a configuration is based on the idea
   that tasks consist of a sequence of operations each of which can be
   parametrised. A configuration is a selection between available operations
   and values for the parameters for each of these operations. On the command
   line such a configuration is represented as a string that consists of
   so-called options followed by values for the arguments of this option. We
   chose to reuse this model but will not use the same implementation.

   The controller is assumed to be oblivious to any data that is produced by
   tools other than the fact that there is a name associated to this data and
   the `type' of the data.  In particular the controller cannot inspect the
   data in order to obtain more knowledge about the data.

   The controller is the only party with knowledge about available data
   sources. This knowledge is obtained from configurations. By means of a
   configuration a tool informs the controller on what data (and its type) are
   produced by the task. To represent the type of the data the MIME standard is
   used. The use of MIME is wide-spread among desktop applications and in other
   communication protocols. The application of a tool to do a specific task is
   restricted by the kind of data the task requires as input.

   The structure of the data contained in a configuration message was specified
   at a higher level in the previous section. The translation to XML is as
   follows:

    \begin{verbatim}
     <configuration fresh="true" category="debugging">
      <option id="-v">
       <argument type="integer">1</argument>
      </option>
      <object id="in" type="input" location="" format=""/>
      <object id="out" type="output" location="" format=""/>
     </configuration>\end{verbatim}

   \noindent If the configuration does not contain the \text{fresh} attribute
   it is assumed to be false. A \textbf{configuration} element can contain an
   arbitrary number of \textbf{object} and \textbf{option} elements in any
   order. The \textbf{object} and \textbf{option} elements represent data
   sources and objects respectively.  The \textbf{id} attributes of
   \textbf{option} and \textbf{object} elements represent the identifiers and
   all must be unique within the containing \textbf{configuration} element.  

   An \textbf{option} element may contain an arbitrary number of
   \textbf{argument} elements that each represent a single typed-argument to
   the option. A number of predefined types is available for arguments to
   options: string, enumeration, integer, natural, positive, real.
%   The types are used later to relieve the tool developer from having to ensure
%   that the values that a user inputs as a string have the correct pattern.

   An \textbf{object} element must contain the \textbf{type} attribute, which
   specifies whether the tool takes it as input or produces it as output. The
   \textbf{location} attribute must specify a URI (see \cite{rfc3305}),
   and the \textbf{format} attribute contains a data format specifier
   using the MIME standard.

  \subsubsection{Display}
   
   Display manipulation is limited to replacing the entire content of the
   display at once or modifying the state of controls on the display.  In
   particular it is not possible to manipulate the layout itself. Design of a
   good user interface may require more flexibility such as separate
   manipulation of layout properties.  A good example is the combination of
   HTML, CSS, and JavaScript because it essentially offers this flexibility by
   means of open standards. The complexity of these standards make them
   unsuitable to rely on at this time.

   We have chosen a limited subset of controls that can be used by tools in
   display layouts. Layout construction itself is left very simple on purpose.
   The downside of this choice is that a tool developer has less control over
   the exact way a layout looks on the display. Within this context it has
   always been the intention of making it easy to add new controls to the
   repertoire. A detailed treatment of layout structure and the available
   controls follows.
   
   The available elements are: \textbf{box-layout-manager}, \textbf{progress-bar},
   \textbf{radio-button}, \textbf{button}, \textbf{label}, \textbf{checkbox},
   \textbf{text\_field}.  A detailed introduction of the controls is next
   followed by a description on how the controls are positioned by means of a
   layout specification.

   \paragraph{Controls}

   It is assumed that the reader is familiar with each of the controls their
   basic function and purpose. The name and function of controls and the
   way a layout is built are based on concepts and terminology used in Java
   Swing.

   The \textbf{id} attribute is mandatory for all controls and must be unique
   within the scope of the containing \textbf{display-layout} element
   (introduced later).

   The following example shows a specification for a label, a button and a
   checkbox with text Cancel.

    \begin{verbatim}
<label id="x"><![CDATA[Cancel]]></label>
<button id="y"><![CDATA[Cancel]]></button>
<checkbox id="y" checked="true"><![CDATA[Cancel]]></checkbox>\end{verbatim}

   \noindent When the button is pressed or the checkbox is toggled this fact is
   communicated by sending a \msg{display\_data} message with the complete
   respective button or checkbox specification. The checkbox has a description
   and is always in one of two states: checked or not. A more complex example
   is the radio-button control, because it is not a stand-alone control.

    \begin{verbatim}
<radio-button id="x" connected="y"><![CDATA[first]]></radio-button>
<radio-button id="y" connected="x" selected="true"><![CDATA[second]]></radio-button>\end{verbatim}

   \noindent Radio buttons are always grouped and only a single button in the group is
   selected (pressed).  By default the first radio button is selected, if
   another radio button should be selected then the \textbf{select} attribute
   must have value true. The radio buttons are connected by means of the
   \textbf{connected} attribute that contains the value of the \textbf{id}
   attribute of another radio button in the group. Every radio button in the
   group can be found by repeatedly following the \textbf{connected} attribute
   to connected radio-button elements from any given \textbf{radio-button}
   element in the group. If the selection is changed then only the
   specification of the radio button that gets selected needs to be send by
   means of a \msg{display\_data} message to inform the other side of this
   event.

    \begin{verbatim}
  <text-field id="x"><text><![CDATA[100]]><text></text-field>\end{verbatim}

   \noindent The text field displays an input control for the user to input text. It is
   like the button control, but it contains a child element \textbf{text} that
   holds the content of the control. The reason for this difference is that
   input validation may be added in the future. The system then has the means
   to check and inform the user whether the data entered by the user matches
   the expectations of the tool developer. For example if the input box should
   contain a number then it can be automatically checked to not contain
   non-digit characters.

   The progress bar is used to show progress to a user. It models progress by
   means of a sub range of the integer domain, specified by a minimum and
   maximum value and shows progress by colouring part of this domain up to some
   `current' value that \emph{must} be in the domain $[ minimum \ldots
   maximum ]$. The example is self explanatory:

    \begin{verbatim}
  <progress-bar id="x" minimum="10" maximum="20" current="15"/>\end{verbatim}

   \noindent Updates to the state of a control are specified in the same way as in the
   layout specification. The \textbf{id} attribute identifies the control of
   which the state is to be updated. The attributes then specify the new value
   for the attribute with the same name and child elements specify other
   aspects of the state. When attributes are missing, the value remains
   unchanged.

   \paragraph{Layout}

   A display layout specification is represented by a \textbf{display-layout}
   element that contains a single \text{layout-manager} element.

    \begin{verbatim}
     <display-layout>
      <layout-manager>
       <box-layout-manager variant="vertical" id="">
        ...
       </box-layout-manager>
      </layout-manager>
     </display-layout>\end{verbatim}

   \noindent The layout manager specifies the way in which elements are laid out across
   the display.  Child elements are laid out on the screen horizontally, or
   vertically and are expanded to fill space. The \textbf{box-layout-manager}
   has a \textbf{variant} attribute that specifies the direction in which the
   elements directly contained in it are laid out on the available space. The
   layout managers can be nested for more control on relative position of
   elements.

   A number of layout properties provide more control over how elements are
   positioned and if they are visible and functional (enabled/disabled). The
   available properties are: alignment, margins (top, right, bottom, left),
   vertical alignment (top, middle, bottom), horizontal-alignment (left,
   center, right), element visibility and element activity. Every element that
   is a child of a layout manager is associated with a value for each of the
   layout properties.  Having an explicit value for each of the properties for
   each element in the specification would drastically increase its size, so a
   set of implicitly assumed default values are chosen.  The default properties
   are: alignment is left, no margins, elements are enabled and visible. The
   effective properties of an element are relative to that of the previous
   child. For example:

    \begin{verbatim}
 <box-layout-manager variant="vertical" id="">
  <properties margin-top="1" margin-bottom="1" horizontal-alignment="right">
  <button>Ok</button>
  <properties>
  <button>Cancel</button>
 </box-layout-manager>\end{verbatim}

   \noindent The layout properties for both buttons are the same, top and bottom margins
   are one pixel, vertical alignment is middle and horizontal alignment is right
   and both elements are visible and enabled. The \textbf{properties} element
   directly preceding a control defines the properties for that control. If
   there is no properties element then the previous \textbf{properties} element
   within the same layout manager defines the layout properties for that
   element.

  \subsection{Extensibility}

   Protocol extension is performed by increasing the major version number and
   introducing the changes. The increase in version number is supposed to make
   it easy to test whether additional functionality with respect to previous
   versions is available and/or to signal compatibility mode. The nature of the
   changes is not described and therefore not restricted. Changes can be
   anything but only proper extensions are recommended.
\pagebreak
  \section{Conclusion}

   The purpose of the protocol is make it possible for a user to control of a
   tool through facilities offered by a separate system that acts as
   intermediary. The protocol describes how a tool can be configured to perform
   a task, how it can be made to perform this task and report the results.  A
   task configuration comes into being by interaction with the user through the
   system. The protocol describes this process.  The most important
   functionality offered by the protocol is repeatability of the configuration
   process based on a previous configuration.
   
   The deskSQuADT application currently uses the protocol as its only method
   for controlling tools. The most important features provided by the current
   version are:
    \begin{itemize}
     \item dependencies generated by application of tools on files are
     visualised
     \item change propagation through (semi-)automated task execution for
     repeating tasks; changes in input are detected and tools are re-executed
     on request to ensure up-to-date outputs
     \item data consistency is guarded by avoiding concurrent execution of
     tasks that share inputs or outputs
    \end{itemize}
   All of these features are the result of functionality purposefully built
   into the protocol.
   
   Experience so far has told us that the current ability to fill and
   manipulate the display is rather limited. Interaction with the user would
   improve with a broader choice in controls and more fine-grained control over
   the layout. A good example of useful additional control over layout would be
   hiding or disabling controls in a layout when they are not needed.

%   Since the decision was made to create a custom implementation for the
%   protocol, see section \ref{s:protocol_implementation}, the XMPP protocol has
%   been formalised by the Internet Engineering Task Force (IETF). This means it
%   is now an open internet standard. Because he notion of a message is
%   approximately the same for the protocols, It was (and remains) an option to
%   use our protocol on top of XMPP.

%   It is still interesting to consider implementing this protocol on top of
%   XMPP. The latter has additional functionality that can be used to help solve
%   other tool integration problems. An example of this is active communication
%   between multiple tools through a publish-subscribe mechanism.  A future
%   extension to the protocol or perhaps even another protocol can offer such
%   functionality.

  \enlargethispage*{4pt}
  \bibliography{references}

\end{document}
