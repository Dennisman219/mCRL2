%------------------------------------------------------------
%                          LaTeX slides file
%                               ---
%                              Usenko
%                            LaQuSo, TUE
%                         October 1, 2004
%%%%%%%%%%%%%%%%%%%%%%%%%%%%%%%%%%%%%%%%%%%%%%%%%%%%%%%%%%%%%%%%%%%%%%%%%%%%%%%%
%
%-----------------------------------------------------------------------------------------------------------------
% Use slifonts.
\PassOptionsToPackage{coloremph,colormath,colorhighlight,lightbackground}{texpower}
\RequirePackage{tpslifonts}

% Input the generic preamble.

\input{__TPpreamble}
\hypersetup{pdftitle={texpower highlighting example}}

% The package soul is needed for \highlighttext to work.

\usepackage{soul}

%\backgroundstyle{vgradient}

%-----------------------------------------------------------------------------------------------------------------
%

%\documentclass[fleqn,17pt]{article}
%\RequirePackage[T1]{fontenc}
%\usepackage[usenames]{color}
%\usepackage[screen,sectionbreak]{pdfscreen}
\usepackage{amsmath,amsfonts,amssymb}
%\usepackage{textcomp,txfonts}
\usepackage{textcomp,stmaryrd}
%\usepackage[usenames]{color}
%\usepackage{fixseminar}
%\usepackage[ps2pdf]{hyperref} % remove for printing
\usepackage{graphicx}

%\usepackage{tpslifonts}
%\usepackage{geometry}
\usepackage{mymath,mcrl}

\hypersetup{
  pdfauthor={Yaroslav S. Usenko},
  pdftitle={Timed Extensions of muCRL},
  pdfsubject={Timed Extensions of muCRL},
  pdfkeywords={algebraic specification, real time, verification},
  pdfpagemode={FullScreen}
}

%\bottombuttons
%\margins{2ex}{2ex}{2ex}{2ex}
%\screensize{6.25in}{8in}
%\pagenumbering
%\overlay{tuebgkuk}

%\usepackage[display,coloremph,colormath,colorhighlight]{texpower}
%\usepackage{tpslifonts}

\DeclarePanel{bottom}{\hfill\thepage}
\backgroundstyle{plain}

%\renewcommand{\slideleftmargin}{.5cm}
%\renewcommand{\sliderightmargin}{.5cm}
%\renewcommand{\slidetopmargin}{.5cm}
%\mklength{\slidetopmargin}{\bottompanelheight*\ratio{.2cm}{\semcm}+.5cm}

\title{{\large *}{\Huge Real{\large *}-time-based extensions of \mcrl:}\\[.3ex]
\color[rgb]{0,.4,.3}{\LARGE current status}\\
\color[rgb]{0,.4,.3}{\LARGE possible developments}\\
\color[rgb]{0,.4,.3}{\LARGE place in the context}}
\author{\Large\color[rgb]{0,.4,0}{Yaroslav S. Usenko}}
\date{\large\color[rgb]{.4,.4,0}{26 January 2005}}

%\leftfooter{Y.S. Usenko}
%\MyLogo{}
%\rightfooter{\thepage}

%\setlength{\mathindent}{0.5em}
%\setlength{\parindent}{0em}


\newcommand{\xto}[1]{\xrightarrow{#1}}
\newcommand{\restrict}{\upharpoonright}
\newcommand{\tshift}{\mathbin{\uparrow}}
\newcommand{\timeshift}[2]{{#1}\tshift{#2}}
\newcommand{\superps}{\oplus}
\newcommand{\csins}{\mathbin{\mathop{^\curvearrowright}}}
\newcommand{\delay}[1]{\mathop{\sigma_{\!\!\nm{rel}}^{#1}}}
\newcommand{\cond}{\mathbin{\mathop{:\rightarrow}}}
\newcommand{\tpref}[2]
        {\left[
                {#1}
        \right]_{#2}}
\newcommand{\superpos}[1]    {\mathop{\underset{#1}{\bigoplus}}}

\begin{document}
\begin{slide}
%###################################################
%###################################################
\phantom{}
\vspace{3ex}

\maketitle
%%###################################################
%\newslide
%\tableofcontents

%###################################################
\newslide\section*{Motivation}
\begin{itemize}
\item A large number of modeling formalisms
  \begin{itemize}
  \item PA: CSP, CCS, ACP, LOTOS, \mcrl
  \item Time: timed versions of each, TA
  \item Stochastic: Spades, IMC, IGSMP, MoDeST
  \item Hybrid: HyPA, ACP$^{srt}_{hs}$, BHPC, hybrid $\chi$
  \end{itemize}
\item Each formalism requires learning
\item Little compatibility
  \begin{itemize}
  \item little reuse of algorithms
  \item difficult to compare the results
  \end{itemize}
\end{itemize}
%###################################################
%###################################################
\newslide\section*{Labeled Transition Systems}
Labeled Transition System (LTS): directed graph;
vertexes -- states, arcs -- actions.
\pause
\begin{center}
\includegraphics[width=\textwidth]{LTSs}
\end{center}
\pause
Properties of LTSs are expressed in temporal logics.
%###################################################
%###################################################
\newslide\section*{\mcrl\ Language}
\mcrl\ based on ACP and abstract data types.
\pause
Process expressions:
\begin{gather*}
p\syntdef\act{a}(\vect{t})\syntor\delta\syntor\pr{Y}(\vect{t})\syntor p+p\syntor p\seqc p\syntor p\parc p\syntor\sum_{d\ap D}p\syntor\\[-1ex]
p\lcond c\rcond p\syntor\hide{I}(p)\syntor\encap{H}(p)\syntor\ren{R}(p)
\end{gather*}
\pause
Value-passing communication:\\
$\quad\act{a}(5)\comm\sum_{n\ap\Nat}\act{b}(n)~\id~\sum_{n\ap\Nat}\act{c}(n)\lcond n=5\rcond\delta~\id~\act{c}(5)$
\pause
Process equations: $\pr{X}(n\ap\Nat)=\act{a}(n)\seqc\pr{X}(succ(n))+\act{b}(n)$
%###################################################
%###################################################
\newslide\section*{Crucial Axioms}
\vspace{-1.5ex}
\begin{footnotesize}
  \begin{align}
    +&\text{ is CAI, }\delta\text{ is a 0 element}\notag\\
    (x+y)\seqc z&\id x\seqc z+ y\seqc z\tag{A4}\label{eq:A4}\\
    (x\seqc y)\seqc z&\id x\seqc (y\seqc z)\tag{A5}\label{eq:A5}\\
    \delta\seqc x&\id \delta\tag{A7}\label{eq:A7}%\\[1.8ex]
  \end{align}
  \begin{align}
    x\parc y&\id (x\leftm y+y\leftm x)+x\comm y\tag{CM1}\label{eq:CM1}\\
    b\leftm x&\id b\seqc x\tag{CM2}\label{eq:CM2}\\
    (b\seqc x)\leftm y&\id b\seqc(x\parc y)\tag{CM3}\label{eq:CM3}\\
    (b\seqc x)\comm(b'\seqc y)&\id (b\comm b')\seqc(x\parc y)\tag{CM7}\label{eq:CM7}\\
    \act{a}(\vect{d})\comm \act{a'}(\vect{d'})&\id \gamma(\act{a},\act{a'})(\vect{d})\lcond
    \vect{d}=\vect{d'}\rcond\delta\quad\text{if}~\gamma(\act{a},\act{a'})~\text{is defined}\tag{CF1}\label{eq:CF1}\\
    \act{a}(\vect{d})\comm \act{a'}(\vect{d'})&\id \delta\quad\text{otherwise}\tag{CF2}\label{eq:CF2}
  \end{align}
\end{footnotesize}
%%###################################################
%###################################################
\newslide\section*{Linear Process Equations}
Linear Process Equation (LPE):
\begin{itemize}
\item a restricted form of \mcrl\ equation
\item a symbolic representation of LTS.
\end{itemize}
\pause
Form ($I$ and $J$ are disjoint finite sets of indexes):
\begin{gather*}
\begin{split}
\pr{X}(\vect{d\ap D})=&\sum_{i\in I}\sum_{\vect{e_i\ap E_i}} \act{a}_i(\vect{f_i}(\vect{d,e_i}))\seqc
\pr{X}(\vect{g_i}(\vect{d,e_i}))\lcond c_i(\vect{d,e_i})\rcond\delta\\
&+\sum_{j\in J}\sum_{\vect{e_j\ap E_j}} \act{a}_j(\vect{f_j}(\vect{d,e_j}))\lcond c_j(\vect{d,e_j})\rcond\delta
\end{split}
\end{gather*}
%###################################################
%###################################################
\newslide\section*{Overview of the \mcrl\ Toolset}
\begin{center}
\includegraphics[width=\textwidth]{toolset}
\end{center}
%###################################################
%###################################################
\newslide\section*{Discrete time in \mcrl: a simple approach}
\begin{itemize}
\item Passage of time: $\act{tick}$ action (one time unit passed)
\item \emph{All} processes synchronize to $\act{tick}$
\pause
\item Technically: use \emph{renaming} and \emph{multi-party} communication:
$\gamma(\act{tick},\act{tick})=\act{\_tick}$ and
\end{itemize}
\vspace{-2ex}
\begin{gather*}
\begin{split}
S=\encap{\set{\act{tick}}}(\pr{P}_0\parc&\ren{\set{\act{\_tick}->\act{tick}}}(\encap{\set{\act{tick}}}(\pr{P}_1\parc\\
&\ren{\set{\act{\_tick}->\act{tick}}}(\encap{\set{\act{tick}}}(\pr{P}_2\parc\pr{P}_3)))))
\end{split}
\end{gather*}
\pause
\begin{itemize}
\item Many advantages: reuse of \mcrl\ Toolset, untimed formalism, relative time, etc
\item One disadvantage: \emph{fixed} time progress only
\end{itemize}
%###################################################
%###################################################
\newslide\section*{Time Transition Systems}
Can be classified by 3 criteria:
\begin{itemize}
\item discrete vs. real time ($\act{a}\at{1}\seqc\act{b}\at{100}$ vs $\act{a}\at{1.34}\seqc\act{b}\at{54.5}$)
\item absolute vs. relative time
\item time-stamped vs. two-phase models ($\act{a}\at{1}\seqc\act{b}\at{100}$ vs. $\sigma^1(\act{a})\seqc\sigma^{99}(\act{b})$)
\end{itemize}
\pause
\begin{center}
\includegraphics[width=.9\textwidth]{TLTSs}
\end{center}
%###################################################
%###################################################
\newslide\section*{Properties of Time Transition Systems}
\begin{itemize}
\item \emph{time determinism}: if $B \to_t B_1$ and $B \to_t B_2$, then $B_1\equiv B_2$
\item \emph{time interpolation}: if $B \to_t B_1$ and $B_1 \to_u B_2$, then $B \to_{t+u}B_2$;
\item \emph{time additivity}: if $B \to_{t+u} B_1$, then there is a closed term $B'$ such that $B\to_{t} B'$ and $B'\to_{u} B_1$
\end{itemize}
\pause
Two-phase transition systems have to be saturated to satisfy the last 2 properties.
%###################################################
%###################################################
\newslide\section*{Timed \mcrl: design choices and operations}
\begin{itemize}
\item Choice: absolute time, time-stamped model, *arbitrary* time domain; variable time progress
\item Time domain: must have $0$ and total order $\leq$
\item Main operation: ``at''-operation $x\at t$
\end{itemize}
\pause
\subsection*{Advantages and disadvantages}
\begin{itemize}
\item[+] no saturations: easier bisimulation and parallel composition definition
\item[+] relatively easy branching bisimulation definition
\item[-] sometimes more infinite transition systems.
\end{itemize}
%###################################################
%###################################################
\newslide\section*{Timed \mcrl: Crucial Axioms}
\vspace{-1.5ex}
\begin{footnotesize}
  \begin{align}
&  x\id \sum_{t\ap\Time}x\at t\tag{AT1}\label{eq:AT1}\\ %at
&  b\at t\seqc y\id b\at t\seqc t\ti y\tag{AT2}\label{eq:AT2}\\ %at2
%  (b\at t)\seqc y\id (b\at t)\seqc(t\tii y)\tag{AT3}\label{eq:AT3}\\ %at3
&  x\at t\at u\id x\at t\lcond t=u\rcond\daz+\udd(x)\at\nm{min}(t,u)\tag{ATA1$'$}\label{eq:ATA1'}\\ %ata1
&  (b\at t\seqc x)\leftm y\id (b\at t\tb y)\seqc((t\ti x)\parc y)\tag{CM3T}\label{eq:CM3T}\\
&  (x\comm y)\at t\id x\at t\comm y\tag{ATA7}\label{eq:ATA7}\\
&  x\tb(y\at t)\id \sum_{u\ap\Time}(x\tb y)\at u\lcond u\leq t\rcond\daz\tag{ATC11}\label{eq:ATC11}\\ %atc11
&  \delta\tb x\id \udd(x)\tag{ATCC0}\label{eq:ATCC0} %atcc0
  \end{align}
\end{footnotesize}
%###################################################
%###################################################
\newslide\section*{Timed \mcrl: Synchronization}
\vspace{-1.5ex}
\begin{footnotesize}
  \begin{align*}
&  \encap{\set{\act{a},\act{\_a}}} ((\act{a}\seqc x+\act{\_b}\seqc y)\parc \act{\_a}\seqc z)\id\\
&  \quad\act{\_\_a}\seqc (x\parc z)+\act{\_b}\seqc\encap{\set{\act{a},\act{\_a}}} (y\parc\act{\_a}\seqc z)\\
&  \\
&  \encap{\set{\act{a},\act{\_a}}} ((\act{a}\at 3\seqc x+\act{\_b}\at 5\seqc y)\parc \act{\_a}\at 3\seqc z)\id\\
&  \quad\act{\_\_a}\at 3\seqc (x\parc z)\\
&  \\
&  \encap{\set{\act{empty},\act{stop},\act{\_empty},\act{\_stop}}} ((\sum_{t\ap\Time}\act{empty}\at t\seqc x\lcond t<5\rcond\daz+\act{\_stop}\at 5\seqc y)\parc\\
&\t4 (\sum_{t\ap\Time}\act{stop}\at t\seqc z'\lcond t<5\rcond\daz+\act{\_empty}\at 5\seqc z)\id\delta\at 5
  \end{align*}
\end{footnotesize}
%###################################################
%###################################################
\newslide\section*{Timed \mcrl: status}
Current:
\begin{itemize}
\item Soundness and completeness results w.r.t. strong bisimulation (GRWvdZ 2002)
\item Branching bisimulation definition (vdZwaag 2001)
\item Translation from timed automata (Willemse, 2003)
\item Linearization; TLPE$\to$LPE (RU 2002,2005), Implementation (JFG 2005)
\end{itemize}
\pause
Future:
\begin{itemize}
\item ADT for zones and regions (replace sort Time)
\item General abstract interpretation for sort Time
\end{itemize}
%###################################################
%###################################################
\newslide\section*{Stochastic transition systems}
\begin{itemize}
\item Discrete case: several definitions (Bartels, Sokolova and de Vink, TCS, 2004: co-algebraic comparison).
\pause
\item Continuous probabilistic transition systems (all time models -- two-phase)
  \begin{itemize}
  \item PTS and Timed PTS -- d'Argenio,
  \item ISTTS (Interactive Stochastic Timed TS) -- Bravetti's. Weak bisimulation. Parallel composition and hiding.
  \item PTTS (Probabilistic Timed TS): Bravetti and d'Argenio
  \end{itemize}
\end{itemize}
%###################################################
%###################################################
\newslide\section*{Continuous Probabilistic Transition Systems}
$S$ -- states. ${\cal F}\subseteq 2^S$ -- $\sigma$-algebra of sets of states.

A PTS is a structure $(S, {\cal F}, L, \to)$ such that:
\begin{itemize}
\item $\to\subseteq S\x L\x(\algb{F}\to[0,1])$
\item for any $s\in S$ and $a\in L$, if $(s,a,P)\in\to$, then $P$ is a probability measure on $(S,\algb{F})$.
%\item for any $\sigma\in\algb{F}$, $a\in L$
%and $p\in[0,1]$, we have
%\[\set{s\suchthat \exists P ~~ (s,a,P)\in\to~\land~P(\sigma)=p}\in\algb{F} \]
\end{itemize}
\pause
\begin{center}
\includegraphics[width=.5\textwidth]{PTSs}
\end{center}
%###################################################
%###################################################
\newslide\section*{Bisimulation for Continuous PTSs}
Let $T=(S,\algb{F}, L, \to)$ be a PTS. An equivalence relation\\
$R$ on $S$, compatible with $\algb{F}$, is a probabilistic bisimulation on $T$ if:
\[ (s R t\land s\trans{a} P) \to \exists Q~(t\trans{a} Q \land P \equiv_R Q) \]
\pause
$P$ and $Q$ are \emph{equal up to} $R$ (notation $P\equiv_R Q$) if
for any set $S'\in\algb{F}$ such that
\[\forall s_1,s_2\in S~~s_1 R s_2 \to (s_1\in S'\leftrightarrow s_2\in S')\]
we have $P(S')=Q(S')$. ($S'$ are \emph{arbitrary} unions of the equivalence classes of $R$)
%###################################################
%###################################################
\newslide\section*{Stochastic Systems: status}
Current:
\begin{itemize}
\item Definition of LTS and strong bisimulation
\item Ongoing work on weak and branching bisimulation definition (with d'Argenio, Hermanns, Stoelinga)
\end{itemize}
\pause
Future:
\begin{itemize}
\item Branching bisimulation for time-stamped PTS.
\item Operations, axiomatization, LPE format, linearization.
\end{itemize}
%###################################################
%###################################################
\newslide\section*{Hybride Transition Systems Ingredients}
\begin{itemize}
\item Discrete actions vs. continious trajectories
\item Trajectory: mapping from (port) names to real-valued functions
\pause
\item Trajectory determinism vs. time determinism
  \begin{center}
  \input{superpos.tex}
  \end{center}
\item Superposition does not preserve abstraction of trajectory ports
\end{itemize}
%\begin{enumerate}
%\item if $B \xto{\varphi}_t B_1$ and $B \xto{\varphi}_t B_2$, then $B_1\equiv B_2$;
%\item if $B \xto{\varphi}_t B_1$ and $B_1 \xto{\psi}_u B_2$, then $B \xto{\varphi;\psi}_{t+u}B_2$;
%\item if $B \xto{\varphi}_{t+u} B_1$, then there is a closed term $B'$ such that $B\xto{\varphi\restrict t}_{t} B'$ and $B'\xto{\timeshift{\varphi}{u}}_{u} B_1$.
%\end{enumerate}
%###################################################
%###################################################
\newslide\section*{Hybride Transition Systems models}
\begin{itemize}
\item ACP$^{srt}_{hs}$:
  \begin{itemize}
  \item time: two-phase, hybrid trajectories are state-based in the calculus
  \item operations: \emph{continuous signal insertion operator} ($\varphi\csins x$),
    \emph{relative time delay} ($\delay{r}(x)$)
  \item crucial axiom: $\varphi\csins x+\psi\csins y\id \varphi\wedge\psi\csins(x+y)$
  \end{itemize}
\pause
\item BHPC:
  \begin{itemize}
  \item two-phase, relative time
  \item operations: \emph{traj. prefix} $\tpref{\varphi}{t}\seqc x$, \emph{superposition} $\superps$
  \item crucial axiom: $\tpref{\varphi}{t}\seqc x\superps \tpref{\varphi}{t}\seqc y\id \tpref{\varphi}{t}\seqc (x\superps y)$
  \end{itemize}
\end{itemize}
%###################################################
%###################################################
\newslide\section*{Connection between ACP$^{srt}_{hs}$ and BHPC}
\begin{itemize}
\item trajectories $\varphi$ and $\psi$ are different on the entire interval $(0,t]$ for some $t$
\item consider $\tpref{\varphi}{t}\seqc x+\tpref{\psi}{u}\seqc y$
\item can we translate it as $\varphi\csins\delay{t}(x)+\psi\csins\delay{u}(y)$ ?
\pause
\item or as $\varphi\vee\psi\csins(\varphi\cond\delay{t}(x)+\psi\cond\delay{u}(y))$ ?
\end{itemize}

\[(\tpref{\varphi}{t}\seqc x+\tpref{\psi}{u}\seqc y)\parc^A_{\pi(\varphi)}\tpref{\varphi}{t}\seqc z\id \tpref{\varphi}{t}\seqc (x\parc^A_{\pi(\varphi)} z)\]
\[\varphi\vee\psi\csins(\varphi\cond\delay{t}(x)+\psi\cond\delay{u}(y))\parc\varphi\csins\delay{t}(z)\id \varphi\csins\delay{t}(x\parc z)\]
%###################################################
%###################################################
\newslide\section*{BHPC and Binders}
\begin{itemize}
\item $\superpos{\varphi\ap\Phi}\tpref{\varphi}{}\seqc \pr{Y}(\varphi)$
\item Difficult to give SOS rules and axioms.
\item $\pr{A}=\superpos{\varphi\ap\Time}\tpref{\varphi}{}\seqc\act{a}(\varphi)$
and $\pr{B}=\superpos{\psi\ap\Time}\tpref{\psi}{}\seqc\act{b}(\psi)$
\item $\pr{A}\parc\pr{B}=\superpos{\chi\ap\Time}\tpref{\chi}{}\seqc(\act{a}(\varphi)\seqc\superpos{\psi\ap\Time}\tpref{\psi}{}\seqc\act{b}(\psi)+\act{b}(\varphi)\seqc\superpos{\varphi\ap\Time}\tpref{\varphi}{}\seqc\act{a}(\varphi))$
\end{itemize}

%###################################################
%###################################################
\newslide\section*{Hybrid Systems: status}
Current:
\begin{itemize}
\item A transition system model becomes clear. (2 calculi with a similar semantic model)
\item None of them really compatible with timed \mcrl
\item Other models (HyPA, hybrid-$\chi$)
\end{itemize}
\pause
Future:
\begin{itemize}
\item Try to come-up with a timed stamped HTS definition
\item Equivalences, operations, axiomatization, LPE format, linearization
\item Try to combine with stochastic TSs
\end{itemize}
%###################################################
%###################################################
\newslide\section*{Embedding \mcrl\ into a software development process}
\begin{itemize}
\item Extension with multi-actions (to connect to Petri-Nets)
\item Extensions with disrupts and priorities to connect to LOTOS and UML
\item Combine with some refinement-based technique (RAISE/Z, maybe CASL) on top (frontend)
\item Combine/design a code generation framework (may be via PROMELA) at bottom (backend)
\end{itemize}
%###################################################
%###################################################
\newslide\section*{Conclusions}
\begin{itemize}
\item We attempt to define an action-based transition system formalism
where real-time, stochastic and hybrid systems can be specified,
and that is compatible with the current \mcrl\ Toolset.
\pause
\item There are still open ends with respect to the design choices
and open fundamental semantic questions.
\pause
\item We need characteristic case studies to justify the design choices
and focus attention to particular aspects.
\end{itemize}
%###################################################


\end{slide}
\end{document}
