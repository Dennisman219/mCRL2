\documentclass[a4paper,fleqn,10pt]{article}

\title{mCRL2 syntax definition}
\author{Aad Mathijssen}

% packages
% --------

\usepackage{amsmath}
\usepackage{amsfonts}
\usepackage{graphicx,theorem,ifthen,float}
\usepackage[T1]{fontenc}
\usepackage[latin1]{inputenc}
\usepackage[english]{babel}
%\usepackage{algorithm}
%\usepackage{algorithmic}
\usepackage{multicol}
\usepackage{array}
\usepackage{stmaryrd}


% increase pagewidth
\addtolength{\textwidth}{20mm}
\addtolength{\oddsidemargin}{-10mm}
\addtolength{\evensidemargin}{10mm}


% column types that change column types l,c,r from math mode to LR
% and the other way round
\newcolumntype{L}{>{$}l<{$}}%stopzone%stopzone%stopzone
\newcolumntype{C}{>{$}c<{$}}%stopzone%stopzone%stopzone
\newcolumntype{R}{>{$}r<{$}}%stopzone%stopzone%stopzone


% commands
% --------


% List operator
\newcommand{\conc}{{\fontsize{8pt}{0pt} + \!\!\!\! +}}

% mathematical constant
\newcommand{\f}[1]{\ensuremath{\mathit{#1}}}

% definition font
\newcommand{\deffont}[1]{{\bf #1}}

% interpretation of a sort in a data algebra
\newcommand{\meansrt}[2]{
  {#1}^{#2}
}

% interpretation of an operation in a data algebra
\newcommand{\meanop}[2]{
  {#1}^{#2}
}

% interpretation of a data term in a data environment and a data algebra
\newcommand{\meandta}[3]{
  \llbracket  #1 \rrbracket_{#2}^{#3}
}

% interpretation of a data term in a data environment
\newcommand{\meandt}[2]{
  \llbracket  #1 \rrbracket_{#2}
}

\newcommand{\lbanana}{{\langle\hspace{-.465ex}\mid}}
\newcommand{\rbanana}{{\mid\hspace{-.465ex}\rangle}}

\newcommand{\banana}[3]{{\lbanana #1 \rbanana_{#2}^{#3}}}


% interpretation of an action in a data environment and a data algebra
\newcommand{\meanacta}[3]{
  \llbracket  #1 \rrbracket_{#2}^{#3}
}

% interpretation of an action in a data environment
\newcommand{\meanact}[2]{
  \llbracket  #1 \rrbracket_{#2}
}

% interpretation of an action formula in a data environment and a data algebra
\newcommand{\meanafa}[3]{
  \llbracket  #1 \rrbracket_{#2}^{#3}
}

% interpretation of an action formula in a data environment
\newcommand{\meanaf}[2]{
  \llbracket  #1 \rrbracket_{#2}
}

% interpretation of a state formula in a data environment and a propositional
% environment
\newcommand{\meansf}[3]{
  \llbracket  #1 \rrbracket_{{#2}{#3}}
}
% solution of a PBES in the context of a data environment and a propositional
% environment
\newcommand{\pbessol}[3]{
  [  #1 ]_{{#2}{#3}}
}

% boolean sort: B
\newcommand{\sbool}{\ensuremath{\mathsf{B}}}
\newcommand{\strue}{\ensuremath{\mathsf{true}}}
\newcommand{\sfalse}{\ensuremath{\mathsf{false}}}

% booleans
\newcommand{\bool}{\ensuremath{\mathbb{B}}}

% natural sort: N
\newcommand{\snat}{\ensuremath{\mathsf{N}}}

% positive sort: N^+
\newcommand{\spos}{\ensuremath{\snat^{+}}}

% integral sort: Z
\newcommand{\sint}{\ensuremath{\mathsf{Z}}}

% real sort: R
\newcommand{\sreal}{\ensuremath{\mathsf{R}}}

% non-negative real sort: R^{>=0}
\newcommand{\snnreal}{\ensuremath{\mathsf{R}_{\scriptscriptstyle 0}}}

% non-negative reals
\newcommand{\nnreal}{\ensuremath{\mathbb{R}_{\scriptscriptstyle 0}}}

%tuple
\newcommand{\tuple}[1]{\ensuremath{\langle {#1} \rangle}}

%set of all transitions
\newcommand{\strans}{\ensuremath{\f{Tr}}}

%transition
\newcommand{\trans}[1]{\,{}^{\underrightarrow{\ #1 \ }}\,}

%subset or equal on functions
\newcommand{\subseteqfun}{\,{\dot\subseteq}\,}

% environments
% ------------

% proof: proof of a theorem
\newenvironment{proof}
  {\textbf{Proof}}%
  %{\frm{\Box}%
  % \vspace{1ex}%
  %}

% tightarray: array with no outcolumn spacing and tighter intercolumn spacing
\newenvironment{tightarray}[1]
  {\setlength{\arraycolsep}{2pt}%
   \begin{array}{@{}#1@{}}%
  }
  {\end{array}%
  }

% theorems
% --------

\newtheorem{thdefinition}{Definition}[section]
\newenvironment{definition}
  {\begin{thdefinition}\em}
  {\end{thdefinition}}

\newtheorem{ththeorem}[thdefinition]{Theorem}
\newenvironment{theorem}
  {\begin{ththeorem}\em}
  {\end{ththeorem}}

\newtheorem{thcorollary}[thdefinition]{Corollary}
\newenvironment{corollary}
  {\begin{thcorollary}\em}
  {\end{thcorollary}}

\newtheorem{thlemma}[thdefinition]{Lemma}
\newenvironment{lemma}
  {\begin{thlemma}\em}
  {\end{thlemma}}

\newtheorem{thproperty}[thdefinition]{Property}
\newenvironment{property}
  {\begin{thproperty}\em}
  {\end{thproperty}}

\newtheorem{thexample}[thdefinition]{Example}
\newenvironment{example}
  {\begin{thexample}\em}
  {\end{thexample}}

\newtheorem{thremark}[thdefinition]{Remark}
\newenvironment{remark}
  {\begin{thremark}\em}
  {\end{thremark}}

\newtheorem{thconvention}[thdefinition]{Convention}
\newenvironment{convention}
  {\begin{thconvention}\em}
  {\end{thconvention}}

\newtheorem{thconjecture}[thdefinition]{Conjecture}
\newenvironment{conjecture}
  {\begin{thconjecture}\em}
  {\end{thconjecture}}

\newtheorem{thspecification}[thdefinition]{Specification}
\newenvironment{specification}
  {\begin{thspecification}\em}
  {\end{thspecification}}

\newtheorem{thdeclaration}[thdefinition]{Declaration}
\newenvironment{declaration}
  {\begin{thdeclaration}\em}
  {\end{thdeclaration}}

% sets
% ----

% set of elements: {e}
\newcommand{\set}[1]{\ensuremath{\{\,#1\,\}}}

% bag of elements: {e}
\newcommand{\bag}[1]{\ensuremath{\set{#1}}}

% set difference: s \ t
\newcommand{\sdiff}[2]{\ensuremath{#1\ \backslash\ #2}}

% set comprehension: { e | c }, where e is an expression or a binding and c is
% a condition
\newcommand{\scompr}[2]{\ensuremath{\set{#1\ |\ #2}}}

% lists
% -----

% list of a certain type: L(t)
\newcommand{\List}[1]{\ensuremath{\mathcal{L}{\I{#1}}}}

% list of elements: [e]
\newcommand{\lst}[1]{\ensuremath{[\,#1\,]}}

% empty list
\newcommand{\el}{\ensuremath{[\,]}}

% cons: |>
\newcommand{\cons}{\ensuremath{\hspace{0.12em}\triangleright\hspace{0.08em}}}

% snoc: <|
\newcommand{\snoc}{\ensuremath{\hspace{0.08em}\triangleleft\hspace{0.12em}}}

% concatenation: ++
\newcommand{\concat}{\ensuremath{\hspace{0.12em}\raisebox{0.2ex}
{\text{\footnotesize+}\!\text{\footnotesize+}}\hspace{0.12em}}}

% process algebra
% ---------------

% sequential composition .
\newcommand{\seq}{\mathbin{\cdot}}

% alternative composition +
\newcommand{\alt}{\mathbin{+}}

% parallel merge ||
\newcommand{\pmerge}{\mathbin{\parallel}}

% left merge ||_
\newcommand{\lmerge}{\mathbin{\llfloor}}

% synchronisation |
\newcommand{\sync}{\mathbin{\!\mid\!}}

% block
\newcommand{\block}[1]{\partial_{#1}}

% hide
\newcommand{\hide}[1]{\tau_{#1}}

% rename
\newcommand{\ren}[1]{\rho_{#1}}

% allow
\newcommand{\allow}[1]{\nabla_{#1}}

% communication
\newcommand{\comm}[1]{\Gamma_{#1}}

% at
\font \aap cmmi10
\newcommand{\at}[1]{\mbox{\aap ,} #1}

% initialisation
\newcommand{\pinit}{\gg}

% before
\newcommand{\pbefore}{\ll}

% ???
\newcommand{\ap}{{:}}

% Logic
% -----

% true
\newcommand{\true}{\ensuremath{\f{true}}}

% false
\newcommand{\false}{\ensuremath{\f{false}}}

% implies: =>
\newcommand{\limp}{\ensuremath{\Rightarrow}}

% logical iff
\newcommand{\liff}{\ensuremath{\Leftrightarrow}}

% Misc definitions
% ----------------

% simple bold math
\newcommand{\mb}[1]{\text{\boldmath$#1$}}

% mCRL2 keywords
\newcommand{\kwsort}{{\bf sort}}
\newcommand{\kwcons}{{\bf cons}}
\newcommand{\kwmap}{{\bf map}}
\newcommand{\kwvar}{{\bf var}}
\newcommand{\kweqn}{{\bf eqn}}
\newcommand{\kwact}{{\bf act}}
\newcommand{\kwstruct}{{\bf struct}}
\newcommand{\kwwhr}{{\bf whr}}
\newcommand{\kwend}{{\bf end}}
\newcommand{\kwdiv}{{\bf div}}
\newcommand{\kwmod}{{\bf mod}}
\newcommand{\kwglob}{{\bf glob}}
\newcommand{\kwproc}{{\bf proc}}
\newcommand{\kwinit}{{\bf init}}
\newcommand{\kwpbes}{{\bf pbes}}


% Abbreviations

\newcommand{\isdef}{\ensuremath{=_{\textit{def}}}}

% document
% --------


\begin{document}

\maketitle

This document describes the syntax of mCRL2 expressions and specifications.
We present the syntax in a rich text format.
In Section~\ref{sec:symbols} a translation of rich text to plain text is given, which is needed for using the toolset.

Throughout this document, suggestive dots ($\mb{\ldots}$, $\mb{\cdots}$) are used to indicate repeating patterns with one or more
occurrence. Furthermore, $\mb{|}$ distinguishes alternatives (not to be mistaken with the pipe $|$),
$\mb{(}\f{pattern}\mb{)^{+}}$ indicates one or more occurrences of $\f{pattern}$,
and $\mb{(}\f{pattern}\mb{)^{*}}$ indicates zero or more occurrences of $\f{pattern}$.
As opposed to real EBNF, we do not use quotes to separate the terminals from the non-terminals.

%%%%%%%%%%%%%%%%%%%%%%%%
\section{Lexical syntax}

We defined the notions of identifiers, numbers, whitespace and comments:
\begin{itemize}
\item
An \emph{identifier} is a string matching the pattern ``$[A{-}Za{-}z\_][A{-}Za{-}z\_0{-}9]*$'', excluding the following reserved words:
\[\begin{tightarray}{L}
sort
cons
map
var
eqn
act
glob
proc
pbes
init
\\
struct
Bool
Pos
Nat
Int
Real
List
Set
Bag
\\
true
false
div
mod
in
lambda
forall
exists
whr
end
\\
delta
tau
sum
block
allow
hide
rename
comm
\\
val
mu
nu
delay
yaled
nil
\end{tightarray}\]
Identifiers are used for representing sort names $b$, function names $f$, data variable names $x$, action names $a$, process reference names $P$, and propositional variable names $X$.

\item
A \emph{number} is a string that matches the pattern ``$0$'' or ``$[1-9][0-9]*$''.

\item
Spaces, tabs and newlines are treated as \emph{whitespace}.

\item
A $\%$-sign indicates the beginning of a \emph{comment} that extends to the end of the line.
\end{itemize}

%%%%%%%%%%%%%%%%%%%%%%%%%%%%%
\section{Data specifications}

Sort expressions $s$:
\[\begin{tightarray}{lrl}
s      & \mb{::=} & b\ \mb{|}\ s \to s\ \mb{|}\
               \sbool\ \mb{|}\ \spos\ \mb{|}\ \snat\ \mb{|}\ \sint\ \mb{|}\ \sreal\\
       & \mb{|}   &
               \kwstruct\ \f{scs}\,|\,\mb{\cdots}\,|\,\f{scs}\ \mb{|}\
               \mathsf{List}(s)\ \mb{|}\ \mathsf{Set}(s)\ \mb{|}\ \mathsf{Bag}(s)\ \mb{|}\
               ( s )\\
\f{scs} & \mb{::=} & f\ \mb{|}\ f(\f{spj},\mb{\ldots},\f{spj})\ \mb{|}\
                f ? f\ \mb{|}\ f(\f{spj},\mb{\ldots},\f{spj})?f \\
\f{spj} & \mb{::=} & s\ \mb{|}\ s \times \mb{\cdots} \times s \to s\ \mb{|}\
                f: s\ \mb{|}\ f: s \times \mb{\cdots} \times s \to s
\end{tightarray}\]
Here $\f{scs}$ and $\f{spj}$ stand for the constructor and projection functions of a structured sort.
The binary operator $\to$ associates to the right.

Data expressions $d$:
\[\begin{tightarray}{lrl}
d      & \mb{::=} & x\ \mb{|}\ f\ \mb{|}\ d(d, \mb{\ldots}, d)\ \mb{|}\ N\ \mb{|}\
               \lnot d\ \mb{|}\ -d\ \mb{|}\ \overline{d}\ \mb{|}\ \#d\ \mb{|}\ d\ \oplus\ d\\
       & \mb{|} & \el\ \mb{|}\ [d,\mb{\ldots},d]\ \mb{|}\ \set{}\ \mb{|}\ \set{d,\mb{\ldots},d}\ \mb{|}\
               \set{d: d,\mb{\ldots},d: d}\ \mb{|}\ \scompr{x: s}{d}\\
       & \mb{|} & \lambda_{\f{mvd},\mb{\ldots},\f{mvd}} d\ \mb{|}\
               \forall_{\f{mvd},\mb{\ldots},\f{mvd}} d\ \mb{|}\ \exists_{\f{mvd},\mb{\ldots},\f{mvd}} d\ \mb{|}\
               d\ \kwwhr\ x=d,\mb{\ldots},x=d\ \kwend\\
        & \mb{|} & ( d )\\
\f{mvd}& \mb{::=} & x,\mb{\ldots},x: s\\
\oplus & \mb{::=} & *\ \mb{|}\ .\ \mb{|}\ \cap\ \mb{|}\
               /\ \mb{|}\  \kwdiv\ \mb{|}\ \kwmod\ \mb{|}\
               +\ \mb{|}\ -\ \mb{|}\ \cup\ \mb{|}\
               \concat\ \mb{|}\
               \snoc\ \mb{|}\
               \cons\\
       & \mb{|} &
               <\ \mb{|}\ \leq\ \mb{|}\ \geq\ \mb{|}\ >\ \mb{|}\ \subset\ \mb{|}\ \subseteq\ \mb{|}\ \in\ \mb{|}\
               \approx\ \mb{|}\ \not\approx\ \mb{|}\
               \land\ \mb{|}\ \lor\ \mb{|}\
               \limp
\end{tightarray}\]
Here $\f{mvd}$ stands for a multiple data variable declaration, $\oplus$ for a binary operator, and $N$ for a number.
The unary operators have the highest priority, followed by the infix operators,
followed by $\lambda$, $\forall$ and $\exists$, followed by $\kwwhr\ \kwend$.
The descending order of precedence of the infix operators is:
$\set{*, ., \cap}, \set{/, \kwdiv, \kwmod}, \set{+, -, \cup}, \concat, \snoc,
\cons, \set{<, \leq, \geq, >, \subset, \subseteq, \in}, \set{\approx,\not\approx},
\set{\land, \lor}, \limp$.
Of these operators $*$, $.$, $\cap$, $/$, $\kwdiv$, $\kwmod$, $+$, $-$, $\cup$ and
$\concat$ associate to the left and $\approx$, $\not\approx$, $\land$, $\lor$
and $\limp$ associate to the right.

Data specifications $\f{data\_spec}$:
\[\begin{tightarray}{lrl}
\f{data\_spec}
& \mb{::=} & \kwsort\ \mb{(}\f{sd};\mb{)^{+}}\\
& \mb{|}   & \kwcons\ \mb{(}\f{mfd};\mb{)^{+}}\\
& \mb{|}   & \kwmap\  \mb{(}\f{mfd};\mb{)^{+}}\\
& \mb{|}   & \kwvar\  \mb{(}\f{mvd};\mb{)^{+}}\ \kweqn\ \mb{(}ed;\mb{)^{+}}\\
& \mb{|}   & \kweqn\  \mb{(}ed;\mb{)^{+}}\\
\f{sd}  & \mb{::=} & b\ \mb{|}\ b = s\\
\f{mfd} & \mb{::=} & f, \mb{\ldots}, f: s\\
\f{ed}  & \mb{::=} & d = d\ \mb{|}\ c \to d = d\\
\end{tightarray}\]
Here, $\f{sd}$ stands for sort declaration,
$\f{mfd}$ for multiple function declaration,
$\f{ed}$ for equation declaration,
and $\f{ad}$ for action declaration.

%%%%%%%%%%%%%%%%%%%%%%%%%%%%%%%%
\section{Process specifications}

Process expressions $p$:
\[\begin{tightarray}{lrl}
p   & \mb{::=} & a\ \mb{|}\ \delta\ \mb{|}\ \tau\ \mb{|}\ p \alt p\ \mb{|}\ p \seq p\ \mb{|}\ P\ \mb{|}\
                 p \sync p\ \mb{|}\ p \pmerge p\ \mb{|}\ p \lmerge p\\
    & \mb{|}   & \allow{\set{\f{as},\mb{\ldots},\f{as}}}(p)\ \mb{|}\
                 \block{\set{a,\mb{\ldots},a}}(p)\ \mb{|}\
                 \hide{\set{a,\mb{\ldots},a}}(p)\ \mb{|}\
                 \ren{\set{\f{ar},\mb{\ldots},\f{ar}}}(p)\ \mb{|}\
                 \comm{\set{\f{ac},\mb{\ldots},\f{ac}}}(p)\\
    & \mb{|}   & a(d, \mb{\ldots}, d)\ \mb{|}\
                 P(d, \mb{\ldots}, d)\ \mb{|}\
                 P()\ \mb{|}\
                 P(x = d, \mb{\ldots}, x = d)\\
    & \mb{|}   & c \to p \diamond p\ \mb{|}\
                 c \to p\ \mb{|}\
                 \sum_{\f{mvd},\mb{\ldots},\f{mvd}}p\\
     & \mb{|}  & p\,\at\,t\ \mb{|}\ t \pinit p\ \mb{|}\ p \pbefore q\\
     & \mb{|}  & ( p )\\
as  & \mb{::=} & a |\,\mb{\cdots} | a\\
ar  & \mb{::=} & a \to a\\
ac  & \mb{::=} & a\sync\f{as} \to a\ \mb{|}\ a\sync\f{as} \to \tau\ \mb{|}\ a\sync\f{as}\\
\end{tightarray}\]
Here, $c$ and $t$ stand for data expressions of sort $\sbool$ and $\sreal$,
respectively.
For technical reasons, $c$ and $t$ may not have an infix operator, a
where clause or a quantifier at the top-level (parentheses should be
used instead).
$\f{as}$ represents an action sequence, $\f{ar}$ an action
renaming, and $\f{ac}$ an action communication. The descending order of
precedence of the operators is: $\mid, \at, \seq, \set{\pinit, \pbefore}, \to,
\set{\pmerge,\lmerge},\sum,\alt$. Of these operators $\alt$, $\pmerge$,
$\lmerge$, $\seq$ and $\mid$ associate to the right.

Process specifications $\f{proc\_spec}$:
\[\begin{tightarray}{lrl}
\f{proc\_spec}      & \mb{::=} & \mb{(}\f{proc\_spec\_elt}\mb{)^{*}}\\
\f{proc\_spec\_elt} & \mb{::=} & \f{data\_spec}\\
                    & \mb{|}   & \kwact\ \mb{(}\f{ad};\mb{)^{+}}\\
                    & \mb{|}   & \kwglob\ \mb{(}\f{mvd};\mb{)^{+}}\\
                    & \mb{|}   & \kwproc\ \mb{(}\f{pd};\mb{)^{+}}\\
                    & \mb{|}   & \kwinit\ p;\\
\f{pd}              & \mb{::=} & P = p\ \mb{|}\ P(\f{mvd},\mb{\ldots},\f{mvd}) = p\\
\f{ad}              & \mb{::=} & a\ \mb{|}\ a: s \times \mb{\cdots} \times s
\end{tightarray}\]
Here $\f{proc\_spec\_elt}$ represents a process specification element, $\f{pd}$ a process definition, and $\f{ad}$ an action declaration. Furthermore, we impose the restriction that $\f{proc\_spec}$ should contain precisely one occurrence of the keyword $\kwinit$.

%%%%%%%%%%%%%%%%%%%%%%%%%%%%%%
\section{Mu-calculus formulae}

Multiactions $\f{ma}$:
\[\begin{tightarray}{lrl}
\f{ma} & \mb{::=} & \tau\ \mb{|}\ \f{pa}\,|\,\mb{\cdots} |\,\f{pa}\\
\f{pa} & \mb{::=} & a\ \mb{|}\ a(d,\mb{\ldots},d)
\end{tightarray}\]
Here, $\f{pa}$ represents a parameterised action.

Action formulae $\alpha$:
\[\begin{tightarray}{lrl}
\alpha & \mb{::=} & \f{ma}\ \mb{|}\
                    \alpha\at t\ \mb{|}\
                    \mathsf{val}(c)\ \mb{|}\
                    ( \alpha )\\
       & \mb{|}   & \mathsf{true}\ \mb{|}\
                    \mathsf{false}\ \mb{|}\
                    \neg \alpha\ \mb{|}\
                    \alpha \land \alpha\ \mb{|}\
                    \alpha \lor  \alpha\ \mb{|}\
                    \alpha \limp \alpha\ \mb{|}\
                    \forall_{\f{mvd},\mb{\ldots},\f{mvd}} \alpha\ \mb{|}\
                    \exists_{\f{mvd},\mb{\ldots},\f{mvd}} \alpha\\
\end{tightarray}\]
Here, $c$ and $t$ stand for data expressions of sort $\sbool$ and $\sreal$,
respectively.
For technical reasons, $t$ may not have an infix operator, a where
clause or a quantifier at the top-level (parentheses should be used
instead).
The descending order of precedence of the operators is: $\neg,
\at, \set{\land,\lor}, \limp, \set{\forall, \exists}$.  Of the infix operators
$\at$ associates to the left and $\land$, $\lor$ and $\limp$ associate to the
right.

Regular formulae $\varphi_r$:
\[\begin{tightarray}{lrl}
\varphi_r & \mb{::=} & \alpha\ \mb{|}\
                       \epsilon\ \mb{|}\
                       \varphi_r \cdot \varphi_r\ \mb{|}\
                       \varphi_r + \varphi_r\ \mb{|}\
                       \varphi_r^{*} \ \mb{|}\
                       \varphi_r^{+} \ \mb{|}\
                       ( \varphi_r )
\end{tightarray}\]
The postfix operators ${}^{*}$ and ${}^{+}$ have the highest priority, followed
by $\cdot$, followed by infix $+$. The infix operators associate to the right.

State formulae $\varphi_s$:
\[\begin{tightarray}{lrl}
\varphi_s & \mb{::=} & [\varphi_r]\varphi_s\ \mb{|}\
                       \langle\varphi_r\rangle\varphi_s\ \mb{|}\
                       \nabla(t)\ \mb{|}\
                       \Delta(t)\ \mb{|}\
                       \nabla\ \mb{|}\
                       \Delta\ \mb{|}\
                       \mathsf{val}(c)\ \mb{|}\
                       ( \varphi_s )\\
          & \mb{|}   & \nu X. \varphi_s\ \mb{|}\
                       \mu X.\varphi_s\ \mb{|}\
                       \nu X(\f{vdi},\mb{\ldots},\f{vdi}). \varphi_s\ \mb{|}\
                       \mu X(\f{vdi},\mb{\ldots},\f{vdi}).\varphi_s\ \mb{|}\
                       X\ \mb{|}\
                       X(d,\mb{\ldots},d)\\
          & \mb{|}   & \mathsf{true}\ \mb{|}\
                       \mathsf{false}\ \mb{|}\
                       \neg \varphi_s\ \mb{|}\
                       \varphi_s \land \varphi_s\ \mb{|}\
                       \varphi_s \lor  \varphi_s\ \mb{|}\
                       \varphi_s \limp \varphi_s\ \mb{|}\
                       \forall_{\f{mvd},\mb{\ldots},\f{mvd}} \varphi_s\ \mb{|}\
                       \exists_{\f{mvd},\mb{\ldots},\f{mvd}} \varphi_s\\
\f{vdi}  & \mb{::=} & x: s = d\\
\end{tightarray}\]
Here $\f{vdi}$ stands for a data variable declaration and initialisation,
and $c$ and $t$ stand for data expressions of sort $\sbool$ and
$\sreal$, respectively.
For technical reasons, $t$ may not have an infix operator, a where
clause or a quantifier at the top-level (parentheses should be used
instead).
The descending order of precedence of the operators is: $\set{\neg,
\_[\_], \_\langle\_\rangle}, \set{\land,\lor}, \limp, \set{\forall,
\exists, \mu, \nu}$.  The infix operators $\land$, $\lor$ and $\limp$
associate to the right.

%%%%%%%%%%%%%%%%
\section{PBES's}

Parameterised boolean expressions $\varphi_e$:
\[\begin{tightarray}{lrl}
\varphi_e & \mb{::=} & \f{pvo}\ \mb{|}\
                       \mathsf{val}(c)\ \mb{|}\
                       ( \varphi_e )\\
          & \mb{|}   & \mathsf{true}\ \mb{|}\
                       \mathsf{false}\ \mb{|}\
                       \neg \varphi_e\ \mb{|}\
                       \varphi_e \land \varphi_e\ \mb{|}\
                       \varphi_e \lor \varphi_e\ \mb{|}\
                       \varphi_e \limp \varphi_e\ \mb{|}\
                       \forall_{\f{mvd},\mb{\ldots},\f{mvd}} \varphi_e\ \mb{|}\
                       \exists_{\f{mvd},\mb{\ldots},\f{mvd}} \varphi_e\\
\f{pvo}  & \mb{::=} & X\ \mb{|}\ X(d,\mb{\ldots},d)\\
\end{tightarray}\]
Here $\f{pvo}$ stands for a propositional variable occurrence,
The descending order of operator precedence is: $\neg, \set{\land, \lor}, \limp, \set{\forall, \exists}$.
The infix operators $\land$, $\lor$ and $\limp$ associate to the right.

Parameterised boolean equations $\f{pb\_eqn}$:
\[\begin{tightarray}{lrl}
\f{pb\_eqn} & \mb{::=} & \sigma\,\f{pvd} = \varphi_e\\
\sigma      & \mb{::=} & \nu\ \mb{|}\ \mu\\
\f{pvd}     & \mb{::=} & X\ \mb{|}\ X(\f{mvd},\mb{\ldots},\f{mvd})\\
\end{tightarray}\]
Here $\sigma$ stands for a fixpoint symbol, and $\f{pvd}$ for a propositional variable declaration.

PBES specifications $\f{pbes\_spec}$:
\[\begin{tightarray}{lrl}
\f{pbes\_spec}      & \mb{::=} & \mb{(}\f{pbes\_spec\_elt}\mb{)^{*}}\\
\f{pbes\_spec\_elt} & \mb{::=} & \f{data\_spec}\\
                    & \mb{|} & \kwglob\ \mb{(}\f{mvd};\mb{)^{+}}\\
                    & \mb{|} & \kwpbes\ \mb{(}\f{pb\_eqn};\mb{)^{+}}\\
                    & \mb{|} & \kwinit\ \f{pvo};
\end{tightarray}\]
Here $\f{pbes\_spec\_elt}$ represents a PBES specification element. We impose the restriction that $\f{pbes\_spec}$ should contain precisely one occurrence of each of the keywords $\kwpbes$ and $\kwinit$.

\newpage

%%%%%%%%%%%%%%%%%%%%%%%%%%
\section{Table of symbols}
\label{sec:symbols}

In the toolset, a plain text format is used as opposed to the rich text format of the previous section.
A mapping from rich text to plain text symbols is provided in Table~\ref{table:symbols}.

\begin{table}[H]
\centering
%\footnotesize
\begin{tabular}{|l@{\qquad}L@{\qquad}l|}
\hline
Symbol                 & \textit{Rich}            & \verb+Plain+\\\hline
arrow                  & \to                      & \verb+->+\\
cross                  & \times                   & \verb+#+\\
diamond                & \diamond                 & \verb+<>+\\
\hline
standard sorts         & \sbool,\spos,\snat,\sint,\sreal
&\verb+Bool+, \verb+Pos+, \verb+Nat+, \verb+Int+, \verb+Real+\\
equality and inequality& \approx,\not\approx      & \verb+==+, \verb+!=+\\
logical operators      & \lnot,\land,\lor,\limp   & \verb+!+, \verb+&&+, \verb+||+, \verb+=>+\\
relational numeric operators & \leq,\geq          & \verb+<=+, \verb+>=+\\
relational set operators & \in, \subseteq, \subset& \verb+in+, \verb+<=+, \verb+<+\\
set operators          & ^{\_},\cup,\cap          & \verb+!+, \verb-+-, \verb+*+\\
list operators         & \cons,\snoc,\concat      & \verb+|>+, \verb+<|+, \verb-++-\\
lambda abstraction     & \lambda_{x:s}d           & \verb+lambda x:s.d+\\
universal quantification&\forall_{x:s}\varphi     & \verb+forall x:s.phi+\\
existential quantification&\exists_{x:s}\varphi   & \verb+exists x:s.phi+\\
\hline
deadlock               & \delta                   & \verb+delta+\\
internal action        & \tau                     & \verb+tau+\\
left merge             & \lmerge                  & \verb+||_+\\
sum                    & \sum_{x:s}p              & \verb+sum x:s.p+\\
allow                  & \allow{\set{a \mid b}}(p)& \verb+allow({a|b},p)+\\
block                  & \block{\set{a}}(p)       & \verb+block({a},p)+\\
hide                   & \hide{\set{a}}(p)        & \verb+hide({a},p)+\\
rename                 & \ren{\set{a \to b}}(p)   & \verb+rename({a -> b},p)+\\
communication          & \comm{\set{a \mid b\to c}}(p) & \verb+comm({a|b -> c},p)+\\
time                   & \at,\pinit,\pbefore      & \verb+@+, \verb+>>+, \verb+<<+\\
\hline
negation of ultimate delay & \nabla               & \verb+yaled+\\
ultimate delay         & \Delta                   & \verb+delay+\\
nil                    & \epsilon                 & \verb+nil+\\
fixpoint symbol        & \nu,\mu                  & \verb+nu+, \verb+mu+\\
maximal fixpoint       & \nu X(x:s = d).\varphi   & \verb+nu X(x:s = d).phi+\\
minimal fixpoint       & \mu X(x:s = d).\varphi   & \verb+mu X(x:s = d).phi+\\
\hline
\end{tabular}
\caption{Mapping from rich to plain text}
\label{table:symbols}
\end{table}

\end{document}
