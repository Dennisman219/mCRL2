\documentclass[a4paper,10pt]{article}
\usepackage{textcomp,amsmath,amssymb,amsthm,stmaryrd}
\usepackage{geometry}
\usepackage{mymath,mythm}

\theoremstyle{plain}
\newtheorem{thmfs}{Theorem}[section]
\theoremstyle{definition}
\newtheorem{tcase}[thmfs]{Test case}

\newcommand{\mcrl}{mCRL2}
\newcommand{\lps}{linear process specification}
\newcommand{\tool}{\textit{lpsinfo}}
\newcommand{\ti}{\textit}
\newcommand{\tb}{\textbf}

\newcommand{\ovr}{\overrightarrow}

\newcommand{\pp}{process parameter}
\newcommand{\pps}{process parameters}
\newcommand{\framework}{\textit{LPS framework} \cite{LPSframework}}

\newcommand{\bisim}{\frac{\leftrightarrow}{}}

\newcommand{\tab}{\hspace*{5.mm} \= \hspace*{5.mm} \= \hspace*{5.mm} \= \hspace*{5.mm} \= \hspace*{5.mm} \= \hspace*{5.mm}  \= \hspace*{5.mm}  \= \hspace*{5.mm}  \= \hspace*{5.mm} \= \hspace*{5.mm} \= \hspace*{5.mm}  \= \hspace*{5.mm}  \= \hspace*{5.mm}\kill}

\font \aap cmmi10        
\newcommand{\at}[1]{\mbox{\aap ,} #1}

%opening
\title{lpsinfo}
\author{F.P.M.Stappers}

\begin{document}

\maketitle

\begin{abstract}
This documentation describes the usage and implementation of the tool \tool\ within the \mcrl\ toolset.
Basically, \tool\ is a tool which displays information about a \lps\ (LPS).
\end{abstract}

\section{Introduction}
This \tool\ tool is a tool for the \mcrl\ studio. The tool is a program which reads from a \ti{input.lps}. The file \ti{input.lps} is
a file in \ti{.lps} format \cite{LPSformat}. We make use of the
\framework\ to read the \ti{input.lps}. The \tool\ will write
information about an \lps\ to stdout. 
\section{Definitions} \label{sec:def}

The equation below is a \lps\ in \mcrl : 
\begin{defn}\lps\ (LPS) \newline
%\at werkt niet
\begin{tabbing}
\tab
$X (\ovr{d: D}) = $ \> \> \> $ \sum_{i \in I} \sum_{\ovr{e_i: E_i}} \ovr{c_i} ( \ovr{d, e_i }) \rightarrow 
(a_i^1 (\ovr{f_{i,1}}(\ovr{d,e_i})) \vert \ldots \vert a_i^{n(i)}(\ovr{f_{i,n(i)}}(\ovr{d,e_i}))) \at \text{ } t_i(\ovr{d,e_i})  \cdot X(\ovr{g_i}(\ovr{d,e_i})) +$ \\ \\
\> \> \> $ \sum_{j \in j} \sum_{\ovr{e_j: E_j}} \ovr{c_j} ( \ovr{d, e_j} ) \rightarrow 
(a_j^1 (\ovr{f_{j,1}}(\ovr{d,e_j})) \vert \ldots \vert a_j^{n(j)}(\ovr{f_{j,n(j)}}(\ovr{d,e_j}))) \at \text{ } t_j(\ovr{d,e_j}) + $ \\ \\
\> \> \> $\sum_{\ovr{e_\delta:E_\delta}} \ovr{c_\delta} ( \ovr{d, e_\delta}) \rightarrow 
\delta \at \text{ } t_\delta(\ovr{d,e_\delta})$ 
\end{tabbing}

Where $I$ and $J$ are disjoint.\\
\end{defn}

\noindent If we speak about an LPS in this article we refer to Definition \ref{def}.  The different states 
%are $\ovr{d}$ 
of the process are represented by the data vector parameter $\ovr{d}:\ovr{D}$. $\ovr{D}$ may be a Cartesian product of $n$ data types, meaning that $\ovr{d}$ may consist of a tuple $(d_1, \ldots, d_n)$. The LPS expresses that in state $\ovr{d}$, it preforms (multi)actions $\rbrace a_i^0, \ldots , a_i^n \rbrace$, carrying data parameters $\ovr{f_{i,0}}(\ovr{d},\ovr{e_i}), \ldots , \ovr{f_{i,n(i)}}(\ovr{d},\ovr{e_i})$ and 
it can reach the new state $\ovr{g_i}(\ovr{d},e_i)$ under the condition that $\ovr{c_i}(\ovr{d},\ovr{e_i})$ is \ti{true}. So for each summand $i$ from $I$ we have at least a function $\ovr{g_i}: \ovr{D} \times \ovr{E} \rightarrow \ovr{D}$ and a function $\ovr{c_i}: \ovr{D} \times \ovr{E} \rightarrow \mathbb{B}$.
Data parameters $\ovr{e_i} : \ovr{E_i}$ are sum variables. These variables are used to range parameters over a data domain. 

\noindent For an more detailed explanation of \lps s  we refer to \cite{LPS_info}.
%\end{defn}

\section{Definition}\label{Def}
The \tool\ displays information about an LPS. To retrieve information about an LPS we make use of the \framework .  The \tool\ will display the following information\footnote{If a option is used, other information is displayed}:
\begin{itemize}
\item Number of summands.
\item Number of free variables.
\item Number of process parameters.
\item Number of actions.
\end{itemize}

\section{Usage}
This section describes the options which can be used.
\begin{tabbing}
\tab
\> \tb{-h} \> \> \tb{--help} \> \> \> Displays help about the \tool .\\
\> \tb{-v} \> \> \tb{--version} \> \> \> Dispays the version of the \tool .\\
\>         \> \> \tb{--pars} \> \> \> Prints the \pps\ of an LPS.\\
\>         \> \> \tb{--npars} \> \> \> Prints the number of \pps\ of an LPS.\\
\end{tabbing}
If no option is given the \tool\ displays the information as described in Section \ref{Def}.

\begin{thebibliography}{99}  \bibitem{LPSformat} Aad Mathijssen\\
   \textit{https://svn.win.tue.nl/viewcvs-checkout/MCRL2/trunk/specs/mcrl2.internal.txt},
   A description of the internal format of the mCRL2 language.
  \bibitem{LPSframework} J.W. Wesselink,
   \textit{http://www.win.tue.nl/~wieger/mcrl2/html/index.html}\\
   A C++ wrapper for the ATerm library.
   \bibitem{LPS_info} unknown author \\
   \textit{Article not ready at the moment},

\end{thebibliography}

\end{document}
