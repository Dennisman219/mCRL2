%TCIDATA{Version=5.50.0.2890}
%TCIDATA{LaTeXparent=1,1,pbes-implementation-notes.tex}
                      

\section{Definitions}

Parameterised Boolean Equation Systems (PBESs) are empty (denoted $\epsilon $%
) or finite sequences of fixed point equations, where each equation is of
the form $(\mu X(d{:}D)=\phi $ or $(\nu X(d{:}D)=\phi $. The left-hand side
of each equation consists of a \emph{fixed point symbol}, where $\mu $
indicates a least and $\nu $ a greatest fixed point, and a sorted predicate
variable $X$ of sort $D\rightarrow B$, taken from some countable domain of
sorted predicate variables $\mathcal{X}$. The right-hand side of each
equation is a predicate formula as defined below.

\begin{definition}
\emph{Predicate formulae} $\phi $ are defined by the following grammar: 
\begin{equation*}
\phi ::=b~|~X(e)~|~\lnot \phi ~|~\phi \oplus \phi ~|~\mathsf{Q}d:D.~\phi
\end{equation*}%
where $\oplus \in \{\wedge ,\vee ,\Rightarrow \}$, $\mathsf{Q}\in \{\forall
,\exists \}$, $b$ is a data term of sort $\mathsf{B}$, $X$ is a predicate
variable, $d$ is a data variable of sort $D$ and $e$ is a vector of data
terms.
\end{definition}

The set of predicate variables that occur in a predicate formula $\phi $,
denoted by $\mathsf{occ}$, is defined recursively as follows, for any
formulae $\phi _{1},\phi _{2}$: 
\begin{equation*}
\begin{array}{llll}
\mathsf{occ(}{b)} & =_{def}\emptyset & \mathsf{occ(}{X(e))} & =_{def}\{X\}
\\ 
\mathsf{occ(}{\phi _{1}\oplus \phi _{2})} & =_{def}\mathsf{occ(}{\phi _{1})}%
\cup \mathsf{occ(}{\phi _{2})}\qquad & \mathsf{occ}(\mathsf{Q}d:D{.~\phi
_{1})} & =_{def}\mathsf{occ(}{\phi _{1})}.%
\end{array}%
\end{equation*}%
Extended to equation systems, $\mathsf{occ}{(}\mathcal{E}{)}$ is the union
of all variables occurring at the right-hand side of equations in $\mathcal{E%
}$. Likewise, the set of predicate variable instantiations that occur in a
predicate formula $\phi $ is denoted by $\mathsf{iocc}$, and is defined
recursively as follows%
\begin{equation*}
\begin{array}{llll}
\mathsf{iocc(}{b)} & =_{def}\emptyset & \mathsf{iocc(}{X(e))} & 
=_{def}\{X(e)\} \\ 
\mathsf{iocc(}{\phi _{1}\oplus \phi _{2})} & =_{def}\mathsf{iocc(}{\phi _{1})%
}\cup \mathsf{iocc(}{\phi _{2})}\qquad & \mathsf{iocc}(\mathsf{Q}d:D{.~\phi
_{1})} & =_{def}\mathsf{iocc(}{\phi _{1})}.%
\end{array}%
\end{equation*}

For any equation system $\mathcal{E}$, the set of \emph{binding predicate
variables}, $\mathsf{bnd(}\mathcal{E})$, is the set of variables occurring
at the left-hand side of some equation in $\mathcal{E}$. Formally, we
define: 
\begin{equation*}
\begin{array}{llll}
\mathsf{bnd(}{\epsilon )} & =_{def}\emptyset \qquad & \mathsf{bnd(}{(\sigma
X(d{:}D)=\phi )~\mathcal{E)}} & =_{def}\mathsf{bnd(}{\mathcal{E)}}\cup \{X\}
\\ 
\mathsf{occ(}{\epsilon )} & =_{def}\emptyset \qquad & \mathsf{occ(}{(\sigma
X(d{:}D)=\phi )~\mathcal{E)}} & =_{def}\mathsf{occ(}{\mathcal{E)}}\cup 
\mathsf{occ(}{\phi )}.%
\end{array}%
\end{equation*}%
Let $\mathsf{dvar}(d)$ be the set of \emph{free data variables} occurring in
a data term $d$. The function $\mathsf{dvar}$ is extended to predicate
formulae using%
\begin{equation*}
\begin{array}{llll}
\mathsf{dvar(}{X(e))} & =_{def}\mathsf{dvar}(e) & \mathsf{dvar}(\mathsf{Q}d:D%
{.~\phi _{1})} & =_{def}\mathsf{dvar(}{\phi _{1})}\setminus \mathsf{dvar(}d{)%
}. \\ 
\mathsf{dvar(}{\phi _{1}\oplus \phi _{2})} & =_{def}\mathsf{dvar(}{\phi _{1})%
}\cup \mathsf{iocc(}{\phi _{2}).}\qquad &  & 
\end{array}%
\end{equation*}

The set of freely occurring predicate variables in $\mathcal{E}$, denoted $%
\mathsf{pvar}(\mathcal{E})$ is defined as $\mathsf{occ(}{\mathcal{E)}}%
\setminus \mathsf{bnd(}{\mathcal{E)}}$. An equation system $\mathcal{E}$ is
said to be \emph{well-formed} iff every binding predicate variable occurs at
the left-hand side of precisely one equation of $\mathcal{E}$. We only
consider well-formed equation systems in this paper.

An equation system $\mathcal{E}$ is called \emph{closed} if $\mathsf{pvar}(%
\mathcal{E})=\emptyset $ and \emph{open} otherwise. An equation $(\sigma
X(d:D)=\phi )$, where $\sigma $ denotes either the fixed point sign $\mu $
or $\nu $, is called \emph{data-closed} if the set of data variables that
occur freely in $\phi $ is contained in the set of variables induced by the
vector of variables $d$. An equation system is called \emph{data-closed} iff
each of its equations is data-closed.\newline

\begin{definition}
\emph{Action formulae} $\alpha $ are defined by the following grammar:%
\begin{equation*}
\alpha ::=b~|~\lnot \alpha ~|~\alpha \oplus \alpha ~|~\mathsf{Q}d:D.\alpha
~|~a(d)~|~\alpha \mbox{\aap ,}t
\end{equation*}%
where $\oplus \in \{\wedge ,\vee ,\Rightarrow \}$, $\mathsf{Q}\in \{\forall
,\exists \}$, $b$ is a data term of sort $\mathsf{B}$, $X$ is a predicate
variable, $d$ is a data variable of sort $D$ and $a$ is an action label.
\end{definition}

\begin{definition}
\emph{State formulae} $\phi $ are defined by the following grammar:%
\begin{equation*}
\phi ::=b~|~X(e)~|~\lnot \phi ~|~\phi \oplus \phi ~|~\mathsf{Q}d:D.~\phi
~|~\langle \alpha \rangle \phi ~|~[\alpha ]\phi ~|~\Delta ~|~\Delta
(t)~|~\nabla ~|~\nabla (t)~|~\sigma X(d{:}D:=e)
\end{equation*}%
where $\oplus \in \{\wedge ,\vee ,\Rightarrow \}$, $\mathsf{Q}\in \{\forall
,\exists \}$, $\sigma \in \{\mu ,\nu \}$, $b$ is a data term of sort $%
\mathsf{B}$, $X$ is a predicate variable, $d$ is a data variable of sort $D$
and $e$ is a vector of data terms and $\alpha $ is an action formula.
\end{definition}

\subsection{Well typedness constraints}

\subsubsection{well typedness constraints for PBES equations}

\begin{itemize}
\item the binding variable parameters have unique names

\item the names of the quantifier variables in the equation are disjoint
with the binding variable parameter names

\item within the scope of a quantifier variable in the formula, no other
quantifier variables with the same name may occur
\end{itemize}

\subsubsection{well typedness constraints for PBESs}

\begin{itemize}
\item the sorts occurring in the global variables of the equations are
declared in the data specification

\item the sorts occurring in the binding variable parameters are declared in
the data specification

\item the sorts occurring in the quantifier variables of the equations are
declared in the data specification

\item the binding variables of the equations have unique names (well
formedness)

\item the global variables occurring in the equations are declared in the
global variable specification

\item the global variables occurring in the equations with the same name are
identical

\item the declared global variables and the quantifier variables occurring
in the equations have different names

\item the predicate variable instantiations occurring in the equations match
with their declarations

\item the predicate variable instantiation occurring in the initial state
matches with the declaration

\item the data specification is well typed
\end{itemize}
