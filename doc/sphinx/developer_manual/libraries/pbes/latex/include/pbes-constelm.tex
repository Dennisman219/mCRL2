%TCIDATA{Version=5.50.0.2890}
%TCIDATA{LaTeXparent=1,1,pbes-implementation-notes.tex}


\section{Constant Parameter Detection and Elimination}

Let $\mathcal{E=(\sigma }_{1}X_{1}(d_{X_{1}}:D_{X_{1}})=\varphi
_{X_{1}})\cdots \mathcal{(\sigma }_{n}X_{n}(d_{X_{n}}:D_{X_{n}})=\varphi
_{X_{n}})$ be a PBES. Here, every $d_{X_i}$ represents a vector of parameters.
Furthermore, let $\hat{X}(\hat{e})$ be an initial state and let $\mathsf{eval}$
be an evaluator function on data expressions. We denote the $i$-th element
of a vector $x$ as $x[i]$. We also use mappings: for a mapping $c$, the image of
$i$ is denoted $c[i]$. The empty mapping is denoted with $\emptyset$ and the
image of an element not present in a mapping is $\bot$. Note that
$\emptyset[i] = \bot$ for all $i$. Then we define the algorithm
\textsc{PbesConstelm} as follows:$\qquad \qquad $%
\begin{equation*}
\begin{array}{l}
\text{\textsc{PbesConstelm(}}\mathcal{E}\text{, }\hat{X}(\hat{e}) \text{, }
\mathsf{eval}\text{\textsc{)}} \\
\text{\textbf{for }}X\in \mathsf{bnd}(\mathcal{E)}\text{ \textbf{do }}%
c_{X}:=\emptyset  \\
c_{\hat{X}}:=update(c_{\hat{X}},\mathsf{eval}(\hat{e})) \\
todo:= \{\hat{X}\} \\
\text{\textbf{while }}todo\neq \emptyset \text{ \textbf{do}} \\
\qquad \text{\textbf{choose }}X\in todo \\
\qquad todo:=todo\ \backslash \ \{X\} \\
\qquad \text{\textbf{for }}Y(e)\in \mathsf{iocc}(\varphi _{X}\mathcal{)}%
\text{ \textbf{do}} \\
\qquad \qquad \text{\textbf{if }}\mathsf{eval}(needs\_update(Y(e),\varphi
_{X})[d_{X}:=c_{X}])\neq false\text{ \textbf{then}} \\
\qquad \qquad \qquad c_{Y}^{\prime }:=update(c_{Y},\mathsf{eval}%
(e[d_{X}:=c_{X}])) \\
\qquad \qquad \qquad \text{\textbf{if }}c_{Y}^{\prime }\neq c_{Y}\text{
	\textbf{then}} \\
\qquad \qquad \qquad \qquad c_{Y}:=c_{Y}^{\prime } \\
\qquad \qquad \qquad \qquad todo:=todo\cup \{Y\} \\
constant\_parameters:=\{(X,i)\ |\ c_{X}[i]\neq d_{X}[i]\} \\
\text{\textbf{for }}i:=1\cdots n\text{ \textbf{do }}\varphi
_{X_{i}}:=\varphi _{X_{i}}[d_{X_{i}}:=c_{X_{i}}] \\
\text{\textbf{return }}constant\_parameters%
\end{array}%
\end{equation*}

where $update_{X}$ is defined as follows:%
\begin{equation*}
update(c,e)=_{def}c^{\prime },\text{ with }c^{\prime }[i]=\left\{
\begin{array}{ll}
\bot  & \text{if }c=\emptyset \text{ and }e=[] \\
e[i] & \text{if }c=\emptyset \text{ and }e[i]\text{ is constant} \\
c[i] & \text{if }e[i]=c[i] \\
d_{X}[i] & \text{otherwise}%
\end{array}%
\right.
\end{equation*}

and where $needs\_update$ is a boolean function that determines whether an
update should be performed. A safe choice for this function is the constant
function $true$. [Simon Janssen, 2008] originally proposed an alternative based
on a syntactical analysis of predicate formulae. The following is an improved
version of his definitions.

Let $c$ be defined as

\begin{equation*}
\begin{array}{llllll}
c_{T}(c) & = & c &
	c_{F}(c) & = & \lnot c \\
c_{T}(\lnot \varphi) & = & c_{F}(\varphi) &
	c_{F}(\lnot \varphi) & = & c_{T}(\varphi) \\
c_{T}(X(e)) & = & true &
	c_{F}(X(e)) & = & true \\
c_{T}(\mathsf{Q}d:{D}.\varphi) & = & \mathsf{Q}d:{D}.c_{T}(\varphi) &
	c_{F}(\mathsf{Q}d:{D}.\varphi) & = & \mathsf{Q}d:{D}.c_{F}(\varphi) \\
c_{T}(\varphi \land \psi) & = & c_{T}(\varphi) \land c_{T}(\psi) &
	c_{F}(\varphi \land \psi) & = & c_{F}(\varphi) \lor c_{F}(\psi) \\
c_{T}(\varphi \lor \psi) & = & c_{T}(\varphi) \lor c_{T}(\psi) &
	c_{F}(\varphi \lor \psi) & = & c_{F}(\varphi) \land c_{F}(\psi) \\
c_{T}(\varphi \Rightarrow \psi) & = & c_{F}(\varphi) \lor c_{T}(\psi) &
	c_{F}(\varphi \Rightarrow \psi) & = & c_{T}(\varphi) \land c_{F}(\psi)%
\end{array}%
\end{equation*}%
and let the set $cond$ be defined as%
\begin{equation*}
\begin{array}{lll}
cond(X(e),c) & = & \emptyset \\
cond(X(e),Y(f)) & = & \emptyset \\
cond(X(e),\lnot \varphi) & = & cond(X(e),\varphi) \\
cond(X(e),\varphi \land \psi) & = & \left\{
\begin{array}{ll}
\{c_{T}(\psi)\}\cup cond(X(e),\varphi) &
\text{if }X(e)\in \mathsf{iocc}(\varphi) \\
\{c_{T}(\varphi)\}\cup cond(X(e),\psi) & \text{%
if }X(e)\in \mathsf{iocc}(\psi) \\
\emptyset & \text{otherwise}%
\end{array}%
\right. \\
cond(X(e),\varphi \lor \psi) & = & \left\{
\begin{array}{ll}
\{c_{F}(\psi)\}\cup cond(X(e),\varphi) &
\text{if }X(e)\in \mathsf{iocc}(\varphi) \\
\{c_{F}(\varphi)\}\cup cond(X(e),\psi) & \text{%
if }X(e)\in \mathsf{iocc}(\psi) \\
\emptyset & \text{otherwise}%
\end{array}%
\right. \\
cond(X(e),\varphi \Rightarrow \psi) & = & \left\{
\begin{array}{ll}
\{c_{F}(\psi)\}\cup cond(X(e),\varphi) &
\text{if }X(e)\in \mathsf{iocc}(\varphi) \\
\{c_{T}(\varphi)\}\cup cond(X(e),\psi) & \text{%
if }X(e)\in \mathsf{iocc}(\psi) \\
\emptyset & \text{otherwise}%
\end{array}%
\right. \\
cond(X(e),\mathsf{Q}d:{D}.\varphi) & = & \left\{
\begin{array}{ll}
\{c_{\Gamma}(\forall d:{D}.\varphi)\}\cup \{\exists d:{D}.\theta \mid
\theta \in cond(X(e),\varphi)\} & \text{if }X(e)\in \mathsf{iocc}%
(\varphi) \\
\emptyset & \text{otherwise}%
\end{array}%
\right.
\end{array}%
\end{equation*}%
with $\mathsf{Q} \in \{\forall, \exists\}$. Then we define%
\begin{equation*}
\mathit{needs\_update}(X(e),\varphi )=\bigwedge_{c\in cond(X(e),\varphi
)} c
\end{equation*}

The implementation of these three functions is integrated into one recursive
traverser. The resulting condition is quadratic in the number of quantifier
alternations in which scope X(e) occurs and linear in the other operators. Most
PBESs stemming from model checking do not yield conditions larger than those
contained in the LPS. Furthermore, this traverser is only executed once, a
priori. This means that computing the conditions is relatively cheap.
\newpage
