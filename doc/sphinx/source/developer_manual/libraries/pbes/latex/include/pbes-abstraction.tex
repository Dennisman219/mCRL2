%TCIDATA{Version=5.50.0.2890}
%TCIDATA{LaTeXparent=1,1,pbes-implementation-notes.tex}
                      

\section{PBES abstraction}

Let $\mathcal{E=(\sigma }_{1}X_{1}(d_{X_{1}}:D_{X_{1}})=\varphi
_{X_{1}})\cdots \mathcal{(\sigma }_{n}X_{n}(d_{X_{n}}:D_{X_{n}})=\varphi
_{X_{n}})$ be a PBES, and let $V_{i}$ be a subset of the parameters in $%
d_{X_{i}}$ for $i=1\cdots n$. Let $e\in \{true,false\}$ be a data
expression. Then we define the algorithm $\mathsf{abstract}$ as follows:%
\begin{equation*}
\begin{array}{lll}
\mathsf{abstract}(d,V,e) & = & \left\{ 
\begin{array}{ll}
d & \text{\textsf{if}\ }\mathsf{freevar}(d)\cap V=\emptyset \\ 
e & \text{\textsf{otherwise}}%
\end{array}%
\right. \\ 
\mathsf{abstract}(\varphi \oplus \psi ,V,e) & = & \mathsf{abstract}(\varphi
,V,e)\oplus \mathsf{abstract}(\varphi ,V,e) \\ 
\mathsf{abstract}(\mathsf{Q}_{d_{_{1}}:D_{1},\cdots ,d_{m}:D_{m}}.\varphi
,V,e) & = & \mathsf{Q}_{d_{_{1}}:D_{1},\cdots ,d_{m}:D_{m}}.\mathsf{abstract}%
(\varphi ,V\setminus \{d_{_{1}}:D_{1},\cdots ,d_{m}:D_{m}\},e) \\ 
\mathsf{abstract}(\mathcal{\sigma }X(d_{_{1}}:D_{1},\cdots
,d_{m}:D_{m})=\varphi ,V,e) & = & \mathcal{\sigma }X(d_{_{1}}:D_{1},\cdots
,d_{m}:D_{m})=\mathsf{abstract}(\varphi ,V,e) \\ 
\mathsf{abstract}(\mathcal{E},[V_{1},\cdots ,V_{n}],e) & = & 
\begin{array}{c}
\mathcal{(\sigma }_{1}X_{1}(d_{X_{1}}:D_{X_{1}})=\mathsf{abstract(}\varphi
_{X_{1}},V_{1},e)) \\ 
\cdots \\ 
\mathcal{(\sigma }_{n}X_{n}(d_{X_{n}}:D_{X_{n}})=\mathsf{abstract(}\varphi
_{X_{n}},V_{n},e)%
\end{array}%
\end{array}%
\end{equation*}%
with $d$ a data expression, $\oplus \in \{\wedge ,\vee ,\Rightarrow \}$, and 
$\mathsf{Q\in \{\forall ,\exists \}}$ and $V\subset \{d_{_{1}}:D_{1},\cdots
,d_{m}:D_{m}\}$.

\subsection{Motivation}

The motivation for this algorithm is that if the solution of $\mathsf{%
abstract}(\mathcal{E},[V_{1},\cdots ,V_{n}],false)$ is $true$, this implies
that the solution of $\mathcal{E}$ is $true$ as well, and if the solution of 
$\mathsf{abstract}(\mathcal{E},[V_{1},\cdots ,V_{n}],true)$ is $false$, this
implies that the solution of $\mathcal{E}$ is $false$.\newpage 
