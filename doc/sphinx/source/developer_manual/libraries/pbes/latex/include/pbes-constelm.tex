%TCIDATA{Version=5.50.0.2890}
%TCIDATA{LaTeXparent=1,1,pbes-implementation-notes.tex}


\section{Constant Parameter Detection and Elimination}

Let $\mathcal{E=(\sigma }_{1}X_{1}(d_{X_{1}}:D_{X_{1}})=\varphi
_{X_{1}})\cdots \mathcal{(\sigma }_{n}X_{n}(d_{X_{n}}:D_{X_{n}})=\varphi
_{X_{n}})$ be a PBES. Here, every $d_{X_i}$ represents a vector of parameters.
Furthermore, let $\hat{X}(\hat{e})$ be an initial state and let $\mathsf{eval}$
be an evaluator function on data expressions. We denote the $i$-th element
of a vector $x$ as $x[i]$. We also use mappings: for a mapping $c$, the image of
$i$ is denoted $c[i]$. The empty mapping is denoted with $\emptyset$ and the
image of an element not present in a mapping is $\bot$. Note that
$\emptyset[i] = \bot$ for all $i$. Then we define the algorithm
\textsc{PbesConstelm} as follows:$\qquad \qquad $%
\begin{equation*}
\begin{array}{l}
\text{\textsc{PbesConstelm(}}\mathcal{E}\text{, }\hat{X}(\hat{e}) \text{, }
\mathsf{eval}\text{\textsc{)}} \\
\text{\textbf{for }}X\in \mathsf{bnd}(\mathcal{E)}\text{ \textbf{do }}%
c_{X}:=\emptyset  \\
c_{\hat{X}}:=update(c_{\hat{X}},\mathsf{eval}(\hat{e})) \\
todo:= \{\hat{X}\} \\
\text{\textbf{while }}todo\neq \emptyset \text{ \textbf{do}} \\
\qquad \text{\textbf{choose }}X\in todo \\
\qquad todo:=todo\ \backslash \ \{X\} \\
\qquad \text{\textbf{for }}Y(e)\in \mathsf{iocc}(\varphi _{X}\mathcal{)}%
\text{ \textbf{do}} \\
\qquad \qquad \text{\textbf{if }}\mathsf{eval}(needs\_update(Y(e),\varphi
_{X})[d_{X}:=c_{X}])\neq false\text{ \textbf{then}} \\
\qquad \qquad \qquad c_{Y}^{\prime }:=update(c_{Y},\mathsf{eval}%
(e[d_{X}:=c_{X}])) \\
\qquad \qquad \qquad \text{\textbf{if }}c_{Y}^{\prime }\neq c_{Y}\text{
	\textbf{then}} \\
\qquad \qquad \qquad \qquad c_{Y}:=c_{Y}^{\prime } \\
\qquad \qquad \qquad \qquad todo:=todo\cup \{Y\} \\
constant\_parameters:=\{(X,i)\ |\ c_{X}[i]\neq d_{X}[i]\} \\
\text{\textbf{for }}i:=1\cdots n\text{ \textbf{do }}\varphi
_{X_{i}}:=\varphi _{X_{i}}[d_{X_{i}}:=c_{X_{i}}] \\
\text{\textbf{return }}constant\_parameters%
\end{array}%
\end{equation*}

where $update_{X}$ is defined as follows:%
\begin{equation*}
update(c,e)=_{def}c^{\prime },\text{ with }c^{\prime }[i]=\left\{
\begin{array}{ll}
\bot  & \text{if }c=\emptyset \text{ and }e=[] \\
e[i] & \text{if }c=\emptyset \text{ and }e[i]\text{ is constant} \\
c[i] & \text{if }e[i]=c[i] \\
d_{X}[i] & \text{otherwise}%
\end{array}%
\right.
\end{equation*}

and where $needs\_update$ is a boolean function that determines whether an
update should be performed. A safe choice for this function is the constant
function $true$. In [Simon Janssen, 2008] the following alternative is
proposed.

Let $c_{T}$ and $c_{F}$ be defined as

\begin{equation*}
\begin{array}{lll}
c_{T}(c) & = & c \\
c_{F}(c) & = & \lnot c \\
c_{T}(\lnot \varphi ) & = & c_{F}(\varphi ) \\
c_{F}(\lnot \varphi ) & = & c_{T}(\varphi ) \\
c_{\Gamma }(X(e)) & = & false \\
c_{\Gamma }(\mathsf{Q}d:{D}.\varphi ) & = & \mathsf{Q}d:{D}.c_{\Gamma
}(\varphi ) \\
c_{T}(\varphi \wedge \psi ) & = & c_{T}(\varphi )\wedge c_{T}(\psi ) \\
c_{T}(\varphi \vee \psi ) & = & c_{T}(\varphi )\vee c_{T}(\psi ) \\
c_{T}(\varphi \Rightarrow \psi ) & = & c_{F}(\varphi )\vee c_{T}(\psi ) \\
c_{F}(\varphi \wedge \psi ) & = & c_{F}(\varphi )\vee c_{F}(\psi ) \\
c_{F}(\varphi \vee \psi ) & = & c_{F}(\varphi )\wedge c_{F}(\psi ) \\
c_{F}(\varphi \Rightarrow \psi ) & = & c_{T}(\varphi )\wedge c_{F}(\psi )%
\end{array}%
\end{equation*}%
and let the multi sets $cond_{T}$ and $cond_{F}$ be defined as%
\begin{equation*}
\begin{array}{lll}
cond_{\Gamma }(X(e),c) & = & \emptyset \\
cond_{\Gamma }(X(e),Y(f)) & = & \emptyset \\
cond_{\Gamma }(X(e),\lnot \varphi ) & = & cond_{\Gamma }(X(e),\varphi ) \\
cond_{\Gamma }(X(e),\varphi \oplus \psi ) & = & \left\{
\begin{array}{cc}
\{c_{\Gamma }(\varphi \oplus \psi )\}\cup cond_{\Gamma }(X(e),\varphi ) &
\text{if }X(e)\in \mathsf{iocc}(\varphi ) \\
\{c_{\Gamma }(\varphi \oplus \psi )\}\cup cond_{\Gamma }(X(e),\psi ) & \text{%
if }X(e)\in \mathsf{iocc}(\psi ) \\
\emptyset & \text{otherwise}%
\end{array}%
\right. \\
cond_{\Gamma}(X(e),\mathsf{Q}d:{D}.\varphi ) & = & \left\{
\begin{array}{cc}
\{c_{\Gamma}(\mathsf{Q}d:{D}.\varphi )\}\cup \{\mathsf{Q}d:{D}.\theta \ |\
\theta \in cond_{\Gamma }(X(e),\varphi )\} & \text{if }X(e)\in \mathsf{iocc}%
(\varphi ) \\
\emptyset & \text{otherwise}%
\end{array}%
\right.
\end{array}%
\end{equation*}%
with $\Gamma \in \{T,F\}$ and $\mathsf{Q} \in \{\forall, \exists\}$. Then we
define%
\begin{equation*}
needs\_update(X(e),\varphi )=\dbigwedge\limits_{c\in cond_{T}(X(e),\varphi
)\cup cond_{F}(X(e),\varphi )}\lnot c
\end{equation*}

\begin{remark}
Note that a direct implementation of this function is extremely inefficient.
It is expected that a more suitable representation can be found.\newpage
\end{remark}
