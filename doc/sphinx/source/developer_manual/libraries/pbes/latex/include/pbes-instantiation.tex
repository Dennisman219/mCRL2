%TCIDATA{Version=5.50.0.2890}
%TCIDATA{LaTeXparent=1,1,pbes-implementation-notes.tex}
                      

\section{PBES instantiation}

In this section we describe two instantiation algorithms for PBESs.

\subsection{Lazy algorithm}

In this section we describe an implementation of the lazy instantiation
algorithm \textsc{Pbes2besLazy} that uses instantiation to compute a BES. It
takes two extra parameters, an injective function $\rho $ that renames
proposition variables to predicate variables, and a rewriter $R$ that
eliminates quantifiers from predicate formulae. Let $\mathcal{E=(\sigma }%
_{1}X_{1}(d_{1}:D_{1})=\varphi _{1})\ldots \mathcal{(\sigma }%
_{n}X_{n}(d_{n}:D_{n})=\varphi _{n})$ be a PBES, and $X_{init}(e_{init})$ an
initial state.%
\begin{equation*}
\begin{array}{l}
\text{\textsc{Pbes2besLazy(}}\mathcal{E}\text{, }X_{init}(e_{init})\text{, }R%
\text{, }\rho \text{\textsc{)}} \\ 
\text{\textbf{for }}i:=1\cdots n\text{ \textbf{do }}\mathcal{E}_{i}:=\epsilon
\\ 
todo:=\{R(X_{init}(e_{init}))\} \\ 
done:=\emptyset \\ 
\text{\textbf{while }}todo\neq \emptyset \text{ \textbf{do}} \\ 
\qquad \text{\textbf{choose }}X_{k}(e)\in todo \\ 
\qquad todo:=todo\ \backslash \ \{X_{k}(e)\} \\ 
\qquad done:=done\cup \{X_{k}(e)\} \\ 
\qquad X^{e}:=\rho (X_{k}(e)) \\ 
\qquad \psi ^{e}:=R(\varphi _{k}[d_{k}:=e]) \\ 
\qquad \mathcal{E}_{k}:=\mathcal{E}_{k}(\mathcal{\sigma }_{k}X^{e}=\rho
(\psi ^{e})) \\ 
\qquad todo:=todo\cup \{Y(f)\in \mathsf{occ}(\psi ^{e})\ |\ Y(f)\notin done\}
\\ 
\text{\textbf{return }}\mathcal{E}_{1}\cdots \mathcal{E}_{n},%
\end{array}%
\end{equation*}%
where $\rho $ is extended from predicate variables to quantifier free
predicate formulae using%
\begin{equation*}
\begin{array}{cc}
\rho (b)=_{def}b & \quad \rho (\varphi \oplus \psi )=_{def}\rho (\varphi
)\oplus \rho (\psi )%
\end{array}%
\end{equation*}%
\pagebreak

\subsection{Finite algorithm}

Let $\mathcal{E=(\sigma }_{1}X_{1}(d_{1}:D_{1},e_{1}:E_{1})=\varphi
_{1})\cdots \mathcal{(\sigma }_{n}X_{n}(d_{n}:D_{n},e_{n}:E_{n})=\varphi
_{n})$ be a PBES. We assume that all data sorts $D_{i}$ are finite and all
data sorts $E_{i}$ are infinite. Let $r$ be a data rewriter, and let $\rho $
be an injective function that creates a unique predicate variable from a
predicate variable name and a data value according to $\rho
(X(d:D,e:E),d_{0})\rightarrow Y(e:E)$, where $D$ is finite and $E$ is
infinite and $d_{0}\in D$. Note that $D$ and $D_{i}$ may be
multi-dimensional sorts.%
\begin{equation*}
\begin{array}{l}
\text{\textsc{Pbes2besFinite(}}\mathcal{E}\text{, }r\text{, }\rho \text{%
\textsc{)}} \\ 
\text{\textbf{for }}i:=1\cdots n\text{ \textbf{do}} \\ 
\qquad \mathcal{E}_{i}:=\{\mathcal{\sigma }_{i}\rho (X_{i},d)=R(\varphi
_{k}[d_{k}:=d])\ |\ d\in D_{i}\} \\ 
\text{\textbf{return }}\mathcal{E}_{1}\cdots \mathcal{E}_{n},%
\end{array}%
\end{equation*}%
with $R$ a rewriter on pbes expressions that is defined as follows:%
\begin{eqnarray*}
R(b) &=&b \\
R(\lnot \varphi ) &=&\lnot R(\varphi ) \\
R(\varphi \oplus \psi ) &=&R(\varphi )\oplus (\psi ) \\
R(X_{i}(d,e)) &=&\left\{ 
\begin{array}{cc}
\rho (X_{i},r(d))(r(e)) & \text{if }FV(d)=\emptyset \\ 
\dbigvee\limits_{d_{i}\in D_{i}}r(d=d_{i})\wedge \rho (X_{i},d_{i})(r(e)) & 
\text{if }FV(d)\neq \emptyset%
\end{array}%
\right. \\
R(\forall _{d:D}.\varphi ) &=&\forall _{d:D}.R(\varphi ) \\
R(\exists _{d:D}.\varphi ) &=&\exists _{d:D}.R(\varphi )
\end{eqnarray*}%
where $\oplus \in \{\vee ,\wedge ,\Rightarrow \}$, $b$ a data expression and 
$\varphi $ and $\psi $ pbes expressions and $FV(d)$ is the set of free
variables appearing in $d$.\newpage
