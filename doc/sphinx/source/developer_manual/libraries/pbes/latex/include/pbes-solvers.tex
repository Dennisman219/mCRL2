%TCIDATA{Version=5.50.0.2890}
%TCIDATA{LaTeXparent=1,1,pbes-implementation-notes.tex}
                      

\section{PBES solvers}

In this section we describe algorithms for solving PBESs.

\subsection{Recursive procedure for solving structure graphs}

Let $A$ be a set of vertices of a structure graph $G=(V,E,d,r)$. Then we
define the following algorithms for computing an attractor set of $%
A\subseteq V$. As a side effect, a strategy $\tau $ is computed. The
following notations are used for player Even and Odd:%
\begin{equation*}
\begin{array}{cc}
Even & Odd \\ 
0 & 1 \\ 
\Diamond  & \square  \\ 
\vee  & \wedge  \\ 
\blacktriangledown  & \blacktriangle  \\ 
\nu  & \mu  \\ 
\top  & \bot 
\end{array}%
\end{equation*}%
In the algoritms below we assume that the decoration $\blacktriangledown $
has value 0 and the decoration $\blacktriangle $ has value 1.%
\begin{equation*}
\begin{array}{l}
\text{\textsc{Attr(}}A,\alpha \text{\textsc{)}} \\ 
todo:=\bigcup\limits_{u\in A}\left( pred(u)\setminus A\right)  \\ 
\text{\textbf{while }}todo\neq \emptyset \text{ \textbf{do}} \\ 
\qquad \text{\textbf{choose }}u\in todo \\ 
\qquad todo:=todo\ \backslash \ \{u\} \\ 
\qquad \text{\textbf{if} }attr_{\alpha }(A,u)\text{ \textbf{then}} \\ 
\qquad \qquad \text{\textbf{if} }d(u)\neq 1 - \alpha \text{ \textbf{then }}\tau
\lbrack u]:=v\text{ \textbf{with} }v\in A\cap succ(u) \\ 
\qquad \qquad A:=A\cup \{u\} \\ 
\qquad \qquad todo:=todo\cup \left( pred(u)\setminus A\right)  \\ 
\text{\textbf{return }}A%
\end{array}%
\end{equation*}%
where%
\begin{eqnarray*}
pred(v) &=&\{u\in V\mid (u,v)\in E\} \\
succ(u) &=&\{v\in V\mid (u,v)\in E\}
\end{eqnarray*}%
and where%
\begin{equation*}
attr_{\alpha }(A,u)=\{d(u)\neq 1-\alpha \vee \left( d(u)\neq \alpha \wedge
succ(u)\subseteq A\right) \}
\end{equation*}%
The following algorithm is used to compute a partitioning of $V$ into $%
\left( W_{0},W_{1}\right) $ of vertices $W_{1}$ that represent equations
evaluating to false and vertices $W_{0}$ that represent equations evaluating
to true. A precondition of this algorithm is that it contains no nodes with
decoration $\top $ or $\bot $.%
\begin{equation*}
\begin{array}{l}
\text{\textsc{SolveRecursive(}}V\text{\textsc{)}} \\ 
\text{\textbf{if} }V=\emptyset \text{ \textbf{then return }}(\emptyset
,\emptyset ) \\ 
m:=\min (\{r(v)\mid v\in V\}) \\ 
\alpha :=m\func{mod}2 \\ 
U:=\{v\in V\mid r(v)=m\} \\ 
\text{\textbf{for} }u\in U\text{ \textbf{if} }d(u)=\alpha \wedge succ(u)\neq
\emptyset \text{ \textbf{then }}\tau \lbrack u]:=v\text{ \textbf{with} }v\in
succ(u) \\ 
A:=\text{\textsc{Attr}}\left( U,\alpha \right)  \\ 
W_{0}^{\prime },W_{1}^{\prime }:=\text{\textsc{SolveRecursive(}}V\setminus A%
\text{)} \\ 
B:=\text{\textsc{Attr}}\left( W_{1-\alpha }^{\prime },1-\alpha \right)  \\ 
\text{\textbf{if} }W_{1-\alpha }^{\prime }=B\text{ \textbf{then}} \\ 
\qquad W_{\alpha },W_{1-\alpha }:=A\cup W_{\alpha }^{\prime },B \\ 
\text{\textbf{else}} \\ 
\qquad W_{0},W_{1}:=\text{\textsc{SolveRecursive(}}V\setminus B\text{)} \\ 
\qquad W_{1-\alpha }:=W_{1-\alpha }\cup B \\ 
\text{\textbf{return }}W_{0},W_{1}%
\end{array}%
\end{equation*}%
where%
\begin{equation*}
succ(u,U)=\left\{ 
\begin{array}{ll}
succ(u)\cap U & \text{if }succ(u)\cap U\neq \emptyset  \\ 
succ(u) & \text{otherwise}%
\end{array}%
\right. 
\end{equation*}

The algorithm can be extended to sets $V$ containing nodes with decoration $%
\top $ or $\bot $ as follows:%
\begin{equation*}
\begin{array}{l}
\text{\textsc{SolveRecursiveExtended(}}V\text{\textsc{)}} \\ 
V_{1}:=Attr\left( \{v\in V\mid v\bot \},1\right) \\ 
V_{0}:=Attr\left( \{v\in V\mid v\top \},0\right) \\ 
\left( W_{0},W_{1}\right) :=\text{\textsc{SolveRecursive(}}V\setminus
(V_{0}\cup V_{1})\text{\textsc{)}} \\ 
\text{\textbf{return }}\left( W_{0}\cup V_{1},W_{0}\cup V_{1}\right)%
\end{array}%
\end{equation*}%
Note that a possible optimization of the \textsc{SolveRecursive} algorithm
is to insert the following shortcuts after the initial strategy assignment.
This doesn't seem to have much effect in practice, so currently it isn't
enabled in the code.%
\begin{equation*}
\begin{array}{l}
\text{\textbf{if }}h=m\wedge even(m)\text{ \textbf{then return} }(\emptyset
,V) \\ 
\text{\textbf{if }}h=m\wedge odd(m)\text{ \textbf{then return} }(V,\emptyset
)%
\end{array}%
\end{equation*}%
\newpage
