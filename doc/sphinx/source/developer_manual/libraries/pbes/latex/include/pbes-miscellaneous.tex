%TCIDATA{Version=5.50.0.2890}
%TCIDATA{LaTeXparent=1,1,pbes-implementation-notes.tex}
                      

\section{ReachableVariables}

Let $\mathcal{E=(\sigma }_{1}X_{1}(d_{X_{1}}:D_{X_{1}})=\varphi
_{X_{1}})\cdots \mathcal{(\sigma }_{n}X_{n}(d_{X_{n}}:D_{X_{n}})=\varphi
_{X_{n}})$ be a PBES, and let $X_{init}(e_{init})$ be the initial state. The
algorithm \textsc{ReachableVariables} computes the reachable predicate
variables.

\begin{equation*}
\begin{array}{l}
\text{\textsc{ReachableVariables(}}\mathcal{E},X_{init}\text{\textsc{)}} \\ 
\text{visited }:=\{X_{init}\};\text{ explored }:=\emptyset \\ 
\text{\textbf{while} visited}\neq \emptyset \\ 
\qquad \text{\textbf{choose} }X_{i}\in \text{visited} \\ 
\qquad \text{visited}:=\text{visited}\setminus \{X_{i}\} \\ 
\qquad \text{explored }:=\text{ explored}\cup \{X_{i}\} \\ 
\qquad \text{\textbf{for each} }X_{j}(e)\text{ }\in \mathsf{iocc(}\varphi
_{X_{i}}{)} \\ 
\qquad \qquad \text{\textbf{if} }X_{j}\notin \text{ explored} \\ 
\qquad \qquad \qquad \text{visited }:=\text{ visited}\cup \{X_{j}\} \\ 
\text{\textbf{return} explored}%
\end{array}%
\end{equation*}%
\pagebreak

\section{ \protect\appendix}

\paragraph{ATerm format}

\begin{equation*}
\begin{array}{ll}
\mathtt{<DataExpr>} & c \\ 
\mathtt{StateTrue} & true \\ 
\mathtt{StateFalse} & false \\ 
\mathtt{StateNot(<StateFrm>)} & \lnot \varphi \\ 
\mathtt{StateAnd(<StateFrm>,<StateFrm>)} & \varphi \wedge \varphi \\ 
\mathtt{StateOr(<StateFrm>,<StateFrm>)} & \varphi \vee \varphi \\ 
\mathtt{StateImp(<StateFrm>,<StateFrm>)} & \varphi \Rightarrow \varphi \\ 
\mathtt{StateForall(<DataVarId>+,<StateFrm>)} & \forall x{:}D.\varphi \\ 
\mathtt{StateExists(<DataVarId>+,<StateFrm>)} & \exists x{:}D.\varphi \\ 
\mathtt{StateMust(<RegFrm>,<StateFrm>)} & \langle \alpha \rangle \varphi \\ 
\mathtt{StateMay(<RegFrm>,<StateFrm>)} & [\alpha ]\varphi \\ 
\mathtt{StateYaled} & \nabla \\ 
\mathtt{StateYaledTimed(<DataExpr>)} & \nabla (t) \\ 
\mathtt{StateDelay} & \Delta \\ 
\mathtt{StateDelayTimed(<DataExpr>)} & \Delta (t) \\ 
\mathtt{StateVar(<String>,<DataExpr>\ast )} & X(d) \\ 
\mathtt{StateNu(<String>,<DataVarIdInit>\ast ,<StateFrm>)} & \nu X(x{:}%
D:=d).~\varphi \\ 
\mathtt{StateMu(<String>,<DataVarIdInit>\ast ,<StateFrm>)} & \mu X(x{:}%
D:=d).~\varphi%
\end{array}%
\end{equation*}

\paragraph{Naming conventions}

\begin{equation*}
\begin{array}{lcl}
\mathsf{left}(\varphi \otimes \psi ) & = & \varphi  \\ 
\mathsf{right}(\varphi \otimes \psi ) & = & \psi  \\ 
\arg (\lnot \varphi ) & = & \varphi  \\ 
\arg (\forall d:D.\varphi )=\arg (\exists d:D.\varphi ) & = & \varphi  \\ 
\mathsf{var}(\forall d:D.\varphi )=\mathsf{var}(\exists d:D.\varphi ) & = & 
d:D \\ 
\arg (\left\langle \alpha \right\rangle \varphi )=\arg ([\alpha ]\varphi ) & 
= & \varphi  \\ 
\mathsf{act}(\left\langle \alpha \right\rangle \varphi )=\mathsf{act}%
([\alpha ]\varphi ) & = & \alpha  \\ 
\mathsf{time}(\nabla (t))=\mathsf{time}(\Delta (t)) & = & t \\ 
\mathsf{var}(X(d:D)) & = & d:D \\ 
\mathsf{\arg }(\sigma X(d:D:=e).\varphi ) & = & \varphi  \\ 
\mathsf{name}(\sigma X(d:D:=e).\varphi ) & = & X \\ 
\mathsf{var}(\sigma X(d:D:=e).\varphi ) & = & d:D \\ 
\mathsf{val}(\sigma X(d:D:=e).\varphi ) & = & e%
\end{array}%
\end{equation*}%
where $\sigma $ is either $\mu $ or $\nu $, and $\otimes $ is either $\wedge 
$, $\vee $, or $\Rightarrow $.
