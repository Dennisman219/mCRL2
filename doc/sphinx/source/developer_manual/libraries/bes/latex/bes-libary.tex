\documentclass[a4paper,11pt]{article}

\usepackage{a4wide}
\usepackage{amsmath}
\usepackage{amssymb}
\usepackage{amsfonts}
\usepackage[british]{babel}
\usepackage{xspace}
\usepackage{graphicx}
\usepackage{grffile}
\usepackage{hyperref}

\newcommand{\ie}{\textit{i.e.}\xspace }
\newcommand{\eg}{\textit{e.g.}\xspace }
\newcommand{\viz}{\textit{viz.}\xspace }
%\newcommand{\tle}{\ensuremath{\mathrel{\trianglelefteqslant}}}
\newcommand{\tle}{\ensuremath{\triangleleft}}
\newcommand{\isdef}{\ensuremath{\stackrel{\Delta}{=}}}
\newcommand{\ap}{{:}}

\newcommand{\floor}[1]{\ensuremath{\lfloor #1 \rfloor}}
\newcommand{\ceil}[1]{\ensuremath{\lceil #1 \rceil}}
\renewcommand{\leq}{\ensuremath{\leqslant}}    % make <= look nicer
\renewcommand{\geq}{\ensuremath{\geqslant}}    % make >= look nicer

% Font style definitions
\newcommand{\rel}[1]{\ensuremath{\mathrel{#1}}}
\newcommand{\mc}[1]{\ensuremath{\mathcal{#1}}}

\newcommand{\bool}{\ensuremath{\mathbb{B}}}   % set of booleans
\newcommand{\nat}{\ensuremath{\mathbb{N}}}   % set of booleans
\newcommand{\true}{\ensuremath{\mathsf{true}}}
\newcommand{\false}{\ensuremath{\mathsf{false}}}

% BES related definitions
\newcommand{\bnd}[1]{\ensuremath{\mathsf{bnd}(#1)}}
\newcommand{\var}[1]{\ensuremath{\mathsf{var}(#1)}}
\newcommand{\occ}[1]{\ensuremath{\mathsf{occ}(#1)}}
\newcommand{\rhs}[2]{\ensuremath{\mathsf{rhs}(#1, #2)}}
\newcommand{\sem}[1]{\ensuremath{[\![ #1 ]\!]}}

\newcommand{\ad}[1]{\ensuremath{\mathsf{rank}(#1)}}
\newcommand{\ah}[1]{\ensuremath{\mathsf{ah}(#1)}}
\newcommand{\cad}[3]{\ensuremath{\mathsf{rank}_{#1,#2}(#3)}}
\newcommand{\op}[1]{\ensuremath{\mathsf{op}(#1)}}

\newcommand{\limitrank}[2]{\ensuremath{\mathsf{limrank}_{#1}(#2)}}
\newcommand{\climitrank}[5]{\ensuremath{\mathsf{limrank}_{#1,#2,#3}(#4,#5)}}

\newcommand{\imp}{\ensuremath{\Rightarrow}}
\newcommand{\subst}[1]{\ensuremath{\sigma(#1)}}

\newcommand{\successor}{\mathit{successor}}

\usepackage{pifont} % \ding

% \todo{Note}. Prints Note in the margin.
\newcommand\todo[1]{\ding{56}\marginpar[\raggedleft\footnotesize \sf #1]{
  \raggedright\footnotesize \sf #1
}}

%comment command
\newcommand{\comment}[1]{\begin{quotation} {\sf *** #1 ***} \end{quotation}}

% Environments
\newcounter{theoremcnt}[section]
\renewcommand{\thetheoremcnt}{\thesection.\arabic{theoremcnt}}

\newenvironment{example}%
{\begin{trivlist}\refstepcounter{theoremcnt}
\item{\bf Example \thetheoremcnt.}}
{\end{trivlist}}

\newenvironment{definition}%
{\begin{trivlist}\refstepcounter{theoremcnt}
\item{\bf Definition \thetheoremcnt.}}
{\end{trivlist}}

\newenvironment{lemma}%
{\begin{trivlist}\refstepcounter{theoremcnt}
\item{\bf Lemma \thetheoremcnt.}}
{\end{trivlist}}

\newenvironment{theorem}%
{\begin{trivlist}\refstepcounter{theoremcnt}
\item{\bf Theorem \thetheoremcnt.}}
{\end{trivlist}}

\newenvironment{corollary}%
{\begin{trivlist}\refstepcounter{theoremcnt}
\item{\bf Corollary \thetheoremcnt.}}
{\end{trivlist}}

\newenvironment{property}%
{\begin{trivlist}\refstepcounter{theoremcnt}
\item{\bf Property \thetheoremcnt.}}
{\end{trivlist}}

\newenvironment{remark}%
{\begin{trivlist}\refstepcounter{theoremcnt}
\item{\bf Remark \thetheoremcnt.}}{\end{trivlist}}

\newenvironment{convention}%
{\begin{trivlist}\refstepcounter{theoremcnt}
\item{\bf Convention \thetheoremcnt.}}{\end{trivlist}}

\newenvironment{exercise}%
{\begin{trivlist}\refstepcounter{theoremcnt}
\item{\bf Exercise \thetheoremcnt.}}{\end{trivlist}}

% \newenvironment{algorithm}%
% {\begin{trivlist}\refstepcounter{theoremcnt}
% \item{\bf Algorithm \thetheoremcnt.}}{\end{trivlist}}

%nicer QED
% alternative
 \newcommand{\openbox}{\leavevmode
   \hbox to.77778em{%
   \hfil\vrule
   \vbox to.675em{\hrule width.6em\vfil\hrule}%
   \vrule\hfil}}
 \def\squareforqed{\hbox{\rlap{$\sqcap$}$\sqcup$}}
 \def\qed{\ifmmode\squareforqed\else{\unskip\nobreak\hfil
 \penalty50\hskip1em\null\nobreak\hfil\squareforqed
 \parfillskip=0pt\finalhyphendemerits=0\endgraf}\fi}
 \newenvironment{proof}{\textbf{Proof}\thinspace}{\qed}
 \newenvironment{proofs}{\textbf{Proofs}\thinspace}{\qed}

\title{Some notes on a BES library}
\author{Jeroen Keiren}

\begin{document}
\maketitle

\section{Boolean Equation Systems}
Boolean Equation Systems (BESs) \cite{Mader1997} are a class of equation
system that can be employed to perform model checking of modal $\mu$-calculus
formulae. Basically, BESs are finite sequences of least and greatest fixpoint
equations, where each right-hand side of an equation is a proposition in
positive form. It has been shown \cite{Mader1997} that solving a BES is equivalent
to the model-checking problem. BESs are used for this purpose in \eg the tool sets
CADP \cite{GaravelLMS2007} and mCRL2
\cite{GrooteMWU2007}. Several algorithms for solving BESs exist, see
\cite{Mader1997, Keinanen2006}. Furthermore there are efficient algorithms
for some special cases, see \cite{Mateescu2003, Keinanen2006}. We formally introduce
the theory required for understanding the results obtained in this paper.

\begin{definition}
We assume a set $\mc{X}$ of Boolean variables, with typical elements $X$, $X_1$,
$X_2$, $\dots$ and a type $\bool$ with elements $\true$, $\false$ representing
the Booleans. Furthermore we have fixpoint symbols $\mu$ for least
fixpoint and $\nu$ for greatest fixpoint.

A Boolean Equation System is a system of fixpoint equations, inductively defined
as follows:
\begin{itemize}
\item $\epsilon$ is the empty BES
\item if $\mc{E}$ is a BES, then $(\sigma X = f)\mc{E}$ is also a BES, with
$\sigma \in \{ \mu, \nu \}$ a fixpoint symbol and $f$ a formula
over $\mc{X}$, defined by the following grammar:
\begin{equation*}
  f,g ::= c \mid X \mid \neg f \mid f \land g \mid f \lor g \mid f \imp g
\end{equation*}
\comment{Do we also need to include $if(f,g,h)$ as expression; it is available
in the deprecated BES header? Its use is in the construction of BDD structures for the comparison of formulae. Note that this could also be encoded as $(f \imp g) \land (\neg f \imp g)$}
where $X \in \mc{X}$ is a proposition variable of type $\bool$ and $c \in \{ \true, \false \}$ is a Boolean constant. We refer to the set of formulae over
$\mc{X}$ with $\mc{F}_{\mc{X}}$.
\end{itemize}
\end{definition}

%
For any equation system $\mc{E}$, the set of \emph{bound proposition variables},
$\bnd{\mc{E}}$, is the set of variables occurring at the left-hand side of
some equation in $\mc{E}$. The set of \emph{occurring proposition variables},
$\occ{\mc{E}}$, is the set of variables occurring at the right-hand side of
some equation in $\mc{E}$. 
\begin{align*}
\bnd{\epsilon} & \isdef \emptyset &
\bnd{(\sigma X = f)~\mc{E}} & \isdef \bnd{\mc{E}} \cup \{X\}\\
  \occ{\epsilon} & \isdef \emptyset &
  \occ{(\sigma X = f)~\mc{E}} & \isdef \occ{\mc{E}} \cup \occ{f} 
\end{align*}
%
where $\occ{f}$ is defined inductively as follows:
\begin{align*}
  \occ{c} & \isdef \emptyset  & \occ{X}     & \isdef \{X\}   \\
  \occ{\neg f} & \isdef \occ{f} &
  \occ{f \imp g} & \isdef \occ{f} \cup \occ{g} \\
  \occ{f \vee g} & \isdef \occ{f} \cup \occ{g} & 
  \occ{f \wedge g} & \isdef \occ{f} \cup \occ{g} 
\end{align*}
%
BESs $\mc{E}$ and $\mc{F}$ with $\bnd{\mc{E}} \cap \bnd{\mc{F}} = \emptyset$
are referred to as \emph{non-conflicting} BESs.

As usual, we consider only equation systems $\mc{E}$ in which
every proposition variable occurs at the left-hand side of at most one
equation of $\mc{E}$. We define an ordering $\tle$
on bound variables of an equation system $\mc{E}$, where $X \tle X'$
indicates that the equation for $X$ precedes the equation for $X'$.

Proposition formulae are interpreted in a context of an
\emph{environment} $\eta \ap \mc{X} \to \bool$. For an
arbitrary environment $\eta$, we write $\eta [X:=b]$ for the environment
$\eta$ in which the proposition variable $X$ has
Boolean value $b$.
%

Finding a solution of a BES amounts to finding an assignment of $\true$ or
$\false$ to each variable $X_i$ such that all equations are satisfied.
Furthermore if $\sigma_i = \mu$, then the assignment to $X_i$ is as strong as
possible, and if $\sigma_i = \nu$ it is as weak as possible, where the
leftmost equation takes priority over equations that follow. The concept
of a solution is formalised below.

\begin{definition}
Let $\eta \ap \mc{X} \to \bool$ be an environment.  The
\emph{interpretation} $\sem{f}{\eta}$ maps a proposition formula $f$
to $\true$ or $\false$:
\begin{align*}
\sem{c}{\eta} &\isdef c & \sem{X}{\eta}      &\isdef \eta(X) \\
\sem{f \vee g}{\eta} &\isdef \sem{f}{\eta} \vee \sem{g}{\eta} &
\sem{f \wedge g}{\eta} &\isdef \sem{f}{\eta} \wedge \sem{g}{\eta} 
\end{align*}

Let $\eta$ be an environment. Let $b_{\mu} = \false$ and $b_{\nu} = \true$.
The \emph{solution} of a BES, given $\eta$, is inductively
defined as follows:
\begin{eqnarray*}
\sem{\epsilon}\eta & = & \eta\\
\sem{(\sigma X = f)\mc{E}}\eta & = & \sem{\mc{E}}\eta[X := f(\sem{\mc{E}}\eta[X := b_{\sigma}])]
\end{eqnarray*}
\end{definition}
We also write $\eta(X)$ to denote the interpretation of $X$ in environment $\eta$.
In the sequel, when we refer to solving a BES we mean computing the solution of
the BES.

% In the rest of this paper we assume that the constants $\true$ and $\false$ do
% not occur in the right hand sides of the equations. We may do this as we can
% replace each occurrence of $\true$\ by a reference to $X_{\true}$ and each
% occurrence of $\false$\ by a reference to $X_{\false}$, where $X_{\true}$ and
% $X_{\false}$ are defined as follows:
% \begin{eqnarray*}
% \nu X_{\true} & = & X_{\true}\\
% \mu X_{\false} & = & X_{\false}
% \end{eqnarray*}

%Assuming that $\false < \true$ we define an ordering on boolean equation systems
%as follows:
%\begin{definition} \cite{Mader1997}
%Given boolean equation systems $\mc{E} \equiv (\sigma_1 X_1 = f_1) \dots
%(\sigma_n X_n = f_n)$ and $\mc{E}' \equiv (\sigma_1 X_1 = g_1) \dots (\sigma_n X_n
%= g_n)$, then $\mc{E} \leq \mc{E}'$ if and only if $f_i \leq g_i$. 
%\end{definition}

We introduce the following terminology.
\begin{definition} Let $\mc{E}$ be an equation system. Then
\begin{itemize}
\item $\mc{E}$ is \emph{closed} whenever $\occ{\mc{E}} \subseteq \bnd{\mc{E}}$;
\item $\mc{E}$ is \emph{solved} whenever $\occ{\mc{E}} = \emptyset$;
\end{itemize}
\end{definition}
For closed BES $\mc{E}$, $\sem{\mc{E}}\eta = \sem{\mc{E}}\eta'$ for arbitrary
environments $\eta$ and $\eta'$, hence we may omit the environment in this case.
Also observe that according to the semantics $\land$ and $\lor$ are commutative and
associative, hence we may write \eg $\bigwedge_{i=0}^j f_i$ instead of $f_0 \land
\dots f_n$, for formulae $f_i$.

For a closed BES $\mc{E}$ we define the right hand side $\mathsf{rhs}$ of a
propositional variable $X \in \bnd{\mc{E}}$ as the \emph{right hand side} of the
defining equation of $X$ in $\mc{E}$:
\begin{equation*}
\rhs{X}{(\sigma Y = f)\mc{E}} \isdef \begin{cases}
                                             f & \text{if $X = Y$}\\
                                             \rhs{X}{\mc{E}} & \text{otherwise}
                                           \end{cases}\\
\end{equation*}

Besides assignments we can also define the notion of substitution on a
BES.
\begin{definition}
A substitution $\sigma {:} \mc{X} \to \mc{F}_{\mc{X}}$ is a function assigning
a boolean expression to a variable. We define application of a substitution
$\sigma$ to a boolean expression inductively as follows:
\begin{align*}
  \subst{c} & \isdef c  &
  \subst{X}  & \isdef \begin{cases} Y & \text{if $X := Y$ in $\sigma$}\\ X  & \text{otherwise}\end{cases} \\
  \subst{\neg f} & \isdef \neg \subst{f} &
  \subst{f \imp g} & \isdef \subst{f} \imp \subst{g} \\
  \subst{f \vee g} & \isdef \subst{f} \vee \subst{g} & 
  \subst{f \wedge g} & \isdef \subst{f} \wedge \subst{g} 
\end{align*}
\end{definition}

We now introduce some restricted BES formats which exhibit some interesting
theoretic properties.
\begin{definition}
A closed BES $\mc{E}$ is in simple form (SF) if every equation in $\mc{E}$ is of the form
$\sigma X = f$, $\sigma X = \bigwedge_{i=0}^n f_{i}$ or
$\sigma X = \bigvee_{i=0}^n f_{i}$, where $n > 0$, and $f$ is either a propositional
variable, or one of the Boolean constants $\true$ or $\false$.
\end{definition}
That is, a closed BES is in simple form if every right hand side is
either a single variable or Boolean constant, or it is a conjunction or a
disjunction over propositional variables or Boolean constants.
Conjunctions and disjunctions may not appear mixed in a
single right hand side. Note that every closed BES can be transformed into
simple form in polynomial time in such a way that the variables in the original BES
are preserved, and variables that occur in both BESs have the same solution, see
\cite{Mader1997}. An equation can, for example, be transformed to simple form as
follows.
Given an equation $\sigma X = \bigwedge_{i=0}^k f_i$, and some $f_j$ is disjunctive,
replace this single equation by two equations $(\sigma X = \bigwedge_{i=0}^{j-1} \land X' \bigwedge_{i=j+1}^k)(\sigma X' = f_j)$,
where $X'$ is fresh. The case for $\lor$ is analogous, and the transformation can
be repeated until a BES in simple form is obtained.

We can also restrict a BES such that it does not contain Boolean constants. This is
referred to as recursive form.
\begin{definition}
A closed BES $\mc{E}$ is in recursive form (RF) if the Boolean constants $\true$ and $\false$ do not occur in $\mc{E}$.
\end{definition}
The transformation of a BES to a BES in RF can also be done in a solution preserving
way, introducing auxiliary equations for Boolean constants $\true$ and $\false$.

When we combine the notions of simple form and recursive form we obtain the concept
of simple recursive form.
\begin{definition}
A closed BES $\mc{E}$ is in simple recursive form (SRF) if $\mc{E}$ is in simple form, and the Boolean constants $\true$ and $\false$ do not occur in $\mc{E}$.
\end{definition}
The translation of a BES to SRF is simply the composition of the translations of a BES
to SF and RF, and hence is also solution preserving.

%\begin{definition}[Standard form]
%A BES $\mc{E}$ is in standard form if every $\phi_i$ is either $X_j$ or $X_k \lor X_k$ or $X_j \land X_k$.
%\end{definition}

\begin{definition}
A BES $\mc{E}$ is in conjunctive form if every equation in $\mc{E}$ is of the form
$\sigma X = \bigwedge_{i=0}^n f_{i}$, with $n \geq 0$, and $f_i$ a propositional
variable or a Boolean constant.
\end{definition}
That is, a BES in conjunctive form only contains conjuncts, single variables or
Boolean constants
as right hand sides. It has been shown \cite{Mader1997}
that given a BES $\mc{E}$ and an environment $\eta$ there is a BES $\mc{E}'$ in
conjunctive form such that $\mc{E}$ and $\mc{E'}$ have the same solutions in $\eta$. 

A similar notion is a BES in disjunctive form, \ie a BES that only contains
disjuncts, single variables or Boolean constants as right hand sides.
\begin{definition}
A BES $\mc{E}$ is in disjunctive form if every equation in $\mc{E}$ is of the form
$\sigma X = \bigvee_{i=0}^n f_{i}$, with $n \geq 0$, and $f_i$ a propositional
variable or Boolean constant.
\end{definition}

A derived notion of a closed equation system $\mc{E}$ is its 
\emph{dependency graph} $\mc{G}_\mc{E}$, which is defined as a structure
$\langle V, \to \rangle$, where:
\begin{itemize}
\item $V = \bnd{\mc{E}}$;
\item $X \to Y$ iff there is some equation $\sigma X = f$ in $\mc{E}$ with
$Y \in \occ{f}$;
\end{itemize}

We introduce the notion of rank of an equation, and some derived notions. These
notions are an indication of the complexity of the BES, as well as a
measure that occurs in the computational complexity of some of the algorithms for
solving BESs.
\begin{definition} Let $\mc{E}$ be an arbitrary equation system. The
\emph{rank} of some $X \in \bnd{\mc{E}}$, denoted
$\ad{X}$, is defined as $\ad{X} = \cad{\nu}{X}{\mc{E}}$, where
$\cad{\nu}{X}{\mc{E}}$ is defined inductively as follows:
%
\begin{align*}
\cad{\sigma}{X}{\epsilon} & = 0 \\
\cad{\sigma}{X}{(\sigma' Y = f) \mc{E}} & =
\left \{ 
  \begin{array}{ll}
  0 & \text{ if $\sigma = \sigma'$ and $X = Y$} \\
  \cad{\sigma}{X}{\mc{E}} & \text{ if $\sigma = \sigma'$ and $X \not= Y$} \\
  1+ \cad{\sigma'}{X}{(\sigma' Y = f) \mc{E}} & \text{ if $\sigma \not= \sigma'$} \\
  \end{array}
\right .
\end{align*} 
Observe that $\ad{X}$ is odd iff $X$ is defined in a least fixpoint equation.
\end{definition}

The \emph{alternation hierarchy} of an equation system can be thought of
as the number of syntactic alternations of fixpoint signs occurring
in the equation system. The alternation hierarchy $\ah{\mc{E}}$ of an
equation system $\mc{E}$ can be defined as the difference between the largest
and the smallest rank occurring in $\mc{E}$, formally
$\ah{\mc{E}} = \max\{ \ad{X} \mid X \in \bnd{\mc{E}} \} - \min{\{ \ad{X} \mid X \in \bnd{\mc{E}} \}}$.

Given an equation $(\sigma X = f)$ in SF, the function $\op{X}$ 
returns whether $f$ is conjunctive ($\wedge$), disjunctive ($\vee$) or
neither ($\bot$);

%
An alternative characterisation of the solution of a particular
proposition variable $X$ in an equation system
$\mc{E}$ in SRF is obtained through the use of the dependency graph $\mc{G}_\mc{E}$.
We first define the notion of a $\nu$-dominated lasso.
%
\begin{definition} Let $\mc{E}$ be a closed equation system, and let $\mc{G}_\mc{E}$
be its dependency graph. A \emph{lasso} through $\mc{G}_\mc{E}$, starting in
a node $X$, is a finite path $\langle X_0, X_1, \ldots, X_n\rangle$, satisfying
$X_0 = X$, $X_n = X_j$ for some $j \le n$, and for each 
$1 < i \le n$, $X_{i-1} \to X_i$. A lasso is said to be \emph{$\nu$-dominated}
if $\min \{ \ad{X_i} ~|~ j \le i \le n \}$ is even; otherwise, it is
\emph{$\mu$-dominated}.
\end{definition}
The following lemma is loosely based on lemmata taken from Kein\"anen 
(see lemmata~40 and~41 in~\cite{Keinanen2006}).
%
\begin{lemma}\label{lem:d_c}
Let $\mc{E}$ be a closed equation system in SRF, and let
$\mc{G}_{\mc{E}}$ be its dependency graph. Let $X \in \bnd{\mc{E}}$. Then:
\begin{enumerate}
\item If $\mc{E}$ is \emph{disjunctive}, then $\sem{\mc{E}}(X) = \true$
iff some lasso starting in $X$ in $\mc{G}_\mc{E}$ is $\nu$-dominated;
\label{lem:disjunctive_cycles} 

\item If $\mc{E}$ is \emph{conjunctive}, then $\sem{\mc{E}}(X) = \false$
iff some lasso starting in $X$ in $\mc{G}_\mc{E}$ is $\mu$-dominated;
\label{lem:conjunctive_cycles} 
\end{enumerate}
\end{lemma}
\begin{proof}
We only consider the first statement; the proof of the
second statement is analogous.  Observe that when the proposition
variable on the cycle of the lasso has an even lowest rank, it is a
greatest fixpoint equation $\nu X'= f$, with $X' \tle Y$ for all other
equations $\sigma Y = g$ that are on the cycle. This follows from the fact
that these have higher ranks.  \emph{Gau\ss\ elimination}~\cite{Mader1997}
allows one to substitute $g$ for $Y$ in the equation for $X'$, yielding
$\nu X' = f[Y {:=} g]$.  Since, ultimately, $X'$ depends on $X'$ again,
this effectively enables one to rewrite $\nu X' = f$ to $\nu X' = f'
\vee X'$.  The solution to $\nu X' = f' \vee X'$ is easily seen to be $X'
= \true$.  Since all equations on the lasso are disjunctive, this solution
ultimately propagates through the entire lasso, leading to $X = \true$.

Conversely, observe that there is an
equation system $\mc{E}'$ consisting entirely of equations of the form
$\sigma X' = X''$ (follows from Corollary 3.37 in \cite{Mader1997}),
with the additional property that $\sem{\mc{E}} =
\sem{\mc{E}'}$. In $\mc{E}'$, the answer to $X$ can only be $\true$ if it
depends at some point on some $\nu X' = X''$, where ultimately, $X''$
again depends on $X'$, leading to a cycle in the dependency graph with
even lowest rank.
\end{proof}

\section{Some requirements for an implementation}

\subsection{Output formats}
For BESs several output formats are available. First of all, we should
support a boolean output format (probably ATerm based, like the file
format of the PBES library). Furthermore, for compatibility with other
tools it is useful to have output in CWI format (a textual format).
Furthermore we know that BES in SRF coincide with parity games, hence
for this class of equation systems it is useful to be able to output
in the format used by the PGSolver toolset.

In the future it could become useful to write part of an already generated
BES to disk (in order to save internal memory). Hence a design for a BES
library should take this possibility into account.

\subsection{Graph interface}
As was described before, the notion of dependency graphs for BES
in SRF has some nice properties. As an example it is straightforward
to implement reduction modulo strong bisimulation (and also some weaker
equivalences) on top of this interface (see \cite{KeirenW2009} for more
details). A more flexible notion of structure graph is also known
\cite{ReniersW2009}, but at the moment it is not clear what the
advantages of this notion are in an implementation.

\subsection{Flexible storage}
It is unclear how formulae in a BES should be stored exactly in an
implementation. Sometimes \eg storing in BDD format provides some
advantages, whereas in other cases (like BES encoding bisimilarity of
to LPSs), the effect is adverse. Hence the design should be sufficiently
high-level as to leave a possibility for variation in the underlying format.

\section{TODO}
\begin{itemize}
\item Investigate requirements from bes\_deprecated.h
\end{itemize}


\end{document}

