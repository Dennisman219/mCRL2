\documentclass{article}
\usepackage{amsmath,amssymb}
\usepackage{amsfonts}
\usepackage{array}


\newcommand{\ap}{{:}}
\newcommand{\Bool}{{\mathbb B}}
\newcommand{\Real}{\mathbb{R}}
\newcommand{\Nat}{\mathbb{N}}
\newcommand{\Pos}{\mathbb{N^{+}}}
\newcommand{\Int}{\mathbb{Z}}

% true
\newcommand{\true}{\ensuremath{\mathit{true}}}

% false
\newcommand{\false}{\ensuremath{\mathit{false}}}




% column types that change column types l,c,r from math mode to LR
% and the other way round
\newcolumntype{L}{>{$}l<{$}}%stopzone%stopzone%stopzone
\newcolumntype{C}{>{$}c<{$}}%stopzone%stopzone%stopzone
\newcolumntype{R}{>{$}r<{$}}%stopzone%stopzone%stopzone

\newenvironment{mcrl2}%
{\begin{trivlist}
\item\begin{tabular}{@{}>{\bf}p{2.3em}L@{\ }L@{\ }L@{\ }L@{\ }L@{\ }L@{\ }L@{\ }L}}%
{\end{tabular}\end{trivlist}}


\author{Jan Friso Groote}
\title{On free variables in process specifications, LPSs and PBESs}
\date{December 18, 2008}
\begin{document}
\maketitle
\noindent
In process specifications and in parameterised boolean equation systems (PBESs), it is
possible to use free variables. An example is the following:
\begin{mcrl2}
act  & a:\Nat;\\
glob & x:\Nat;\\
proc & P=a(x).P;\\
init & P;
\end{mcrl2}
This represents a whole class of processes, namely for every value of
$x$ this process has a different value. The keyword ${\bf glob}$ stands
for {\it global variable}.
In each specification the keyword {\bf glob} can
be used once any arbitrary of times. The scope of the variables
mentioned in a {\bf glob} declaration is the global file. All {\bf glob}
declarations are grouped together. The names of the variables cannot coincide with other
declared functions, processes, actions, variables (both in equations and
in sum operators) and process and PBES parameters.
Global variables can occur in
process equations, in parameterised fixed point formulas and in {\bf init}
sections.

Such global variables can be used in the common mathematical way. Consider
for instance the polynomial:
\[ax^2+bx+c=0.\]
There are four variables in this equation, namely $a$, $b$, $c$ and $x$.
The use of the variables $a$, $b$ and $c$ allow to study this polynomial in
a far more general setting than when these variables would have concrete values.

In some cases, the concrete values for global variables do not have influence
on the process. In such a case instantiating the global variables to various concrete
values will mean that the process has the same behaviour modulo strong bisimulation,
or the same solution as a parameterised boolean equation system. In this case
we call the process or PBES {\it global variable insensitive}.
An example is the following linearisation of the buffer
\begin{mcrl2}
sort & D;\\
glob & {\it dummy}_1,{\it dummy}_2:D;\\
proc& P(b:\Bool,d\ap D)=\sum_{e\ap D} b\rightarrow {\it read}(e){\it \cdot}P(\false,e)+
\neg b\rightarrow{\it send}(d){\cdot}P(\true,{\it dummy}_1);\\
init& P(\true,{\it dummy}_2);
\end{mcrl2}
The idea is that if the parameter $b$ is $\true$, the value of the second parameter
is not relevant anymore. Therefore, it can be set to any arbitrary value, which is
indicated by the use of ${\it dummy}_1$ and ${\it dummy}_2$. As this specification is
global variable insensitive, concrete values can be chosen for these global variables
when this would be fruitful.

The tool {\tt mcrl22lps} generates linear processes with global variables. It guarantees
that the resulting specification is global variable insensitive. Certain transformation
tools, like {\tt lpsconstelm} and {\tt lps2pbes} yield global variable insensitive output,
provided the input in global variable insensitive. In case systems are not global variable
insensitive the output of these tools can be garbage. It is the responsibility of those who
apply the tools that the tools are used in a proper way. It is likely that most tools
leave global variables untouched, unless a switch indicates that global variable
insensitivity can be used.
\end{document}
