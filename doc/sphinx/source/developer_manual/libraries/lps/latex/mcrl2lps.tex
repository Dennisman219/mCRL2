%%%%%%%%%%%%%%%%%%%%%%%%%%%%%%%%%%%%%%%%%%%%%%%%%%%%%%%%%%%%%%%%%%
%%%  MTLPSs                                                    %%%
%%%%%%%%%%%%%%%%%%%%%%%%%%%%%%%%%%%%%%%%%%%%%%%%%%%%%%%%%%%%%%%%%%
\documentclass[fleqn,a4paper,dvips]{article}
\usepackage{textcomp,amsmath,amssymb,amsthm,stmaryrd}
\usepackage{geometry}
%\usepackage[ps2pdf]{hyperref} % remove for printing
%\usepackage[active]{srcltx}
\usepackage{mymath,mythm,mcrl}

\setlength{\mathindent}{0.5em}

\def\YSU{\texttt{ YSU: TODO :USY\marginpar{!!!} }}
\def\JFG{\texttt{ JFG: TODO :GFJ\marginpar{!!!} }}

\newcommand{\aterm}[1]{\mathsf{#1}}
\newcommand{\afunc}[3]{\aterm{#1}:#2\rightarrow#3}

\newcommand{\Sig}{\nm{Sig}}
\newcommand{\Fun}{\nm{Fun}}
\newcommand{\nul}{{\bf 0}}
\newcommand{\wor}{\stackrel{\mathrm{def}}{=}}
\newcommand{\NV}{{\cal V}}
\newcommand{\type}{\mathbb{T}}
\newcommand{\PNI}{\nm{PNI}}
\newcommand{\PN}{\nm{PN}}
\newcommand{\SB}{\nm{SB}}
\newcommand{\LSB}{\nm{LSB}}
\newcommand{\List}{\nm{List}}
\newcommand{\Set}{\nm{Set}}
\newcommand{\Bag}{\nm{Bag}}
\newcommand{\TypeAny}{\nm{TypeAny}}
\newcommand{\typecompat}{\equiv_t}
\newcommand{\wt}{\nm{wt}}
\newcommand{\Int}{\nm{Int}}
\newcommand{\Pos}{\nm{Pos}}
\newcommand{\Type}{\nm{Type}}
\newcommand{\String}{\nm{String}}

\newcommand{\Vars}{\mbox{\em Vars}}

\title{LPS definitions and an ATerm representation format for \mcrl\
  with multiactions and time}

\author{Jan Friso Groote \and Yaroslav S. Usenko}

\begin{document}
\maketitle
\tableofcontents

\section{LPS definitions}

The equation below represents a Linear Process Equation for \mcrl\
with multiactions and time (MTLPS).
\begin{gather}
\begin{split}
\pr{X}(\vect{d\ap D})=&\sum_{i\in I}\sum_{\vect{e_i\ap E_i}} c_i(\vect{d,e_i})\to\act{a}^0_i(\vect{f_{i,0}}(\vect{d,e_i}))\comm\dots\comm\act{a}^{n(i)}_i(\vect{f_{i,n(i)}}(\vect{d,e_i}))\at t_i(\vect{d,e_i})\seqc
\pr{X}_i(\vect{g_i}(\vect{d,e_i}))\notag\\
&+\sum_{j\in J}\sum_{\vect{e_j\ap E_j}} c_j(\vect{d,e_j})\to\act{a}^0_j(\vect{f_{j,0}}(\vect{d,e_j}))\comm\dots\comm\act{a}^{n(j)}_j(\vect{f_{j,n(j)}}(\vect{d,e_j}))\at t_j(\vect{d,e_j})\\
&+\sum_{\vect{e_\delta\ap E_\delta}} c_\delta(\vect{d,e_\delta})\to\delta\at t_\delta(\vect{d,e_\delta})
\end{split}
\end{gather}
where $I$ and $J$ are disjoint.

It is possible to translate multiactions to regular \mcrl\ actions
(with longer parameter lists).  In this way a MTLPS can be translated
to a TLPS, preserving equivalence.  The TLPS that corresponds to the
above MTLPS is defined in the following way.
\begin{gather}
\begin{split}
\pr{X}(\vect{d\ap D})=&\sum_{i\in I}\sum_{\vect{e_i\ap E_i}} c_i(\vect{d,e_i})\to\act{a}^0_i\_\act{a}^1_i\_\dots\_\act{a}^{n(i)}_i(\vect{f_{i,0}}(\vect{d,e_i}),\dots,\vect{f_{i,n(i)}}(\vect{d,e_i}))\at t_i(\vect{d,e_i})\seqc
\pr{X}_i(\vect{g_i}(\vect{d,e_i}))\notag\\
&+\sum_{j\in J}\sum_{\vect{e_j\ap E_j}} c_j(\vect{d,e_j})\to\act{a}^0_j\_\act{a}^1_j\_\dots\_\act{a}^{n(j)}_j(\vect{f_{j,0}}(\vect{d,e_j}),\dots,\vect{f_{j,n(j)}}(\vect{d,e_j}))\at t_j(\vect{d,e_j})\\
&+\sum_{\vect{e_\delta\ap E_\delta}} c_\delta(\vect{d,e_\delta})\to\delta\at t_\delta(\vect{d,e_\delta})
\end{split}
\end{gather}
where $I$ and $J$ are disjoint, and
$\act{a}^0_i\_\act{a}^1_i\_\dots\_\act{a}^{n(i)}_i$ and
$\act{a}^0_j\_\act{a}^1_j\_\dots\_\act{a}^{n(j)}_j$ are new actions
(for each $i$ and $j$), parameterized by the concatenation of the
parameter lists of the contained actions.

\YSU formalize below

\begin{thm}
  Given
  $\mathrm{MTLPS1}=(\mathrm{mCRL2})=\mathrm{MTLPS2}$,\\
  $\mathrm{TLPS(MTLPS1)}=(\mathrm{timed\ mcrl})=\mathrm{TLPS(MTLPS2)}$.
\end{thm}

Time can be eliminated from TLPSs in a similar way (see page 106 of
the thesis).

\appendix
\newpage
\section{Aterm format for mCRL2 after parsing}

\section{Static Semantics and Well-formedness}
\label{section:SSC}
In this section it is defined when a specification is correctly defined.
We use the syntactical categories from the previous section (in teletype
font) to refer to items in a specification. If we denote a
concrete part of a specification, we prefer using the latex symbols, to
increase readability. The definitions below are an adapted copy from those in
\cite{GroPo94a}.

In
essence the static semantics says that functions and terms are well
typed, and some sorts and functions are present in the specification. The
validity of all static semantic requirements can efficiently be decided for any
specification.

A specification is well formed, if it satisfies the static semantic requirements,
there are no empty sorts and the sort $\Time$ is appropriately defined. We
only give an operational semantics to well-formed specifications.

\subsection{Static semantics}
A {\tt Specification} must be internally consistent. This means
that all objects that are used must be declared exactly once and
are used such that the sorts are correct. It also means that action,
process, constant and variable names cannot be confused.
Furthermore, it means that communications are specified in a functional
way and that it is guaranteed that the terms used in an equation are well-typed.
Because all these properties can be statically decided, a
specification that is internally consistent is called SSC
({\em Statically Semantically Correct}). All next definitions
culminate in Definition \ref{SSC-mCRL}.

\subsubsection{SSC of \texttt{Specification}}
We assume that the specification has the form
$\aterm{spec}(\nm{sortspec?},\nm{opspec?},\nm{eqnspec?},\nm{actspec?},\nm{proscpec?},\nm{init})$
(an easy transformation of the input aterm brings it to this form).
All of the parameters are optional except the last one (the minimal specification is
$\aterm{spec}(\aterm{init}(\aterm{tau}())))$.
Sometimes part of the specification is not used. For example, any sort specification is useless
unless some functions are defined for them. And also functions specifications are useless if they do not
occur in expressions. Such specifications are still considered SSC, although an implementation of the checker
may issue a warning in such cases.

Let $Sig$ be a signature and $\NV$ be a set of variables over $Sig$. We
define the predicate `is SSC wrt.\ $Sig$'
inductively over the syntax of a {\tt Specification}.
\paragraph{Sorts}
Sort declarations:
\begin{itemize}
\item
  A {\tt SortSpec}
  $\aterm{SortSpec}([\nm{sspec}_1,\dots,\nm{sspec}_m])$ with $m\geq 1$
  is SSC wrt.\ $Sig$ iff
  \begin{itemize}
  \item all $\nm{sspec}_1,\dots,\nm{sspec}_m$ are SSC wrt.\ $Sig$.
  \item Defined sort names are different: for all $i<j$, $\nm{defined\_sorts}(\nm{sspec}_i)\neq\nm{defined\_sorts}(\nm{sspec}_j)$.
  \end{itemize}
\item
  A {\tt SortDecl}
  $\aterm{SortDeclStandard}([n_1\cdots n_m])$ with $m\geq 1$
  is SSC wrt.\ $Sig$ iff all $n_1,\ldots,n_m$ are pairwise
  different.
\item
  A {\tt SortDecl}
  $\aterm{SortDeclRef}(n,s)$ with $m\geq 1$
  is SSC wrt.\ $Sig$ iff the \emph{sort expression} $s$ is SSC w.r.t $Sig\setminus [n_1\cdots n_m]$.
  Here we note that no recursive sort references are allowed.
\item
  A {\tt SortDecl}
  $\aterm{SortDeclStruct}(n,[\nm{cons}_1\dots,\nm{cons}_m])$ with $m\geq 1$
  is SSC wrt.\ $Sig$ iff all \emph{constructor expressions} $\nm{cons}$ are SSC w.r.t $Sig$.
\item
  A {\tt ConstDecl}
  $\aterm{StructDeclCons}(n,[\nm{proj}_1\dots,\nm{proj}_m]),k)$ with $m\geq 0$
  is SSC wrt.\ $Sig$ iff
  \begin{itemize}
  \item both $n$ and $k$ are not declared as function or map (or $k==nil()$)
  \item all \emph{projection expressions} $\nm{proj}$ are SSC w.r.t $Sig$.
  \end{itemize}
\item
  A {\tt ProjDecl}
  $\aterm{StructDeclProj}(n,\aterm{Dom}([s_1\cdots s_m]))$ with $m\geq 1$
  is SSC wrt.\ $Sig$ iff
  \begin{itemize}
  \item both $n$ is not declared as function or map (or $n==nil()$)
  \item \emph{sort expressions} $s$ are SSC w.r.t $Sig$
  \end{itemize}
\end{itemize}

\paragraph{Data types}
\begin{itemize}
\item
  A {\tt OpSpec}\\
  $\aterm{ConsSpec}([IdDecls([n_{11},\dots,n_{1l_1}],s_1),\dots,IdDecls([n_{m1},\dots,n_{ml_m}],s_m)])$
  or\\
  $\aterm{MapSpec}([IdDecls([n_{11},\dots,n_{1l_1}],s_1),\dots,IdDecls([n_{m1},\dots,n_{ml_m}],s_m)])$
  with $m\geq 1$, $l_i\geq 1$, $k_i\geq 0$ for $1\leq i\leq m$
  is SSC wrt.\ $Sig$ iff
  \begin{itemize}
  \item
    for all $1\leq i\leq m$  $n_{i1},\ldots,n_{il_i}$ are
    pairwise different,
  \item
    for all $1\leq i\leq m$ it holds that $s_i$ is SSC wrt.\ $Sig$.
  \item
    for all $1\leq i<j\leq m$ it holds that
    if $n_{ik}\equiv n_{jk'}$ for some $1\leq k\leq l_i$ and $1\leq
    k'\leq l_j$, then $\nm{type}_{Sig}(s_i)\neq\nm{type}_{Sig}(s_j)$,
  \end{itemize}
\item
  A {\tt EqnSpec} of the form:\\
\begin{gather}
  \aterm{EqnSpec}([IdDecls([n_{11},\ldots,n_{1l_1}],s_1),\dots,IdDecls([n_{m1},\ldots,n_{ml_m}],s_m)],\\
  [\aterm{EqnSpec}(d_1,d_1')\dots\aterm{EqnSpec}(d_k,d_k')])
\end{gather}
  with $m\geq 1$, $l_i\geq 1$, $k_i\geq 0$ for $1\leq i\leq m$
  is SSC wrt.\ $Sig$ iff
  \begin{itemize}
  \item
    for all $1\leq i,j\leq m$  $n_{ij}$ are pairwise different,
  \item
    for all $1\leq i\leq m$ it holds that $s_i$ is SSC wrt.\ $Sig$.
  \item
    for all $1\leq j\leq k$ it holds that $d_i$ and $d_i'$ is SSC wrt.\ $Sig+ns$.
  \item
    for all $1\leq j\leq k$ it holds that the types of $d_i$ and $d_i'$ are uniquly compatible (wrt. $Sig$).
  \end{itemize}
\end{itemize}

\paragraph{Actions}
  A {\tt ActSpec} of the form:\\
$\aterm{ActSpec}([\aterm{ActDecl}([n_{11},\dots,n_{1l_1}],d_1)\dots\aterm{ActDecl}([n_{m1},\dots,n_{ml_m}],d_m)])$
with $m\geq 1$ is SSC wrt. $Sig$ iff
\begin{itemize}
\item for all $1\leq i\leq m$ all $n_{ij}$ are pairwise different.
\item none of them are in $\Sig.\Fun\cup\Sig.\Map$
\item $d_i$ is $\nm{Nil}()$ or $d_i$ is $\nm{Dom}([s_1,\dots,s_n])$, and all $s_j$ are SSC wrt.\ $\Sig$.
\end{itemize}

\paragraph{Processes}
\begin{itemize}
\item
  A {\tt ProcSpec}
  $\aterm{ProcSpec}([\aterm{ProcDecl}(n_1,\nm{vars}_1,p_1),\dots,\aterm{ProcDecl}(n_m,\nm{vars}_m,p_m)])$
  with $m\geq 1$ is SSC wrt.\ $Sig$ iff
  \begin{itemize}
  \item
    for each $1\leq i<j\leq m$:
    if $\nm{type}(\nm{vars}_i)=\nm{type}(\nm{vars}_j)$, then $n_i\neq n_j$,
  \item none of them are in $\Sig.\Fun\cup\Sig.\Map\cup\Sig.\Act$
  \item
    for each {\tt Name} $S'$ it holds that $n\ap S_{1}\times\cdots\times
    S_{k}\rightarrow S'\notin \Sig.\Fun\cup\Sig.\Map$,
  \item the {\tt Name}s $x_{1},\ldots,x_{k}$ are pairwise different
    and $\{\langle x_{j}\ap S_{j}\rangle\mid 1\leq j\leq k\}$ is a set
    of variables over $Sig$,
  \item $p$ is SSC wrt.\ $Sig$ and $\{\langle
    x_{j}\ap S_{j}\rangle\mid 1\leq j\leq k\}$.
  \end{itemize}
\item
  A {\tt Init} of the form $\aterm{Init}(p)$ is SSC wrt.\ $\Sig$
  iff $p$ SSC wrt.\ to $\Sig$ and $\emptyset$.
\end{itemize}

\begin{defn}\label{def:SSC-mCRL}
  Let $E$ be a {\tt Specification}. We say that $E$ is SSC iff $E$ is SSC
  wrt.\ $\Sig(E)$.
\end{defn}


\subsubsection{Process and Data Terms. (Sub-)Typing}

\paragraph{Process terms}
Let $\Sig$ be a signature and $\NV$ be a set of variables over $\Sig$.
We say that a {\tt Process-term} $p$ is {\it SSC wrt.\ to $\Sig$ and
$\NV$} iff one of the following hold:
\begin{itemize}
\item
  $p\equiv p_1+p_2$,
  $p\equiv p_1\parallel p_2$, $p\equiv p_1\leftm p_2$, $p\equiv p_1\mid
  p_2$,
  $p\equiv p_1 {\cdot} p_2$ or $p\equiv p_1\tb p_2$ and both $p_1$ and $p_2$ are SSC wrt.\ $Sig$ and $\NV$,
\item
  $p\equiv p_1\triangleleft t \triangleright p_2$ and
  \begin{itemize}
  \item
    $p_1$ and $p_2$ are SSC wrt.\ $Sig$ and $\NV$,
  \item
    $t$ is SSC wrt.\ $Sig$ and $\NV$ and $sort_{Sig,\NV}(t)=\seq{\Bool}$.
  \end{itemize}
\item
  $p\equiv p_1\at{t}$ and
  \begin{itemize}
\item
  $p_1$ is SSC wrt.\ $Sig$ and $\NV$
\item
  $t$ is SSC wrt.\ $Sig$ and $\NV$ and $sort_{Sig,\NV}(t)=\Time$.
\end{itemize}
\item
  $p\equiv \delta$ or $p\equiv\tau$.
\item
  $p\equiv\partial_{\{n_1,\ldots,n_m\}}p_1$ or
  $p\equiv \tau_{\{n_1,\ldots,n_m\}}p_1$ with $m\geq 1$ and
  \begin{itemize}
\item
  for all $1\leq i\neq j \leq m$ $n_i\not\sqsubseteq n_j$,
\item
  for $1\leq i\leq m$, if $n_i=n_{i,1}\comm\dots\comm n_{i,k}$, then $n_{i,j}\in \Sig.\nm{ActNames}$.
\item
  $p_1$ is SSC wrt.\ $Sig$ and $\NV$.
\end{itemize}
\item
  $p\equiv\nabla_{\{n_1,\ldots,n_m\}}p_1$ with $m\geq 1$ and
  \begin{itemize}
  \item
    for all $1\leq i< j \leq m$ $n_i\not\equiv n_j$,
  \item
    for $1\leq i\leq m$, if $n_i=n_{i,1}\comm\dots\comm n_{i,k}$, then $n_{i,j}\in \Sig.\nm{ActNames}$.
  \item
    $p_1$ is SSC wrt.\ $Sig$ and $\NV$.
\end{itemize}
\item
  $p\equiv\rho_{\{n_1\rightarrow n_1',\ldots,n_m\rightarrow
    n_m'\}}p_1$ and
  \begin{itemize}
  \item
    for $1\leq i\leq m$ both $n_i,n_i'\in \Sig.\nm{ActNames}$.
  \item
    for each $1\leq i<j\leq m$ it holds that $n_i\not\equiv n_j$,
  \item
    for $1\leq i\leq m$, the types of $n_i$ and $n_i'$ are the same in $Sig$.
  \item
    $p_1$ is SSC wrt.\ $Sig$ and $\NV$.
  \end{itemize}
  $p\equiv\Gamma_{\{n_1\rightarrow n_1',\ldots,n_m\rightarrow
    n_m'\}}p_1$ and
  \begin{itemize}
  \item
    for $1\leq i\leq m$, if $n_i=n_{i,1}\comm\dots\comm n_{i,k}$, then $n_{i,j}\in \Sig.\nm{ActNames}$.
  \item
    for $1\leq i\leq m$ either $n'_i\in \Sig.\nm{ActNames}$ or $n'_i=\tau$.
  \item
    for each $1\leq i\neq j\leq m$ it holds that $n_i\not\sqsubseteq n_j$,
  \item
    for $1\leq i\leq m$ it holds that, if $n_i=n_{i,1}\comm\dots\comm n_{i,k}$, then the types of all
    $n_{i,j}$ and $n'_i$ are the same in $Sig$.
  \item
    $p_1$ is SSC wrt.\ $Sig$ and $\NV$.
  \end{itemize}
\item
  $p\equiv \Sigma_{x\ap S}p_1$ and
  iff
  \begin{itemize}
\item
  $(\NV\backslash\{\langle x\ap S'\rangle \mid S'
  \mbox{ is a {\tt Name}}\})\cup\{\langle x\ap S\rangle\}$
  is a set of variables over $Sig$,
\item
  $p_1$ is SSC wrt.\ $Sig$ and
  $(\NV\backslash\{\langle x\ap S'\rangle \mid S'\mbox{ is a
    {\tt Name}}\})\cup\{\langle x\ap S\rangle\}$.
\end{itemize}
\item
  $p\equiv n$ and
  $n=p'\in \Sig.\Proc$ for some {\tt Process-term} $p'$, or $n\in
  \Sig.\Act$.
\item
  $p\equiv n(t_1,\ldots,t_m)$
  with $m\geq 1$ and
  \begin{itemize}
  \item
    $n(x_1\ap sort_{Sig,\NV}(t_1),\ldots,x_m\ap sort_{Sig,\NV}(t_m))=p'\in
    \Sig.\Proc$ for {\tt Name}s $x_1,\ldots,x_m$ and\\ {\tt Process-term} $p'$,
    or
    $n\ap sort_{Sig,\NV}(t_1)\times \cdots\times sort_{Sig,\NV}(t_m)\in
    \Sig.\Act$,
  \item
    for $1\leq i\leq m$ the {\tt Data-term}
    $t_i$ is SSC wrt.\ $Sig$ and $\NV$.
  \end{itemize}
\end{itemize}

\paragraph{Sort expressions}
\begin{itemize}
\item
  A {\tt SortExpr}
  $\aterm{SortBool}(),\aterm{SortPos}(),\aterm{SortNat}(),\aterm{SortInt}()$ are SSC.
\item
  A {\tt SortExpr}
  $\aterm{SortList}(s),\aterm{SortSet}(s),\aterm{SortBag}(s)$ are SSC wrt.\ $Sig$ iff
  sort expression $s$ is SSC wrt.\ $Sig$.
\item
  A {\tt SortExpr}
  $\aterm{SortRef}(n)$ is SSC wrt.\ $Sig$ iff $n\in\nm{Sorts}(Sig)$.
\item
  A {\tt SortExpr}
  $\aterm{SortArrow}(\aterm{Dom}([n_1,\dots,n_m]),n)$ with $m\geq 1$
  is SSC wrt.\ $Sig$ iff all \emph{sort expressions} $n$ are SSC w.r.t $Sig$.
\end{itemize}
Any sort expression that is SCC is also well-typed. I.e. it is impossible to specify
an incorrectly typed sort.


\section{Context Information}
The context consists of two parts. The static part corresponds to the
global information in the specification. The dynamic part contains the
definitions of the variables, and can change depending on the scope.
Given a context of a specification $\kappa$, we denote the static context as
$\nm{Sig(\kappa)}$ and the dynamic part as $\nm{Vars}(\kappa)$.
The static context is a tupple
\[(\nm{BasicSorts},\nm{DefinedSorts},\nm{Operations},\nm{Actions},\nm{Processes})\]
which represents the names and types of the sorts, operations, actions and processes
defined in the specification.
The types of the context operatinos are defined below:
\begin{align*}
\nm{BasicSorts}=\set{\String}\\
\func{\nm{DefinedSorts}}{\String}{\Type}\\
\nm{Operations}\in\String\x\Type\\
\nm{Actions}\in\String\x\Type\\
\func{\nm{Processes}}{\String}{\Type}
\end{align*}
The sort $\Type$ is a sort expression containing defined sorts, a list
of such expressions, or the empty type (unit type).  It can be also
unknown.  The function $\func{\nm{basicType}}{\Type}{\Type}$ unfolds
all occurrences of derived sort names in a type expression.

The variables are defined as a function from Variable name to a variable type
$\func{\nm{Variables}}{\String}{\Type}$.

\paragraph{Data Terms}
Let $\Sig$ be a signature, and let $\NV$ be a set of variables over
$\Sig$.
A {\tt Data-term} $t$ is called SSC wrt.\ $\Sig$ and $\NV$ iff one of
the following holds
\begin{itemize}
\item
  $t\equiv n$ with $n$ a {\tt Name} and
  $\langle n\ap S\rangle\in \NV$ for some $S$, or
  $n\ap  \rightarrow sort_{Sig,\NV}(n)\in \Sig.\Fun\cup\Sig.\Map$.
\item
  $t\equiv n(t_1,\ldots,t_m)$ ($m\geq 1$) and
  $n\ap sort_{Sig,\NV}(t_1)\times\cdots\times sort_{Sig,\NV}(t_m)
  \rightarrow sort_{Sig,\NV}(n(t_1,\ldots,t_m))\in \Sig.\Fun\cup\Sig.\Map$ and
  all
  $t_i$ ($1\leq i\leq m$) are SSC wrt.\ $Sig$ and $\NV$.
\end{itemize}

The typing rules of the built-in data types can be defined as follows:
As for the sort and process expressions we introduce the following functions for data
expressions: $\nm{is\_well\_named}$ -- all ids are defined, $\nm{id\_vars}$ to identify
the variables, $\nm{types}$ -- all possible types the term can be,
$\nm{is\_well\_typed}$ -- is the term well-typed?

The function $\type_\kappa$ is defined defined as (well-namedness of the arguments is assumed):
\begin{displaymath}
\begin{array}{lll}
\aterm{DataVar}(\nm{String})         &                                                            & \nm{type}(\kappa,\nm{String})\\
\aterm{OpId}(\nm{String})            &                                                            & \nm{type}(\kappa,\nm{String})\\
\aterm{Number}(\nm{NumberString})    &                                                            & \PNI(\nm{NumberString})\\
\aterm{ListEnum}(d_0,\dots, d_n)     & \forall i\in\overline{0,n}~\type(d_i)\typecompat\type(d_0) & \List(\nm{minC}(\type(d_0),\dots,\type(d_n)))\\
\aterm{SetEnum}(d_0,\dots, d_n)      & \forall i\in\overline{0,n}~\type(d_i)\typecompat\type(d_0) & \Set(\nm{minC}(\type(d_0),\dots,\type(d_n)))\\
\aterm{BagEnum}(\aterm{BagEnumElt}(d_0,d_0'),\\
\t2\dots,\aterm{BagEnumElt}(d_n,d_n'))
                                     & \forall i\in\overline{0,n}~\left(\type(d_i)\typecompat\type(d_0)\land\type(d_i')\typecompat\PN\right)
                                                                                                  & \Bag(\nm{minC}(\type(d_0),\dots,\type(d_n)))\\
\aterm{SetBagComp}(\aterm{IdDecl}(v,s),d)
                                     & \wt(\kappa+(v,s),d)\land                                   & \Set(s)~\text{if}~\type_{\kappa'}(d)=\Bool \\
                                     &\t1(\type_{\kappa'}(d)=\Bool\lor\type_{\kappa'}(d)\typecompat\PN)               & \Bag(s)~\text{if}~\type_{\kappa'}(d)\typecompat\PN\\
\aterm{DataApp}(d, d_0,\dots,d_n)    & \type(d)=A_0 \dots A_n\to B\land\\
                                     & \t1\type(d_0)\typecompat A_0\land\dots\land\type(d_n)\typecompat A_n & B\\
\aterm{Forall}([\aterm{IdsDecl}(\vect{v_0},s_0),\
                                     &  \wt(\kappa+(\vect{v_0},s_0,\dots,\vect{v_n},s_n),d)       & \Bool\\
\t2\dots,\aterm{IdsDecl}(\vect{v_n},s_n)], d)
                                     & \t1\land\type_{\kappa'}(d)=\Bool                                     &\\
\aterm{Exists}([\aterm{IdsDecl}(\vect{v_0},s_0),
                                     & \wt(\kappa+(\vect{v_0},s_0,\dots,\vect{v_n},s_n),d)        & \Bool\\
\t2\dots,\aterm{IdsDecl}(\vect{v_n},s_n)], d)
                                     & \t1\land\type_{\kappa'}(d)=\Bool                                     &\\
                   & \Bool\\
\aterm{Lambda}([\aterm{IdsDecl}(\vect{v_0},s_0),\\
\t2\dots,\aterm{IdsDecl}(\vect{v_n},s_n)], d)
                                    & \wt(\kappa+(\vect{v_0},s_0,\dots,\vect{v_n},s_n),d)         & s_0^{\nm{len}(\vect{v_0})},\dots,s_n^{\nm{len}(\vect{v_n})}\to\type_{\kappa'}(d)\\
\aterm{Whr}(d,[v_0,d_0,\dots,v_n,d_n])& \wt(\kappa+(v_0,\type(d_0),\dots,v_n,\type(d_n)),d)       & \type_{\kappa'}(d)
\end{array}
\end{displaymath}

The following internal, or system, identifiers have the corresponding (polymorphic) types:
\begin{displaymath}
\begin{array}{lll}
\aterm{EmptyList}()                  &                                                            & \List(\TypeAny)\\
\aterm{EmptySetBag}()                &                                                            & \SB(\TypeAny)\\
\aterm{NotOrCompl}(d)                & \type(d)=\Bool\lor\type(d)\typecompat\SB(\TypeAny)         & \type(d)\\
\aterm{Neg}(d)                       & \type(d)\typecompat\PNI                                    & \Int\\
\aterm{Size}(d)                      & \type(d)\typecompat\LSB(\TypeAny)                          & \Nat\\
\aterm{ListAt}(d, d')                & \type(d)\typecompat\List(\TypeAny)\land\type(d')\typecompat\PN & \nm{ListArg}(d)\\
\aterm{Div}(d, d')                   & \type(d)\typecompat\PNI\typecompat\type(d')                & \nm{div}(\PNI)\\
\aterm{Mod}(d, d')                   & \type(d)\typecompat\PNI\land\type(d')\typecompat\Pos       & \nm{mod}(\PNI)\\
\aterm{MultOrIntersect}(d, d')       & (\type(d)\typecompat\PNI\lor\type(d)\typecompat\SB(\TypeAny))
                                                                                                  & \nm{maxMoI}(\type(d),\type(d'))\\
\aterm{AddOrUnion}(d, d')            & \t2\land\type(d)\typecompat\type(d')                          &\\
\aterm{SubtOrDiff}(d, d')            &                                                            &\\
\aterm{LTOrPropSubset}(d, d')        &                                                            & \Bool\\
\aterm{GTOrPropSupset}(d, d')        &                                                            & \Bool\\
\aterm{LTEOrSubset}(d, d')           &                                                            & \Bool\\
\aterm{GTEOrSupSet}(d, d')           &                                                            & \Bool\\
\aterm{In}(d, d')                    & \LSB(\type(d))\typecompat\type(d')                         & \Bool\\
\aterm{Cons}(d, d')                  & \List(\type(d))\typecompat\type(d')                        & \nm{max}(\List(\type(d)),\type(d'))\\
\aterm{Snoc}(d, d')                  & \List(\type(d'))\typecompat\type(d)                        & \nm{max}(\List(\type(d')),\type(d))\\
\aterm{Concat}(d, d')                & \type(d)\typecompat\List(\TypeAny)\typecompat\type(d')     & \List(\nm{max}(\type(d),\type(d')))\\
\aterm{EqNeq}(d, d')                 & \type(d)\typecompat\type(d')                               & \Bool
\end{array}
\end{displaymath}

\begin{displaymath}
\begin{array}{lll}
\aterm{True}()                       &                                                            & \Bool\\
\aterm{False}()                      &                                                            & \Bool\\
\aterm{Imp}(d, d')                   & \type(d)\typecompat\type(d')=\Bool                         &\\
\aterm{And}(d, d')                   & \type(d)\typecompat\type(d')=\Bool                         &
\end{array}
\end{displaymath}

\subsection{The signature of a specification}
\begin{defn}
  The signature $\Sig(E)$ of a {\tt Specification} $E$ consists of a
  seven-tuple \[(\Sort,\Fun,\Map,\Act,\Comm,\Proc,\Init)\] where
  each component is a set containing all elements of a main syntactical category
  declared in $E$.
  The signature $\Sig(E)$ of $E$ is inductively defined as follows:
  \newcommand{\fd}{{\it fd}}
  \newcommand{\md}{{\it md}}
  \newcommand{\pe}{{\it pe}}
  \newcommand{\ad}{{\it ad}}
  \begin{itemize}
  \item
    If $E\equiv \sortkw~n_1\cdots n_m$ with $m\geq 1$,
    then
    $\Sig(E)\wor(\{n_1,\ldots,n_m\},\emptyset,\emptyset,\emptyset,
    \emptyset,\emptyset,\emptyset).$
\item
If $E\equiv
\func~\fd_1\cdots \fd_m$ with $m\geq 1$,
then
$\Sig(E)\wor(\emptyset,\Fun,
\emptyset,\emptyset,
\emptyset,\emptyset,\emptyset),$
where \[\begin{array}{lll}
\Fun &\wor&
\{n_{ij}\ap \rightarrow S_i\mid \fd_i\equiv
n_{i1},\ldots,n_{il_i}\ap \rightarrow S_i, 1\leq i\leq m, 1\leq j\leq
l_i\}\\
&\cup&\{n_{ij}\ap S_{i1}\times\cdots\times S_{ik_i}\rightarrow S_i\mid\\
&&~~~~~\fd_i\equiv
n_{i1},\ldots,n_{il_i}\ap S_{i1}\times\cdots\times S_{ik_i}\rightarrow
S_i, 1\leq i\leq
m, 1\leq j\leq l_i\}.
\end{array}
\]
\item
If $E\equiv
\mapkw~\md_1\cdots \md_m$ with $m\geq 1$,
then
$\Sig(E)\wor(\emptyset,\emptyset,\Map, \emptyset,\emptyset,
\emptyset,\emptyset),$
where \[\begin{array}{lll}
\Map &\wor&
\{n_{ij}\ap \rightarrow S_i\mid \md_i\equiv
n_{i1},\ldots,n_{il_i}\ap \rightarrow S_i, 1\leq i\leq m, 1\leq j\leq
l_i\}\\
&\cup&\{n_{ij}\ap S_{i1}\times\cdots\times S_{ik_i}\rightarrow S_i\mid\\
&&~~~~~\md_i\equiv
n_{i1},\ldots,n_{il_i}\ap S_{i1}\times\cdots\times S_{ik_i}\rightarrow
S_i, 1\leq i\leq
m, 1\leq j\leq l_i\}.
\end{array}
\]
\item
If $E$ is a {\tt Equation-specification},
then
$\Sig(E)\wor(\emptyset,\emptyset,\emptyset,\emptyset,\emptyset,\emptyset,\emptyset)$.
\item
If $E\equiv
\act~\ad_1\cdots \ad_m$ with $m\geq 1$,
then
$\Sig(E)\wor(\emptyset,\emptyset,\emptyset,\Act,
\emptyset,\emptyset,
\emptyset)$, where
\[\begin{array}{lll}
\Act&\wor&
\{n_{i}\mid ad_i\equiv
n_{i}, 1\leq i\leq m \}\\
&\cup&\{n_{ij}\ap S_{i1}\times\cdots\times S_{ik_i}\mid\\
&&~~~~~ad_i\equiv
n_{i1},\ldots,n_{il_i}\ap S_{i1}\times\cdots\times S_{ik_i}, 1\leq i\leq
m, 1\leq j\leq l_i\}.
\end{array}
\]
\item
If $E\equiv
\commkw~cd_1\cdots cd_m$ with $m\geq 1$,
then
$\Sig(E)\wor(\emptyset,\emptyset,\emptyset,\emptyset,
\{cd_i\mid 1\leq
i\leq m\}, \emptyset,\emptyset).$
\item
If $E\equiv
\prockw~pd_1\cdots pd_m$ with $m\geq 1$,
then
$\Sig(E)\wor(\emptyset,\emptyset,\emptyset,\emptyset,\emptyset,
\{pd_i\mid 1\leq i\leq m\},\emptyset).$
\item
If $E\equiv
\initkw~\pe$ then $\Sig(E)\wor(\emptyset,\emptyset,\emptyset,\emptyset,\emptyset,
\emptyset,\{\pe\}).$
\item
If $E\equiv E_1~E_2$
with $\Sig(E_i)=(Sort_{i},\Fun_{i},\Map_i,
\Act_{i},\Comm_{i},\Proc_{i},\Init_i)$ for $i=1,2$, then
\[\begin{array}{l}
\Sig(E)\wor(Sort_{1}\cup Sort_{2},\Fun_{1}\cup \Fun_{2},
\Map_1\cup\Map_2,\\
\hspace{8em}\Act_{1}\cup \Act_{2},\Comm_{1}\cup \Comm_{2},\Proc_{1}\cup
\Proc_{2},
\Init_1\cup\Init_2).
\end{array}\]
\end{itemize}
\end{defn}
\begin{defn}
Let $Sig=(Sort,\Fun,\Map,\Act,\Comm,\Proc,\Init)$ be a signature. We write
\[\begin{array}{llll}
\Sig.Sort~\mbox{for}~Sort,&
\Sig.\Fun~\mbox{for}~\Fun,&
\Sig.\Map~{\rm for}~\Map,&
\Sig.\Act~\mbox{for}~\Act,\\
\Sig.\Comm~\mbox{for}~\Comm,&
\Sig.\Proc~\mbox{for}~\Proc,&
\Sig.\Init~\mbox{for}~\Init.
\end{array}\]
\end{defn}
\subsection{Variables}
Variables play an important role in specifications. The next definition
says given a specification $E$, which elements from {\tt Name} can play the
role of a variable without confusion with
defined constants. Moreover, variables must have an unambiguous and
declared sort.
\begin{defn}
Let $Sig$ be a signature. A set $\NV$ containing
pairs
$\langle x\ap S\rangle$ with $x$ and $S$ from {\tt Name},
is called a {\it set of variables} over $Sig$ iff for each $\langle
x\ap S\rangle\in \NV$:
\begin{itemize}
\item
for each {\tt Name} $S'$ and {\tt Process-term} $p$ it holds that
$x\ap \rightarrow S'\notin \Sig.\Fun\cup\Sig.\Map$,
$x\notin\Sig.\Act$ and $x=p\notin\Sig.\Proc$,
\item
$S\in \Sig.Sort$,
\item
for each {\tt Name} $S'$ such that $S'\not\equiv S$ it holds
that $\langle x\ap S'\rangle\not\in \NV$.
\end{itemize}
\end{defn}
\begin{defn}
\newcommand{\vd}{{\it vd}}
Let $\vd$ be a {\tt Variable-declaration}.
The function $\Vars$ is defined by:
\[\Vars(\vd)\wor\left\{
\begin{array}{ll}
\emptyset&\mbox{if $\vd$ is empty},\\
\{\langle x_{ij}\ap S_i\rangle\mid 1\leq i\leq m,\\
\phantom{\{\langle x_i\ap S_{ij}\rangle\mid }\hspace*{1mm}1\leq j\leq
l_i\}&
\mbox{if}~\vd\equiv
\varkw~x_{11},\ldots,x_{1l_1}\ap  S_1~\ldots~x_{m1},\ldots,x_{ml_m}\ap S_m.
\end{array}\right.\]
\end{defn}
In the following definitions we give functions yielding the sort of and the
variables in a
{\tt Data-term} $t$.
\begin{defn}
Let $t$ be a {\tt data-term} and $Sig$ a signature. Let $\NV$ be a set
of variables over $Sig$. We define:
\[sort_{Sig,\NV}(t)\wor\left\{
\begin{array}{ll}
\seq{S}&\mbox{if~}t\equiv x\mbox{ and }\langle x\ap S\rangle \in \NV,\\
\seq{S}&\mbox{if $t\equiv n$, $n\ap \rightarrow S\in
\Sig.\Fun\cup\Sig.\Map$ or in constructors.}\\
\seq{\nm{Pos},\nm{Nat},\nm{Int}}&\mbox{if $t\equiv\aterm{Number}(n)$, $n>0$}\\
\seq{\nm{Nat},\nm{Int}}&\mbox{if $t\equiv\aterm{Number}(0)$}\\
\seq{\nm{Int}}&\mbox{if $t\equiv\aterm{Number}(n)$, $n<0$}\\
\seq{\nm{Bool}}&\mbox{if $t\equiv\aterm{True}()$ or $t\equiv\aterm{False}()$}\\
&\hspace*{0.5cm}\mbox{and for no $S'\not\equiv S$}~n\ap \rightarrow
S'\in\Sig.\Fun\cup\Sig.\Map,\\
S&\mbox{if $t\equiv n(t_1,\ldots,t_m)$,}\\
&\mbox{$\hspace{0.5cm}n\ap sort_{Sig,\NV}(t_1)\times \cdots
\times sort_{Sig,\NV}(t_m)\rightarrow S \in \Sig.\Fun\cup\Sig.\Map$}\\
&\hspace{0.5cm}\mbox{and for
no}~S'\not\equiv S~n\ap sort_{Sig,\NV}(t_1)\times \cdots
\times sort_{Sig,\NV}(t_m)\rightarrow\\
&\hspace*{0.5cm}S' \in \Sig.\Fun\cup\Sig.\Map,\\
\perp&\mbox{otherwise}.
\end{array}\right.\]
\end{defn}
If a variable or a function is not or inappropriately declared
no answer can be obtained. In this case $\perp$ results.
\begin{defn}
Let $Sig$ be a signature, $\NV$ a set of
variables over $Sig$ and let $t$ be a {\tt Data-term}.
\[\Var_{Sig,\NV}(t)\wor
\left\{
\begin{array}{ll}
\{\langle x\ap S\rangle\}&\mbox{if }t\equiv x\mbox{ and }\langle
x\ap S\rangle\in \NV,\\
\emptyset&\mbox{if~}t\equiv n\mbox{ and }n\ap \rightarrow S\in
\Sig.\Fun\cup\Sig.\Map,\\
\bigcup_{1\leq i\leq m} \Var_{Sig,\NV}(t_i)&\mbox{if~}t\equiv
n(t_1,\ldots,t_m),\\
\{\perp\}&\mbox{otherwise}.
\end{array}\right.\]
\end{defn}
We call a {\tt Data-term} $t$ {\em closed} wrt.\ a signature
$Sig$ and a set of variables $\NV$ iff $\Var_{Sig,\NV}(t)=
\emptyset$. Note that $\Var_{Sig,\NV}(t)\subseteq\NV\cup\{\perp\}$
for any {\tt data-term} $t$. If $\perp\in \Var_{\Sig,\NV}(t)$, then
due  to some missing or inappropriate declaration
it can not be determined what the variables of $t$ are on basis of
$\Sig$ and $\NV$.


\subsection{Well-formed \mcrl\ specifications}
We define what well-formed specifications are. We only provide
well-formed {\tt Specification}s with a semantics. Well-formedness
is a decidable property.
\begin{defn}
Let $\Sig$ be a signature. We call a {\tt Name} $S$ a {\it constructor sort}
iff $S\in\Sig.\Sort$ and there exists {\tt Name}s $S_1,\ldots, S_k, f$ ($k\geq 0$)
such that $f\ap S_1\times \cdots\times S_k\rightarrow S\in \Sig.\Fun$.
\end{defn}
\begin{defn}
Let $E$ be a {\tt Specification} that is SSC.
We inductively define which sorts are {\it non empty constructor sorts} in $E$.
A constructor sort $S$ is called {\it non empty} iff there is a function
$f\ap S_1\times\cdots\times S_k\rightarrow S\in \Sig.\Fun$ ($k\geq 0$) such
that for all $1\leq i\leq k$ if $S_i$ is a constructor sort, it is non empty.
We say
that $E$ has {\em no empty constructor sorts} iff each constructor sort is non
empty.
\end{defn}
\begin{defn}
Let $E$ be a {\tt Specification}. $E$ is called {\it well-formed} iff
\begin{itemize}
\item $E$ is SSC,
\item $E$ has no empty constructor sorts,
\item There is no indirect set, bag, or list recursion. A=Set(B), B=Ref(A).
\item There is no empty sort due to nonterminating struct recursion.
  C=struct(leaf(C),node(C,C))
\item If $\Time\in\Sig(E).\Sort$, then $\nul\ap \rightarrow\Time\in\Sig(E).\Fun\cup \Sig(E).\Map$
  and $\leq\ap\Time\times\Time\rightarrow\Bool\in \Sig(E).\Map$.
\end{itemize}
\end{defn}

\newpage
\section{ATerm representation format for MTLPSs}
A MTLPS is stored as an ATerm with the following functions. The sort
of stored MTLPS is $\nm{MTLPS}$.

\begin{gather*}
\afunc{spec2gen}{\nm{DataTypes}\x\nm{ActionSpec}^*\x\nm{InitProcSpec}}{\nm{MTLPS}}\\
\afunc{actspec}{\nm{String}\x\nm{String}^*}{\nm{ActionSpec}}\\
\afunc{initprocspec}{\nm{TermAppl}\x\nm{Variable}^*\x\nm{Summand}^*}{\nm{InitProcSpec}}\\
\afunc{smd}{\nm{Variable}^*\x\nm{Action}^*\x\nm{Time}\x\nm{IndexedTerm}^*\x\nm{TermAppl}}{\nm{Summand}}\\
\afunc{act}{\nm{String}\x\nm{TermAppl}}{\nm{Action}}\\
%\afunc{terminated}{}{\nm{NextState}}\\
%\afunc{i}{\nm{IndexedTerm}^*}{\nm{NextState}}\\
\afunc{time}{\nm{TermAppl}}{\nm{Time}}\\
\afunc{notime}{}{\nm{Time}}\\
\afunc{it}{\Nat\x\nm{TermAppl}}{\nm{IndexedTerm}}\\
\afunc{dc}{\Nat}{\nm{IndexedTerm}}\\
\afunc{d}{\nm{Signature}\x\nm{Equation}^*}{\nm{DataTypes}}\\
\afunc{e}{\nm{Variable}^*\x\nm{TermAppl}\x\nm{TermAppl}\x\nm{TermAppl}}{\nm{Equation}}\\
\afunc{v}{\nm{String}\x\nm{String}}{\nm{Variable}}\\
\afunc{s}{\nm{String}^*\x\nm{Function}^*\x\nm{Function}^*}{\nm{Signature}}\\
\afunc{f}{\nm{String}\x\nm{String}^{*}\x\nm{String}}{\nm{Function}}
\end{gather*}
The sort $\nm{TermAppl}$ consists of ATerm terms of the form
$\nm{TermAppl}(f,t)$ or constant/variable symbols. The sort
$\nm{String}$ consists of quoted constants, i.e.\ function symbols of
arity 0. The sort $\Nat$ is the built-in sort of natural numbers in
the ATerm library. The list of elements of sort $D$ is denoted by
$D^{*}$.

The constructor of sort $\nm{InitProcSpec}$ contains the actual LPS
parameters (from \texttt{init}) as the first parameter, the formal LPS
parameters as the second argument, and the list of summands as the
third parameter. The third parameter of $\mathsf{smd}$ is the term of
sort $\nm{Time}$ representing the time at which the multiaction
happens, or $\mathsf{notime}$, indicating that no time info is given.
The last parameter of $\mathsf{smd}$ is the boolean term representing
the condition.

The second parameter of $\mathsf{e}$ is the boolean condition used for
conditional term rewriting.

The first parameter of $\mathsf{v}$ is the variable name, appended with '\#'.
The first parameter of $\mathsf{f}$ is the function name, appended
with its parameter types list, separated by '\#' (for constants only '\#' is appended).

If the delta summand of the TLPS is present, $\delta$ has to be
represented by the ATerm string \texttt{"Delta"}, and actions with
this name should not be allowed.  An alternative is in using a special
summand construction.

\newpage
\section{ATerm representation format for LPSs (for \mcrl\ v1)}
An LPS is stored as an ATerm with the following functions. The sort
of stored LPS is $\nm{LPS}$.

\begin{gather*}
\afunc{spec2gen}{\nm{DataTypes}\x\nm{InitProcSpec}}{\nm{LPS}}\\
\afunc{initprocspec}{\nm{Term}^*\x\nm{Variable}^*\x\nm{Summand}^*}{\nm{InitProcSpec}}\\
\afunc{smd}{\nm{Variable}^*\x\nm{String}\x\nm{Term}^*\x\nm{NextState}\x\nm{Term}}{\nm{Summand}}\\
\afunc{terminated}{}{\nm{NextState}}\\
\afunc{i}{\nm{Term}^*}{\nm{NextState}}\\
\afunc{d}{\nm{Signature}\x\nm{Equation}^*}{\nm{DataTypes}}\\
\afunc{e}{\nm{Variable}^*\x\nm{Term}\x\nm{Term}}{\nm{Equation}}\\
\afunc{v}{\nm{String}\x\nm{String}}{\nm{Variable}}\\
\afunc{s}{\nm{String}^*\x\nm{Function}^*\x\nm{Function}^*}{\nm{Signature}}\\
\afunc{f}{\nm{String}\x\nm{String}^{*}\x\nm{String}}{\nm{Function}}
\end{gather*}
The sort $\nm{Term}$ consists of arbitrary ATerm terms where all function
symbols must be quoted. The sort $\nm{String}$ consists of quoted constants,
i.e.\ function symbols of arity 0. The list
of elements of sort $D$ is denoted by $D^{*}$.

The first parameter of $\mathsf{v}$ is the variable name, appended with '\#'.
The first parameter of $\mathsf{f}$ is the function name, appended
with its parameter types list, separated by '\#' (for constants only '\#' is appended).

The constructor of sort $\nm{InitProcSpec}$ contains the actual LPS
parameters (from \texttt{init}) as the first parameter, the formal LPS
parameters as the second argument, and the list of summands as the
third parameter. The last parameter of $\mathsf{cmd}$ is the boolean term
representing the condition.

\newpage
\section{ATerm representation format for input muCRL (for \mcrl\ v1)}
An LPS is stored as an ATerm with the following functions. The sort
of stored LPS is $\nm{LPS}$.

\begin{gather*}
\afunc{spec2gen}{\nm{DataTypes}\x\nm{InitProcSpec}}{\nm{LPS}}\\
\afunc{initprocspec}{\nm{Term}^*\x\nm{Variable}^*\x\nm{Summand}^*}{\nm{InitProcSpec}}\\
\afunc{smd}{\nm{Variable}^*\x\nm{String}\x\nm{Term}^*\x\nm{NextState}\x\nm{Term}}{\nm{Summand}}\\
\afunc{terminated}{}{\nm{NextState}}\\
\afunc{i}{\nm{Term}^*}{\nm{NextState}}\\
\afunc{d}{\nm{Signature}\x\nm{Equation}^*}{\nm{DataTypes}}\\
\afunc{e}{\nm{Variable}^*\x\nm{Term}\x\nm{Term}}{\nm{Equation}}\\
\afunc{v}{\nm{String}\x\nm{String}}{\nm{Variable}}\\
\afunc{s}{\nm{String}^*\x\nm{Function}^*\x\nm{Function}^*}{\nm{Signature}}\\
\afunc{f}{\nm{String}\x\nm{String}^{*}\x\nm{String}}{\nm{Function}}
\end{gather*}
The sort $\nm{Term}$ consists of arbitrary ATerm terms where all function
symbols must be quoted. The sort $\nm{String}$ consists of quoted constants,
i.e.\ function symbols of arity 0. The list
of elements of sort $D$ is denoted by $D^{*}$.

The first parameter of $\mathsf{v}$ is the variable name, appended with '\#'.
The first parameter of $\mathsf{f}$ is the function name, appended
with its parameter types list, separated by '\#' (for constants only '\#' is appended).

The constructor of sort $\nm{InitProcSpec}$ contains the actual LPS
parameters (from \texttt{init}) as the first parameter, the formal LPS
parameters as the second argument, and the list of summands as the
third parameter. The last parameter of $\mathsf{cmd}$ is the boolean term
representing the condition.

%\bibliographystyle{plain}
%\bibliography{mCRL2}

\end{document}
