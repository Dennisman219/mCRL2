\documentclass[a4paper,10pt]{article}
\usepackage{textcomp,amsmath,amssymb,amsthm,stmaryrd}
\usepackage{geometry}
\usepackage{mymath,mythm}
\usepackage{algorithm,algorithmic}

\theoremstyle{plain}
\newtheorem{thmfs}{Theorem}[section]
\theoremstyle{definition}
\newtheorem{tcase}[thmfs]{Test case}

\newcommand{\mcrl}{mCRL2}
\newcommand{\lps}{linear process specification}
\newcommand{\tool}{\textit{lpsparelm}}
\newcommand{\ti}{\textit}
\newcommand{\tb}{\textbf}

\newcommand{\ovr}{\overrightarrow}

\newcommand{\pp}{process parameter}
\newcommand{\pps}{process parameters}
\newcommand{\framework}{\textit{LPS framework} \cite{LPSframework}}

\newcommand{\bisim}{\frac{\leftrightarrow}{}}

\newcommand{\tab}{\hspace*{5.mm} \= \hspace*{5.mm} \= \hspace*{5.mm} \= \hspace*{5.mm} \= \hspace*{5.mm} \= \hspace*{5.mm}  \= \hspace*{5.mm}  \= \hspace*{5.mm}  \= \hspace*{5.mm} \= \hspace*{5.mm} \= \hspace*{5.mm}  \= \hspace*{5.mm}  \= \hspace*{5.mm}\kill}

\newcommand{\tabw}{\hspace*{15.mm} \= \hspace*{20.mm} \= \hspace*{5.mm} \= \hspace*{5.mm} \= \hspace*{5.mm} \= \hspace*{5.mm}  \= \hspace*{5.mm}  \= \hspace*{5.mm}  \= \hspace*{5.mm} \= \hspace*{5.mm} \= \hspace*{5.mm}  \= \hspace*{5.mm}  \= \hspace*{5.mm}\kill}

\font \aap cmmi10
\newcommand{\at}[1]{\mbox{\aap ,} #1}

%opening
\title{lpsparelm}
\author{F.P.M.Stappers}

\begin{document}

\maketitle

\begin{abstract}
This documentation describes the implementation and test cases of the tool \tool\ in  the \mcrl\ toolset.
Basically, \tool\ is a tool which removes unused \pp\ and unused sum variables from a \lps\ (LPS).
\end{abstract}

\tableofcontents

\section{Introduction}
The \tool\ tool is a tool for \mcrl\ studio. The tool is a
filter which reads from file \ti{input.lps}, which is
a file in \ti{.lps} format \cite{LPSformat}. We make use of the
\framework\ to read \ti{input.lps}. The filter removes \pps\ and sum variables, which are not used, from the
\lps. After the algorithm (Section \ref{sec:alg}) terminates, \tool\
will write the output to an output file \ti{output.lps} (in the \ti{.lps} format.)


\section{Definitions} \label{sec:def}

The equation below is a \lps\ in \mcrl :
\begin{defn}\lps\ (LPS) \newline
\begin{tabbing}
\tab
$X (\ovr{d: D}) = $ \> \> \> $ \sum_{i \in I} \sum_{\ovr{e_i: E_i}} \ovr{c_i} ( \ovr{d, e_i }) \rightarrow
(a_i^1 (\ovr{f_{i,1}}(\ovr{d,e_i})) \vert \ldots \vert a_i^{n(i)}(\ovr{f_{i,n(i)}}(\ovr{d,e_i}))) \at \text{ } t_i(\ovr{d,e_i})  \cdot X(\ovr{g_i}(\ovr{d,e_i})) +$ \\ \\
\> \> \> $ \sum_{j \in J} \sum_{\ovr{e_j: E_j}} \ovr{c_j} ( \ovr{d, e_j} ) \rightarrow
(a_j^1 (\ovr{f_{j,1}}(\ovr{d,e_j})) \vert \ldots \vert a_j^{n(j)}(\ovr{f_{j,n(j)}}(\ovr{d,e_j}))) \at \text{ } t_j(\ovr{d,e_j}) + $ \\ \\
\> \> \> $\sum_{\ovr{e_\delta}:\ovr{E_\delta}} \ovr{c_\delta} ( \ovr{d, e_\delta}) \rightarrow
\delta \at \text{ } t_\delta(\ovr{d,e_\delta})$
\end{tabbing}

Where $I$ and $J$ are disjoint.\\
\end{defn}

If we speak about an LPS in this article we refer to Definition \ref{sec:def}.  The different states
of the process are represented by the data vector parameter $\ovr{d}:\ovr{D}$. $\ovr{D}$ may be a Cartesian product of $n$ data types, meaning that $\ovr{d}$ consist of a tuple $(d_1, \ldots, d_n)$. The LPS expresses that in state $\ovr{d}$ it performs (multi)actions $a_i^1, \ldots , a_i^{n(i)}$, carrying data parameters $\ovr{f_{i,1}}(\ovr{d,e_i}), \ldots , \ovr{f_{i,n(i)}}(\ovr{d, e_i})$ and
it can reach the new state $\ovr{g_i}(\ovr{d ,e_i})$ under the condition that $c_i(\ovr{d , e_i})$ is \ti{true}. So for each summand $i$ from $I$ we have a function $\ovr{g_i}: \ovr{D} \times \ovr{E_i} \rightarrow \ovr{D}$ and a function $c_i: \ovr{D} \times \ovr{E_i} \rightarrow \mathbb{B}$.
Data parameters $\ovr{e_i : E_i}$ are sum variables. These variables are used to action range over a data domain.

For an more detailed explanation of \lps s  we refer to \cite{LPS_info}.

\section{Definition lpsparelm}
A parameter of an LPS which has no influence on condition, action arguments and time is removed from the LPS in the \tool\ filter. Elimination of parameters can lead to a reduction when generating a state space of an LPS. If a \pp\ ranges over an infinite domain, the number of generated states can even be reduced to a finite domain, using this operation. %This typically happens after hiding parts of a specification.

\begin{defn}
Let $\ovr{x}$ be a vector and  $n$ be $\vert \ovr{x} \vert$. If we want to address the $j^{th}$ element ($ 1 \leq j \leq \vert \ovr{x} \vert $) of vector $\ovr{x}$, the do so by:
$\ovr{x}._j$
\end{defn}

\begin{defn}
Let $w \in I \cup J$, where $I$ and $J$ are the set of summand indices; We define \\
 $\ovr{vect^w} = (\ovr{d , e_w})$\\
where $k = \vert \ovr{d} \vert$ and $z = \vert \ovr{e_w} \vert$. The length of $ \ovr{vect^w}$ is equal to $k + z$.
\end{defn}

\begin{defn}[$\boxtimes^w_X$]
Let $w \in W$, where $W$ is a set of summand indices.\\
We define $\boxtimes^w_X$:\\
Let $j \in \lbrace 1, \dots, \vert \ovr{vect^w} \vert \rbrace.$\\
If $j \not\in X$ then $\boxtimes^w_X._j = y \hspace{40pt} \text{where }y \in \ovr{D}._j$ \\
If $j \in X$ then $\boxtimes^w_X._j = \ovr{vect^w}._j$ \\
\end{defn}

% \begin{defn}[\texttt{Dom($x$)}] Let $x$ be an variable, then \texttt{Dom$(x)$} is the collection of all values which variable $x$ can be.
% \end{defn}

\newcommand{\zi}{{\boxtimes}^k_{S_L}}
\newcommand{\zj}{{\boxtimes}^l_{S_L}}
\newcommand{\zd}{{\boxtimes}^\delta_{S_L}}

% \begin{defn}(L) \label{def:lps:L} We define $M$:
% \begin{tabbing}
% \tab
%  \>$\sum_{i \in I} \sum_{\ovr{e_i:E_i}} c_i ( \ovr{vect^i} ) \rightarrow $
% $ a_i^1 (\ovr{f_{i,1}}(\ovr{vect^i})) \vert \ldots \vert a_i^{n(i)}(\ovr{f_{i,n(i)}}(\ovr{vect^i})) \at \text{ } t_i(\ovr{vect^i})  \cdot X(\ovr{g_i}(\ovr{vect^i}))
% $ \\
%  \> $+ \sum_{j \in J} \sum_{\ovr{e_j:E_j}} c_j ( \ovr{vect^j} ) \rightarrow $
% $ a_j^1 (\ovr{f_{j,1}}(\ovr{vect^j})) \vert \ldots \vert a_j^{n(j)}(\ovr{f_{j,n(j)}}(\ovr{vect^j})) \at \text{ } t_j(\ovr{vect^j})
% $ \\
%  \> $ + \sum_{\ovr{e_\delta:E_\delta}} c_\delta ( \ovr{d, e_\delta}) \rightarrow
% \delta \at \text{ } t_\delta(\ovr{vect^{\delta}})$ \\
% \end{tabbing}
% \end{defn}

\begin{defn} \label{def:lps:M} Let $M$ and $L$ be an LPS, with $ S_L \subseteq \lbrace 1, \ldots, n \rbrace $ as the biggest possible set such that:
$M \bisim L$, with $M$ defined as follows:
\begin{tabbing}
\tab
$Y(\ovr{d:D}) = $ \\
\> \>$\sum_{k \in I} \sum_{\ovr{e_k:E_k}} c_k ( \zi ) \rightarrow $
$ a_k^1 (\ovr{f_{k,1}}(\zi) \vert \ldots \vert a_k^{n(k)}(\ovr{f_{k,n(k)}}(\zi) \at \text{ } t_{k}(\zi)  \cdot Y(\ovr{g_{k}}(\zi)
$ \\
\> $+ $\> $\sum_{l \in J} \sum_{\ovr{e_l:E_L}} c_l ( \zj ) \rightarrow $
$ a_{l}^1 (\ovr{f_{l,1}}(\zj) \vert \ldots \vert a_{l}^{n(l)}(\ovr{f_{l,n(l)}}(\zj) \at \text{ } t_{l}(\zj)
$ \\
\> $ + $ \> $ \sum_{\ovr{e_{\delta}:E_{\delta^1}}} c_{\delta} ( \zd ) \rightarrow
{\delta} \at \text{ } t_{\delta}(\zd)$\\
\end{tabbing}
\end{defn}

\section{Description} \label{sec:desc}
It is possible having \pps\ which can have any value without influencing the behavior of the LPS.
If we look at the next states in an LPS in the LPS-format, we see that the LPS only contains \pps\ which influence the behavior and it omits those \pps\ which do not influence the behavior. So, if \pps\ do not occur in condition, action arguments or time tags, they do not influence the behavior in a direct way. If we collect all \pps , which are used in the condition, action arguments and time tags, we know which \pps\ are used. To indicate which \pps\ are used, we keep a set $UsedPP$\footnote{$UsedPP = S_M$ : Due the length off the word $UsedPP$ the Definition \ref{def:lps:M} would get to long and clipped} which contains all indices used \pps . To find all \pps in a condition, action argument or time tag we have function called $FindUsedPP$. This function returns a set of  \pp\ indices which occur in a given condition, action argument or time tag.

A next state is always used as an input state, to calculate another next state. So for each \pp\ we have to check while  all \pps\ which are used to construct a next state are in the set of used \pps . If a \pp\ is used as an input and is not in the set of used \pps\ this \pp\ is added to the set of indices.
Also, we have to check, that all \pps\ on which the added \pp\ depends, are also in the list of used \pps. We continue adding \pps\ dependant on other \pps\ which are also in the set of used \pp\ indices.

In short: Throughout the run of the algorithm \pps\ get marked if they are used in an action, time or condition. Process parameters which are dependant of other marked \pps\ also get marked. If a \pp\ gets marked it should not be removed. Initially, all parameters in the LPS are unmarked. After the algorithm terminates all unmarked \pps\ are removed. A \pp\ gets marked if it occurs in one of the following places:

\begin{itemize}
\item In a condition: $c_i(\ovr{vect^i})$ for some $i \in I \cup J$
\item In an action argument: %$a_i^j
$(\ovr{f_{i,j}}(\ovr{vect^i})$ for some $i \in I \cup J$ and $j \in \lbrace 1, \ldots , n \rbrace$
\item In a time tag: $t_i(\ovr{vect^i})$ for some $i \in I$
\item In an argument: $\ovr{g_i}(\ovr{vect^i})_j$ for some $i \in I \cup J$ and $j \in \lbrace 1, \ldots , n \rbrace $ where the $j$-th \pp\ is marked.
\end{itemize}

\newpage
\section{Algorithm} \label{sec:alg}
\begin{algorithm}[h]
\caption{lpsparelm}
\begin{algorithmic}[1]
\STATE $UsedPP := \emptyset $
  \FORALL{$i \in I \cup J $}
    \STATE $UsedPP := UsedPP \cup FindUsedPP(\ovr{c_i}(\ovr{vect^i}))$
    \FOR{$j := 1$ to $n(i) $}
      \STATE $UsedPP := UsedPP \cup FindUsedPP(\ovr{f_{i,j}}(\ovr{vect^i})$
    \ENDFOR
    \STATE $UsedPP := UsedPP \cup FindUsedPP(\ovr{t_i}(\ovr{vect^i}))$
  \ENDFOR
\STATE $n := 0$
\WHILE{$\vert UsedPP \vert > n $}
\STATE $n := \vert UsedPP \vert$
  \FORALL{$ i \in I$}
    \FORALL{$j \in UsedPP$}
      \STATE $UsedPP := UsedPP \cup FindUsedPP(\ovr{g_i}(\ovr{vect^i})._j) $
    \ENDFOR
  \ENDFOR
\ENDWHILE
\RETURN $UsedPP$
\end{algorithmic}
\line(1,0){120}
\begin{tabbing}
\tab
$UsedPP$ \> \> \> \> : \> The set of indices of used \pps .\\
$FindUsedPP$ \> \> \> \> : \> A routine which returns a set of indices of used\\
\> \> \> \> \> \pps, which occur in a given argument.
\end{tabbing}
\end{algorithm}

\newpage
\section{Test Cases}
All specifications are given in \mcrl\ specification.
Transformation from a \mcrl\ specification to an LPS file is done with the tool:
$mcrl22lps$. Each transformation is executed with the \verb"-no-cluster" option, unless
mentioned otherwise.

\subsection*{Case 1}
inputfile: \verb"$DIR$/tests/lpsparelm/case1.mcrl2"
\begin{tabbing}
\tabw
\ti{info} \> This test case will show if if the \tool\ works without condition, action and time arguments.\\
\line(1,0){120}\\
\tb{act} \> a; \\
\tb{proc} \> X(i:Nat) = \> a.X(i)\\
\tb{init} \> X(2); \\
\line(1,0){120}\\
\ti{Result} \> Process parameter $i$ should be removed.\\
\end{tabbing}

\subsection*{Case 2}
inputfile: \verb"$DIR$/tests/lpsparelm/case2.mcrl2"
\begin{tabbing}
\tabw
\ti{info} \> This test case will show if if the \tool\ works with an action argument.\\
\line(1,0){120}\\
\tb{act} \> a: Nat; \\
\tb{proc} \> X(i,j:Nat) = \> a(i). X(i,j)\\
\tb{init} \> X(0,1); \\
\line(1,0){120}\\
\ti{Result} \> Process parameter $j$ should be removed.\\
\end{tabbing}

\subsection*{Case 3}
inputfile: \verb"$DIR$/tests/lpsparelm/case3.mcrl2"
\begin{tabbing}
\tabw
\ti{info} \> This test case will show if if the \tool\ works with a condition argument.\\
\line(1,0){120}\\
\tb{act} \> a; \\
\tb{proc} \> X(i,j:Nat) = \> (i == 5 ) $\rightarrow$ a. X(i,j)\\
\tb{init} \> X(0,1); \\
\line(1,0){120}\\
\ti{Result} \> Process parameter $j$ should be removed.\\
\end{tabbing}

\subsection*{Case 4}
inputfile: \verb"$DIR$/tests/lpsparelm/case4.mcrl2"
\begin{tabbing}
\tabw
\ti{info} \> This test case will show if if the \tool\ works with a time argument.\\
\line(1,0){120}\\
\tb{act} \> a; \\
\tb{proc} \> X(i,j:Nat) = \> a$@$i. X(i,j)\\
\tb{init} \> X(0,4); \\
\line(1,0){120}\\
\ti{Result} \> Process parameter $j$ should be removed.\\
\end{tabbing}

\subsection*{Case 5}
inputfile: \verb"$DIR$/tests/lpsparelm/case5.mcrl2"
\begin{tabbing}
\tabw
\ti{info} \> This test case will show if the \tool\ checks for dependant \pps .\\
\line(1,0){120}\\
\tb{act} \> a: Nat ; \\
\tb{act} \> b; \\
\tb{proc} \> X(i,j,k:Nat) = \> a(i).X(k,j,k) + \\
          \> \>              b.X(j,j,k);\\
\tb{init} \> X(1,2,3); \\
\line(1,0){120}\\
\ti{Result} \> No process parameters should be removed.\\
\end{tabbing}

\subsection*{Case 6}
inputfile: \verb"$DIR$/tests/lpsparelm/case6.mcrl2"
\begin{tabbing}
\tabw
\ti{info} \> This test case will show how the tool reacts on free variables and multiple summands.\\
\line(1,0){120}\\
\tb{act} \> act1, act2, act3: Nat; \\
\tb{proc} \> X(i: Nat) \>  = \> (i \verb"<" 5) $\rightarrow$ act1(i).X(i+1) + \\
\> \> \>     (i == 5) $\rightarrow$ act3(i).Y(i, i); \\
\>     Y(i,j: Nat) \> = \> act2(j).Y(i,j+1); \\
\tb{init} X(0); \\
\line(1,0){120}\\
\ti{Result} \> No \pp\ should be removed.\\
\> \textbf{Note:} Different linearization might lead to other results.\\
\line(1,0){120}\\
\ti{Genarated LPS} \\
\>  \verb"var  freevar0: Nat;"\\
\>  \verb"proc P(s3: Pos, i,j: Nat)"\\
\>  \verb" (s3 == 1 && i < 5) ->"\\
\>  \verb"   act1(i) ."\\
\>  \verb"   P(s3 := 1, i := i + 1, j := freevar0)"\\
\>  \verb"+ (s3 == 1 && i == 5) ->"\\
\>  \verb"   act3(i) ."\\
\>  \verb"   P(s3 := 2, j := i)"\\
\>  \verb"+ (s3 == 2) ->"\\
\>  \verb"   act2(j) ."\\
\>  \verb"   P(s3 := 2, j := j + 1);"\\
\\
\>  \verb"var  freevar: Nat;"\\
\>  \verb"init P(s3 := 1, i := 0, j := freevar);"\\
\end{tabbing}

\subsection*{Case 7}
inputfile: \verb"$DIR$/tests/lpsparelm/case7.mcrl2"
\begin{tabbing}
\tabw
\ti{info} \> This test case will show how the tool reacts on time, actions, conditions, multiple \\
	\> summands and dependant \pps.\\
\line(1,0){120}\\
\tb{act}  \> act1, act2, act3: Nat;\\
\tb{proc} \> X(i,z,j: Nat) \>    = \>  (i \verb"<"  5) $\rightarrow$ act1(i)$@$z.X(i+1,z, j) +\\
          \>               \>  \> (i == 5) $\rightarrow$ act3(i).X(i, j, 4); \\
\tb{init} X(0,5, 1);\\
\line(1,0){120}\\
\ti{Result} \> No \pp\ should be removed.\\
\line(1,0){120}\\
\ti{Genarated LPS} \\
\>  \verb" proc P(i,z,j: Nat) ="\\
\>  \verb"       (i == 5) ->"\\
\>  \verb"         act3(i) ."\\
\>  \verb"         P(z := j,j := 4)"\\
\>  \verb"     + (i < 5) ->"\\
\>  \verb"         act1(i) @ Nat2Real(z) ."\\
\>  \verb"         P(i := i + 1);"\\
\\
\>  \verb" init P(i := 0, z := Pos2Nat(5), j := Pos2Nat(1));"\\
\end{tabbing}

\begin{thebibliography}{99}
   \bibitem{LPS_info} unknown author \\
   \textit{Article not ready at the moment},
  \bibitem{LPSframework} J.W. Wesselink,
   \textit{http://www.win.tue.nl/~wieger/mcrl2/html/index.html}\\
   A C++ wrapper for the ATerm library.
   \bibitem{LPSformat} Aad Mathijssen\\
   \textit{https://svn.win.tue.nl/viewcvs-checkout/MCRL2/trunk/specs/mcrl2.internal.txt},
   A description of the internal format of the mCRL2 language.



\end{thebibliography}

\end{document}
