\documentclass{article}
\usepackage[utf8]{inputenc}
\usepackage{fullpage}
\usepackage{amsmath}
\usepackage{amsfonts}
\usepackage{amssymb}
\usepackage{subfiles}
\usepackage{blindtext}
\usepackage{xspace}
\usepackage[dvipsnames]{xcolor}

\setcounter{MaxMatrixCols}{10}

\newtheorem{theorem}{Theorem}
\newtheorem{remark}[theorem]{Remark}

\font \aap cmmi10
\newcommand{\at}[1]{\mbox{\aap ,} #1}
\newcommand{\ap}{{:}}
\newcommand{\tuple}[1]{\ensuremath{\langle {#1} \rangle}}
\newcommand{\vars}{\mathit{vars}}
\newcommand{\up}{\blacktriangle}
\newcommand{\down}{\blacktriangledown}
\newcommand{\concat}{\ensuremath{+\!\!+\:}}
\newcommand{\aftertime}{\ensuremath{<\!\!<}}
\newcommand{\emptymap}{\ensuremath{\{ : \}}}
\newcommand{\emptylist}{\ensuremath{[\:]}}
\newcommand{\stochasticstate}[2]{\ensuremath{\{#1 \mapsto #2\}}}

%--- pseudo code ---%
\newcommand{\Space}{\text{\ }}
\newcommand{\If}{\text{\textbf{if }}}
\newcommand{\Do}{\text{\textbf{do }}}
\newcommand{\Then}{\text{\textbf{then }}}
\newcommand{\Else}{\text{\textbf{else }}}
\newcommand{\For}{\text{\textbf{for }}}
\newcommand{\While}{\text{\textbf{while }}}
\newcommand{\Break}{\text{\textbf{break }}}
\newcommand{\Continue}{\text{\textbf{continue }}}
\newcommand{\Choose}{\text{\textbf{choose }}}
\newcommand{\Return}{\text{\textbf{return }}}

\title{State Space Exploration}
\author{Wieger Wesselink}

\begin{document}
\maketitle

\section{Graph Exploration}

State space exploration is an instance of graph exploration. Consider a directed graph
and take a node $s_0$. We assume there is a function $successors$ that returns the successor
nodes of a vertex. An abstract algorithm for exploring the graph starting from vertex
$s_0$ is

\[
\begin{tabular}{l}
\textsc{ExploreGraph}($s_0$) \\
$todo := \{ s_0 \}$ \\
$discovered := \{ s_0 \}$ \\
$\While todo \neq \emptyset \Space \Do$ \\
\qquad \Choose $s \in todo$ \\
\qquad $todo := todo \setminus \{s\}$ \\
\qquad $discovered := discovered \cup \{s\}$ \\
\qquad $\For s' \in successors(s)  \Space \Do$ \\
\qquad \qquad $\If s' \notin discovered \Space \Then$ \\
\qquad \qquad \qquad $discovered := discovered \cup \{s' \}$ \\
\qquad \qquad \qquad $todo := todo \cup \{ s' \}$ \\
\end{tabular}
\]

\subsection{Event points}
There are many different applications of state space exploration. The Boost Graph Library
(\cite{2002:BGL:504206}) uses a clever idea to separate such applications from the exploration
itself. It is done by distinguishing \emph{event points} in the algorithm that the user can respond
to by means of callback functions. For our purposes we select the following events:

\begin{center}
\label{table:eventpoints}
\begin{tabular}{ |l|l| } 
\hline
\textsf{discover\_state} & is invoked when a state is encountered for the first time \\ 
\textsf{examine\_transition} & is invoked on every transition \\ 
\textsf{start\_state} & is invoked on a state right before its outgoing transitions are being explored \\ 
\textsf{finish\_state} & is invoked on a state after all of its outgoing transitions have been explored \\ 
\hline
\end{tabular}
\end{center}
The events are named in terms of states and transitions instead of vertices and edges, since this is
closer to our application domain. The exploration algorithm with event points included looks like this:

\[
\begin{tabular}{l}
\textsc{ExploreGraph}($s_0, \textsf{discover\_state}, \textsf{examine\_transition},
\textsf{start\_state}, \textsf{finish\_state}$) \\
$todo := \{ s_0 \}$ \\
$discovered := \{ s_0 \}$ \\
\colorbox{lightgray}{$\textsf{discover\_state}(s_0)$} \\
$\While todo \neq \emptyset \Space \Do$ \\
\qquad \Choose $s \in todo$ \\
\qquad $todo := todo \setminus \{s\}$ \\
\qquad \colorbox{lightgray}{$\textsf{start\_state}(s)$} \\
\qquad $discovered := discovered \cup \{s\}$ \\
\qquad $\For s' \in successors(s)  \Space \Do$ \\
\qquad \qquad $\If s' \notin discovered \Space \Then$ \\
\qquad \qquad \qquad $discovered := discovered \cup \{s' \}$ \\
\qquad \qquad \qquad \colorbox{lightgray}{$\textsf{discover\_state}(s')$} \\
\qquad \qquad \qquad $todo := todo \cup \{ s' \}$ \\
\qquad \qquad \colorbox{lightgray}{$\textsf{examine\_transition}(s, a, s')$} \\
\qquad \colorbox{lightgray}{$\textsf{finish\_state}(s)$} \\
\end{tabular}
\]

\newpage
\section{Applications}

Many applications can be easily expressed in terms of the given event points. 

\subsection{Deadlock checking}
With deadlock checking we are looking for states that have no outgoing transitions.
By introducing one boolean variable has\_transitions we can implement deadlock checking as follows.
The callback functions are given at the right.

\[
\begin{tabular}{p{7cm}p{6cm}}
\textsc{FindDeadlock}($s_0, \textsf{discover\_state}, \textsf{examine\_transition},
\textsf{start\_state}, \textsf{finish\_state}$) \\
& \colorbox{lightgray}{bool has\_transitions} \\
$todo := \{ s_0 \}$ \\
$discovered := \{ s_0 \}$ \\
$\textsf{discover\_state}(s_0)$ \\
$\While todo \neq \emptyset \Space \Do$ \\
\qquad \Choose $s \in todo$ \\
\qquad $todo := todo \setminus \{s\}$ \\
\qquad $\textsf{start\_state}(s)$ & \colorbox{lightgray}{has\_transitions := false} \\
\qquad $discovered := discovered \cup \{s\}$ \\
\qquad $\For s' \in successors(s)  \Space \Do$ \\
\qquad \qquad $\If s' \notin discovered \Space \Then$ \\
\qquad \qquad \qquad $discovered := discovered \cup \{s' \}$ \\
\qquad \qquad \qquad $\textsf{discover\_state}(s')$ \\
\qquad \qquad \qquad $todo := todo \cup \{ s' \}$ \\
\qquad \qquad $\textsf{examine\_transition}(s, a, s')$ & \colorbox{lightgray}{has\_transitions := true} \\
\qquad $\textsf{finish\_state}(s)$ & \colorbox{lightgray}{if (!has\_transitions) report\_deadlock(s)} \\
\end{tabular}
\]

\newpage
\section{Search strategies}

Exploration can be done with different search strategies. We describe three of them: breadth-first, depth-first and highway. They mainly differ in the order in which the elements of the todo set are processed. In breadth-first search nodes at the present depth are explored before nodes at a higher depth. In depth-first search the highest-depth nodes are explored first. Highway search is a variant that uses a breadth-first search, but it only explores a part of the state space.

In all three cases the $todo$ list is stored in a double ended queue. We use the slicing operator to denote parts of a list. For example, $A[m:n]$ corresponds to the sublist $A[m,\ldots,n-1]$.

\subsection{Breadth-first search}
\[
\begin{tabular}{l}
\textsc{ExploreGraphBreadthFirst}($s_0, \textsf{discover\_state}, \textsf{examine\_transition},
\textsf{start\_state}, \textsf{finish\_state}$) \\
$todo := \colorbox{lightgray}{$[s_0]$}$ \\
$discovered := \{ s_0 \}$ \\
$\textsf{discover\_state}(s_0)$ \\
$\While \colorbox{lightgray}{$|todo| > 0$} \Do$ \\
\qquad \colorbox{lightgray}{$s := todo[0]$} \\
\qquad \colorbox{lightgray}{$todo := todo[1:|todo|]$} \\
\qquad $\textsf{start\_state}(s)$ \\
\qquad $discovered := discovered \cup \{s\}$ \\
\qquad $\For s' \in successors(s)  \Space \Do$ \\
\qquad \qquad $\If s' \notin discovered \Space \Then$ \\
\qquad \qquad \qquad $discovered := discovered \cup \{s' \}$ \\
\qquad \qquad \qquad $\textsf{discover\_state}(s')$ \\
\qquad \qquad \qquad \colorbox{lightgray}{$todo := todo \concat [s']$} \\
\qquad \qquad $\textsf{examine\_transition}(s, a, s')$ \\
\qquad $\textsf{finish\_state}(s)$ \\
\end{tabular}
\]

\subsection{Depth-first search}
\[
\begin{tabular}{l}
\textsc{ExploreGraphDepthFirst}($s_0, \textsf{discover\_state}, \textsf{examine\_transition},
\textsf{start\_state}, \textsf{finish\_state}$) \\
$todo := \colorbox{lightgray}{$[s_0]$}$ \\
$discovered := \{ s_0 \}$ \\
$\textsf{discover\_state}(s_0)$ \\
$\While \colorbox{lightgray}{$|todo| > 0$} \Do$ \\
\qquad \colorbox{lightgray}{$s := todo[|todo| - 1]$} \\
\qquad \colorbox{lightgray}{$todo := todo[0:|todo| - 1]$} \\
\qquad $\textsf{start\_state}(s)$ \\
\qquad $discovered := discovered \cup \{s\}$ \\
\qquad $\For s' \in successors(s)  \Space \Do$ \\
\qquad \qquad $\If s' \notin discovered \Space \Then$ \\
\qquad \qquad \qquad $discovered := discovered \cup \{s' \}$ \\
\qquad \qquad \qquad $\textsf{discover\_state}(s')$ \\
\qquad \qquad \qquad \colorbox{lightgray}{$todo := todo \concat [s']$} \\
\qquad \qquad $\textsf{examine\_transition}(s, a, s')$ \\
\qquad $\textsf{finish\_state}(s)$ \\
\end{tabular}
\]

\newpage
\subsection{Highway search}

In highway search (see \cite{DBLP:journals/jlp/EngelsGWW09}) a breadth first search is done, with the restriction that at most $N$ states are put in the todo list for each level. The variable $L$ maintains the number of states in the todo list corresponding to the current level, and the variable $c$ counts how many elements have been added corresponding to the next level. Once $c$ reaches the maximum value $N$,  elements are being overwritten randomly. 

\begin{remark}
The specification below deviates from the published version of highway search in the
sense that overwritten elements are added to the set $discovered$. To avoid this,
the structure of the algorithm needs to be changed significantly.
\end{remark}

\[
\begin{tabular}{l}
\textsc{ExploreGraphHighway}($s_0, \colorbox{lightgray}{N}, \textsf{discover\_state}, \textsf{examine\_transition}, \textsf{start\_state}, \textsf{finish\_state})$ \\
$todo := [s_0]$ \\
$discovered := \{ s_0 \}$ \\
$\textsf{discover\_state}(s_0)$ \\
\colorbox{lightgray}{$L := |todo|$} \\
\colorbox{lightgray}{$c := 0$} \\
$\While |todo| > 0$ \Do \\
\qquad $s := todo[0]$ \\
\qquad $todo := todo[1:|todo|]$ \\
\qquad $\textsf{start\_state}(s)$ \\
\qquad $\For s' \in successors(s)  \Space \Do$ \\
\qquad \qquad $\If s' \notin discovered \Space \Then$ \\
\qquad \qquad \qquad $discovered := discovered \cup \{s' \}$ \\
\qquad \qquad \qquad $\textsf{discover\_state}(s')$ \\
\qquad \qquad \qquad \colorbox{lightgray}{$c := c + 1$} \\
\qquad \qquad \qquad \colorbox{lightgray}{\If $c \leq N$ \Then} \\
\qquad \qquad \qquad \qquad \colorbox{lightgray}{$todo := todo \concat [s']$} \\
\qquad \qquad \qquad \colorbox{lightgray}{\Else} \\
\qquad \qquad \qquad \qquad \colorbox{lightgray}{$k := random(\{ 1, \ldots, c\})$} \\
\qquad \qquad \qquad \qquad \colorbox{lightgray}{$\If k \leq N \Space \Then todo[|todo| - k] := s'$} \\
\qquad \qquad $\textsf{examine\_transition}(s, a, s')$ \\
\qquad $\textsf{finish\_state}(s)$ \\
\qquad \colorbox{lightgray}{$L := L - 1$} \\
\qquad \colorbox{lightgray}{\If $L = 0$ \Then} \\
\qquad \qquad \colorbox{lightgray}{$L := |todo|$} \\
\qquad \qquad \colorbox{lightgray}{$c := 0$}
\end{tabular}
\]
In Algorithm 1 of \cite{DBLP:journals/jlp/EngelsGWW09}, the set $Q_d$ stores todo elements corresponding to the current level, and the set $Q_{d+1}$ stores todo elements corresponding to the next level. The algorithm above uses only one list $todo$ that stores both of them. At each iteration of the while loop the first $L$ elements of $todo$ list belong to the current level, and the remaining elements belong to the next level. Furthermore, the algorithm above contains only one application of a random generator, compared to two applications in the original version. The element $k$ is chosen randomly in the range $[1, \ldots, c]$. There is an $N/c$ probability that this value is in the range $[1, \ldots, N]$. If $k$ is inside the range, the element in the $todo$ list with index $k$ (counting from the end) is overwritten. This behaviour matches with the published version.

\newpage
\section{Cycle detection}
For cycle detection the event points in table \ref{table:eventpoints} are insufficient. In \cite{2002:BGL:504206} the following recursive depth first algorithm is given:

\[
\begin{tabular}{l}
\textsc{boost\_dfs\_recursive}($u$) \\
$color[u] := gray$ \\
$\textsf{discover\_vertex}(u)$ \\
$\For (a,v) \in out\_edges(u) \Space \Do$ \\
\qquad $\textsf{examine\_edge}(a, v)$ \\
\qquad $\If color[v] = white \Space \Then$ \\
\qquad \qquad $\textsf{tree\_edge}(a, v)$ \\
\qquad \qquad \textsc{dfs\_recursive}($v$) \\
\qquad $\Else \If color[v] = gray \Space \Then$ \\
\qquad \qquad $\textsf{back\_edge}(a, v)$ \\
\qquad \Else \\
\qquad \qquad $\textsf{forward\_or\_cross\_edge}(a, v)$ \\
\qquad $color[u] := black$ \\
\qquad $\textsf{finish\_vertex}(u)$ \\
\end{tabular}
\]

The code in Boost uses an iterative version:

\[
\begin{tabular}{l}
\textsc{boost\_dfs\_iterative}($u$) \\
$color[u] := gray$ \\
$\textsf{discover\_vertex}(u)$ \\
$stack := [(u, out\_edges(u))]$ \\
$\While |stack| > 0 \Space \Do$ \\
\qquad $u, E := stack.pop\_back()$ \\
\qquad $\While |E| > 0 \Space \Do$ \\
\qquad \qquad $a, v := E[0]$ \\
\qquad \qquad $\textsf{examine\_edge}(u, a, v)$ \\
\qquad \qquad $\If color[v] = white \Space \Then$ \\
\qquad \qquad \qquad $\textsf{tree\_edge}(u, a, v)$ \\
\qquad \qquad \qquad $stack.push\_back(u, E[1:])$ \\
\qquad \qquad \qquad $u := v$ \\
\qquad \qquad \qquad $color[u] := gray$ \\
\qquad \qquad \qquad $\textsf{discover\_vertex}(u)$ \\
\qquad \qquad \qquad $E := out\_edges(u)$ \\
\qquad \qquad \Else \\
\qquad \qquad \qquad $\If color[v] = gray \Space \Then$ \\
\qquad \qquad \qquad \qquad $\textsf{back\_edge}(u, a, v)$ \\
\qquad \qquad \qquad \Else \\
\qquad \qquad \qquad \qquad $\textsf{forward\_or\_cross\_edge}(u, a, v)$ \\
\qquad \qquad \qquad $E := E[1:]$ \\
\qquad $color[u] := black$ \\
\qquad $\textsf{finish\_vertex}(u)$ \\
\end{tabular}
\]

For our purposes we rewrite this as:

\[
\begin{tabular}{l}
\textsc{dfs\_recursive}($s_0, gray$) \\
$gray := gray \cup \{ s_0 \}$ \\
$discovered := \{ s_0 \}$ \\
$\textsf{discover\_state}(s_0)$ \\
$\For (a, s_1) \in out\_edges(s_0) \Space \Do$ \\
\qquad $\textsf{examine\_edge}(s_0, a, s_1)$ \\
\qquad $\If s_1 \notin discovered \Space \Then$ \\
\qquad \qquad $\textsf{tree\_edge}(s_0, a, s_1)$ \\
\qquad \qquad $discovered := discovered \cup \{ s_1 \}$ \\
\qquad \qquad \textsc{dfs\_recursive}($s_1, gray$) \\
\qquad $\Else \If s_1 \in todo \Space \Then$ \\
\qquad \qquad $\textsf{back\_edge}(s_0, a, s_1)$ \\
\qquad \Else \\
\qquad \qquad $\textsf{forward\_or\_cross\_edge}(s_0, a, s_1)$ \\
\qquad $gray := gray \setminus \{ s_0 \}$ \\
\qquad $\textsf{finish\_state}(s_0)$ \\
\end{tabular}
\]

\[
\begin{tabular}{l}
\textsc{dfs\_iterative}($s_0$) \\
$todo := [(s_0, out\_edges(s_0))]$ \\
$discovered := \{ s_0 \}$ \\
$\textsf{discover\_state}(s_0)$ \\
$\While |todo| > 0 \Space \Do$ \\
\qquad $s, E := todo.back()$ \\
\qquad $\While |E| > 0 \Space \Do$ \\
\qquad \qquad $a, s_1 := E.pop\_front()$ \\
\qquad \qquad $\textsf{examine\_edge}(s_0, a, s_1)$ \\
\qquad \qquad $\If s_1 \notin discovered \Space \Then$ \\
\qquad \qquad \qquad $\textsf{tree\_edge}(s_0, a, s_1)$ \\
\qquad \qquad \qquad $discovered := discovered \cup \{ s_1 \}$ \\
\qquad \qquad \qquad $\textsf{discover\_state}(s_1)$ \\
\qquad \qquad \qquad $todo.back() := (s, E)$ \\
\qquad \qquad \qquad $todo := todo \concat [(s_1, out\_edges(s_1))]$ \\
\qquad \qquad \qquad $s, E := todo.back()$ \\
\qquad \qquad $\Else \If s_1 \in todo \Space \Then$ \\
\qquad \qquad \qquad $\textsf{back\_edge}(s_0, a, s_1)$ \\
\qquad \qquad \Else \\
\qquad \qquad \qquad $\textsf{forward\_or\_cross\_edge}(s_0, a, s_1)$ \\
\qquad $\textsf{finish\_state}(s)$ \\
\end{tabular}
\]

Whenever the \textsf{back\_edge} event is triggered, a cycle is found.

\newpage
\section{Untimed state space exploration}
Consider the following untimed linear process specification $P$, with initial state $d_0$. 

\[
\begin{array}{l}
P(d)=
\sum\limits_{i\in I}\sum\limits_{e_i}c_i(d, e_i)\rightarrow a_i(f_i(d,e_i)) \cdot P(g_i(d,e_i))
\end{array}
\]
This linear process is a symbolic representation of a state space, or labeled transition system
(LTS). The previously described graph exploration algorithms can be applied to explore a
state space. Let $rewr$ be a rewriter. An algorithm for untimed state space exploration is 

\[
\begin{tabular}{l}
\textsc{ExploreLPS}($P(d), d_0, rewr, \textsf{discover\_state}, \textsf{examine\_transition},
\textsf{start\_state}, \textsf{finish\_state})$ \\
$s_0 := rewr(d_0, [\:])$ \\
$todo := \{ s_0 \}$ \\
$discovered := \{ s_0 \}$ \\
$\textsf{discover\_state}(s_0)$ \\
$\While todo \neq \emptyset \Space \Do$ \\
\qquad \Choose $s \in todo$ \\
\qquad $todo := todo \setminus \{s\}$ \\
\qquad $discovered := discovered \cup \{s\}$ \\
\qquad $\textsf{start\_state}(s)$ \\
\qquad $\For i \in I  \Space \Do$ \\
\qquad \qquad $condition := rewr(c_i(d, e_i), [d := s])$ \\
\qquad \qquad $\If condition = false  \Space \Then \Continue$ \\
\qquad \qquad $E := \{ e \mid rewr(condition, [e_i := e]) = true \}$ \\
\qquad \qquad $\For e \in E  \Space \Do$ \\
\qquad \qquad \qquad $a := a_i(rewr(f_i(d,e_i), [d:=s,e_i:=e]))$ \\
\qquad \qquad \qquad $s' := rewr(g_i(d,e_i), [d:=s,e_i:=e])$ \\
\qquad \qquad \qquad $\If s' \notin discovered \Space \Then$ \\
\qquad \qquad \qquad \qquad $todo := todo \cup \{ s' \}$ \\
\qquad \qquad \qquad \qquad $discovered := discovered \cup \{s'\}$ \\
\qquad \qquad \qquad \qquad $\textsf{discover\_state}(s')$ \\
\qquad \qquad \qquad $\textsf{examine\_transition}(s, a, s')$ \\
\qquad $\textsf{finish\_state}(s)$ \\
\end{tabular}
\]

The set $E$ is computed using the \textsc{Enumerate} algorithm. This computation may be expensive. Hence the condition $c(d,e_i)$ is first rewritten, since if it evaluates to $false$ the computation of $E$ can be skipped.

\newpage
\section{Timed state space exploration}
Consider the following timed linear process specification $P$, with initial state $d_0$.

\[
\begin{array}{l}
P(d)=
\sum\limits_{i\in I}\sum\limits_{e_i}c_i(d, e_i)\rightarrow a_i(f_i(d,e_i))
\colorbox{lightgray}{$\at t_i(d,e_i)$}
\cdot P(g_i(d,e_i)).
\end{array}
\]
Note that the time tag $t_i(d,e_i)$ is optional. If it is omitted, the corresponding action may happen at an arbitrary time. In timed state space exploration, care is taken that on every trace the time tags are
increasing. In order to achieve that, a time stamp is recorded for each state
in the state space. We use the notation $t \aftertime s$ to denote the state $s$
with associated time stamp $t$.
An algorithm for timed state space exploration is 

\[
\begin{tabular}{l}
\textsc{ExploreLPSTimed}($P(d), d_0, rewr, \textsf{discover\_state}, \textsf{examine\_transition},
\textsf{start\_state}, \textsf{finish\_state})$ \\
$s_0 := rewr(d_0, [\:])$ \\
$todo := \{\colorbox{lightgray}{$0 \aftertime s_0)$}\}$ \\
$discovered := \{ \colorbox{lightgray}{$0 \aftertime s_0)$} \}$ \\
$\textsf{discover\_state}(0 \aftertime s_0)$ \\
$\While todo \neq \emptyset \Space \Do$ \\
\qquad \Choose \colorbox{lightgray}{$t \aftertime s$} $\in todo$ \\
\qquad $todo := todo \setminus \{ \colorbox{lightgray}{$t \aftertime s$} \}$ \\
\qquad $discovered := discovered \cup \{ \colorbox{lightgray}{$t \aftertime s$} \}$ \\
\qquad $\textsf{start\_state}(t \aftertime s)$ \\
\qquad $\For i \in I  \Space \Do$ \\
\qquad \qquad $condition := rewr(c_i(d, e_i), [d := s])$ \\
\qquad \qquad $\If condition = false  \Space \Then \Continue$ \\
\qquad \qquad $E := \{ e \mid rewr(condition, [e_i := e]) = true \}$ \\
\qquad \qquad $\For e \in E  \Space \Do$ \\
\qquad \qquad \qquad \colorbox{lightgray}{$t' := rewr(t_i(d,e_i), [d:=s,e_i:=e])$} \\
\qquad \qquad \qquad \colorbox{lightgray}{$\If t' \leq t \Space \Then \Continue$} \\
\qquad \qquad \qquad $a := a_i(rewr(f_i(d,e_i), [d:=s,e_i:=e]))$ \\
\qquad \qquad \qquad $s' := rewr(g_i(d,e_i), [d:=s,e_i:=e])$ \\
\qquad \qquad \qquad $\If \colorbox{lightgray}{$t' \aftertime s'$} \notin discovered \Space \Then$ \\
\qquad \qquad \qquad \qquad $todo := todo \cup \{ \colorbox{lightgray}{$t' \aftertime s'$} \}$ \\
\qquad \qquad \qquad \qquad $discovered := discovered \cup \{ \colorbox{lightgray}{$t' \aftertime s'$} \}$ \\
\qquad \qquad \qquad \qquad $\textsf{discover\_state}( t' \aftertime s' )$ \\
\qquad \qquad \qquad $\textsf{examine\_transition}(t \aftertime s, a \at t', t' \aftertime s')$ \\
\qquad $\textsf{finish\_state}(t \aftertime s)$ \\
\end{tabular}
\]

\newpage
\section{Stochastic state space exploration}
Consider the following stochastic linear process specification $P$, with initial
state $\frac{p(h)}{h} \cdot P(g(h))$.

\begin{equation}
\begin{array}{l}
P(d)=
\sum\limits_{i\in I}\sum\limits_{e_i}c_i(d, e_i)\rightarrow a_i(f_i(d,e_i))
\colorbox{lightgray}{$\frac{p_i(d,e_i,h_i)}{h_i} $}
\cdot P(g_i(d,e_i,h_i)),
\end{array}
\end{equation}
where $p$ and $p_i$ are stochastic distributions.
We define a \emph{stochastic state} as a set
$\{(q_1, s_1), \ldots, (q_m, s_m)\}$ with $q_j, j = 1 \ldots m$ a sequence of non-zero probabilities that sum up to 1, and $s_j, j = 1 \ldots m$ a sequence of states.
The function \textsc{ComputeStochasticState} is used to compute a stochastic state from its symbolic representation.

\[
\begin{tabular}{l}
\textsc{ComputeStochasticState}($h, p, g, rewr, \sigma$) \\
$result := \emptyset$ \\
$H := \{ (h',q) \mid q = rewr(p, \sigma[h := h']) \wedge q > 0 \}$ \\
$\For (h', q) \in H$ \Do \\
\qquad $s := rewr(g, \sigma[h := h'])$ \\
\qquad $result := result \cup \{ (q, s) \}$ \\
\Return $result$
\end{tabular}
\]
The set $H$ is computed using the \textsc{Enumerate} algorithm. 

An algorithm for stochastic state space exploration is 

\[
\begin{tabular}{l}
\textsc{ExploreLPSStochastic}($P(d), \frac{p(h)}{h} \cdot P(g(h)), rewr, \textsf{discover\_state}, \textsf{examine\_transition},
\textsf{start\_state}, \textsf{finish\_state}, \textsf{discover\_initial\_state})$ \\
\colorbox{lightgray}{$\hat{s_0} := \textsc{ComputeStochasticState}(h, p(h), g(h), rewr, \emptylist)$} \\
\colorbox{lightgray}{$S := \{ s_i \mid (q_i, s_i) \in \hat{s_0} \}$} \\
\colorbox{lightgray}{$\textsf{discover\_initial\_state}(\hat{s_0})$} \\
\colorbox{lightgray}{$\For s \in S \Space \Do $}\\
\qquad $todo := todo \cup \{ s \}$ \\
\qquad $discovered := discovered \cup \{ s \} $ \\
\qquad $\textsf{discover\_state}(s)$ \\
$\While todo \neq \emptyset \Space \Do$ \\
\qquad \Choose $s \in todo$ \\
\qquad $todo := todo \setminus \{s\}$ \\
\qquad $discovered := discovered \cup \{s\}$ \\
\qquad $\textsf{start\_state}(s)$ \\
\qquad $\For i \in I  \Space \Do$ \\
\qquad \qquad $condition := rewr(c_i(d, e_i), [d := s])$ \\
\qquad \qquad $\If condition = false  \Space \Then \Continue$ \\
\qquad \qquad $E := \{ e \mid rewr(condition, [e_i := e]) = true \}$ \\
\qquad \qquad $\For e \in E  \Space \Do$ \\
\qquad \qquad \qquad $a := a_i(rewr(f_i(d,e_i), [d:=s,e_i:=e]))$ \\
\qquad \qquad \qquad \colorbox{lightgray}{$\hat{s'} := \textsc{ComputeStochasticState}(h_i, p_i(d,e_i,h_i), g_i(d,e_i,h_i), rewr, [d:=s, e_i:=e])$} \\
\qquad \qquad \qquad \colorbox{lightgray}{$S' := \{ s_i \mid (q_i, s_i) \in \hat{s'} \}$} \\
\qquad \qquad \qquad \colorbox{lightgray}{$\For s' \in S'$ \Do} \\
\qquad \qquad \qquad \qquad $\If s' \notin discovered \Space \Then$ \\
\qquad \qquad \qquad \qquad \qquad $todo := todo \cup \{ s' \}$ \\
\qquad \qquad \qquad \qquad \qquad $discovered := discovered \cup \{s'\}$ \\
\qquad \qquad \qquad \qquad \qquad $\textsf{discover\_state}(s')$ \\
\qquad \qquad \qquad $\textsf{examine\_transition}(s, a, \colorbox{lightgray}{$\hat{s'}$})$ \\
\qquad $\textsf{finish\_state}(s)$ \\
\end{tabular}
\]

\newpage
\section{Caching}
The computation of the set of solutions $E$ in the \textsc{ExploreLPS} is expensive. Therefore it may be a good idea to cache these solutions. Caching can be done locally (i.e. using a separate cache for each summand),
or globally. This leads to the following variants of the algorithm. We assume that $FV$ is a function that
computes free variables of an expression. Let $\mathcal{D}$ be the set of process parameters
(i.e. the elements of $d$).

\subsection{Local caching}
In the local caching algorithm for each summand $i$ a mapping $C_i$ is maintained. The cache key is comprised of the actual values of the process parameters that appear in the condition $c_i(d, e_i)$.
\[
\begin{tabular}{l}
\textsc{ExploreLPSLocallyCached}($P(d), d_0, rewr, \textsf{discover\_state}, \textsf{examine\_transition},
\textsf{start\_state}, \textsf{finish\_state})$ \\
$s_0 := rewr(d_0, [\:])$ \\
$todo := \{ s_0 \}$ \\
$discovered := \{ s_0 \}$ \\
$\textsf{discover\_state}(s_0)$ \\
\colorbox{lightgray}{\For $i \in I$ \Do} \\
\colorbox{lightgray}{\qquad $C_i := \emptymap$} \\
\colorbox{lightgray}{\qquad $\gamma_i := FV(c_i(d, e_i)) \cap \mathcal{D}$} \\
$\While todo \neq \emptyset \Space \Do$ \\
\qquad $\Choose s \in todo$ \\
\qquad $todo := todo \setminus \{s\}$ \\
\qquad $discovered := discovered \cup \{s\}$ \\
\qquad $\textsf{start\_state}(s)$ \\
\qquad $\For i \in I$ \Do \\
\qquad \qquad \colorbox{lightgray}{$key := \gamma_i[d:=s]$} \\
\qquad \qquad \colorbox{lightgray}{\If $key \in keys(C_i)$ \Then} \\
\qquad \qquad \colorbox{lightgray}{\qquad $E := C_i[key]$} \\
\qquad \qquad \colorbox{lightgray}{\textbf{else}} \\
\qquad \qquad \colorbox{lightgray}{\qquad $E := \{ e \mid rewr(c_i(d, e_i), [d:=s,e_i:=e]) = true \}$} \\
\qquad \qquad \colorbox{lightgray}{\qquad $C_i := C_i \cup \{(key, E)\}$} \\
\qquad \qquad $\For e \in E  \Space \Do$ \\
\qquad \qquad \qquad $a := a_i(rewr(f_i(d,e_i), [d:=s,e_i:=e]))$ \\
\qquad \qquad \qquad $s' := rewr(g_i(d,e_i), [d:=s,e_i:=e])$ \\
\qquad \qquad \qquad $\If s' \notin discovered \Space \Then$ \\
\qquad \qquad \qquad \qquad $todo := todo \cup \{ s' \}$ \\
\qquad \qquad \qquad \qquad $discovered := discovered \cup \{s'\}$ \\
\qquad \qquad \qquad \qquad $\textsf{discover\_state}(s')$ \\
\qquad \qquad \qquad $\textsf{examine\_transition}(s, a, s')$ \\
\qquad $\textsf{finish\_state}(s)$ \\
\end{tabular}
\]

\subsection{Global caching}
In the global caching algorithm one mapping $C$ is maintained. To achieve this, the condition of the summands is added to the cache key. If many summands share the same condition, global caching may be beneficial. In practice this doesn't seem to happen much.
\[
\begin{tabular}{l}
\textsc{ExploreLPSGloballyCached}($P(d), d_0, rewr, \textsf{discover\_state}, \textsf{examine\_transition},
\textsf{start\_state}, \textsf{finish\_state})$ \\
$todo := \{d_0\}$ \\
$discovered := \emptyset$ \\
\colorbox{lightgray}{$C := \emptyset$} \\
\colorbox{lightgray}{\For $i \in I$ \Do} \\
\colorbox{lightgray}{\qquad $\gamma_i := FV(c_i(d, e_i)) \cap \mathcal{D}$} \\
$\While todo \neq \emptyset \Space \Do$ \\
\qquad $\Choose s \in todo$ \\
\qquad $todo := todo \setminus \{s\}$ \\
\qquad $discovered := discovered \cup \{s\}$ \\
\qquad $\textsf{start\_state}(s)$ \\
\qquad $\For i \in I \Do$ \\
\qquad \qquad \colorbox{lightgray}{$key := c_i(d,e_i) \concat \gamma_i[d:=s]$} \\
\qquad \qquad \colorbox{lightgray}{\If $key \in keys(C)$ \Then} \\
\qquad \qquad \colorbox{lightgray}{\qquad $T := C[key]$} \\
\qquad \qquad \colorbox{lightgray}{\textbf{else}} \\
\qquad \qquad \colorbox{lightgray}{\qquad $T := \{ t \mid rewr(c_i(d, e_i), [d:=s,e_i:=t]) = true \}$} \\
\qquad \qquad \colorbox{lightgray}{\qquad $C := C \cup \{(key, T)\}$} \\
\qquad \qquad $\For e \in E  \Space \Do$ \\
\qquad \qquad \qquad $a := a_i(rewr(f_i(d,e_i), [d:=s,e_i:=e]))$ \\
\qquad \qquad \qquad $s' := rewr(g_i(d,e_i), [d:=s,e_i:=e])$ \\
\qquad \qquad \qquad $\If s' \notin discovered \Space \Then$ \\
\qquad \qquad \qquad \qquad $todo := todo \cup \{ s' \}$ \\
\qquad \qquad \qquad \qquad $discovered := discovered \cup \{s'\}$ \\
\qquad \qquad \qquad \qquad $\textsf{discover\_state}(s')$ \\
\qquad \qquad \qquad $\textsf{examine\_transition}(s, a, s')$ \\
\qquad $\textsf{finish\_state}(s)$
\end{tabular}
\]

In this algorithm $C$ is a mapping, with $keys(C) = \{ k \mid \exists_{v}: (k,v) \in C \}$. We use the notation $C[k]$ to denote the unique element $v$ such that $(k,v) \in C$.

\newpage
\section{Confluence Reduction}

Confluence reduction (see \cite{DBLP:conf/mfcs/GrooteP00}, \cite{Blom01partialt-confluence} and \cite{DBLP:conf/cav/BlomP02}) is an on-the-fly state space exploration method that produces a reduced state space. For confluence reduction we assume that the set of summands $I$ is partitioned into a set $I_{reqular}$ of 'regular' summands, and a set $I_{confluent}$ of confluent $tau$-summands. The confluent $tau$-summands are used to determine a unique representative state that is reachable via confluent $\tau$ steps. This is done using the graph algorithm \textsc{FindRepresentative}. This leads to the following variant of the algorithm:

\[
\begin{tabular}{l}
\textsc{ExploreLPSConfluence}($P(d), d_0, rewr, \textsf{discover\_state}, \textsf{examine\_transition},
\textsf{start\_state}, \textsf{finish\_state})$ \\
$s_0 := \colorbox{lightgray}{$\textsc{FindRepresentative}(rewr(d_0, [\:]))$}$ \\
$todo := \{\ s_0 \}$ \\
$discovered := \{ s_0 \}$ \\
$\textsf{discover\_state}(s_0)$ \\
\textbf{while} $todo \neq \emptyset $ \Do \\
\qquad \Choose $s \in todo$ \\
\qquad $todo := todo \setminus \{s\}$ \\
\qquad $discovered := discovered \cup \{s\}$ \\
\qquad $\textsf{start\_state}(s)$ \\
\qquad $\For i \in \colorbox{lightgray}{$I_{regular}$} \Do$ \\
\qquad \qquad $condition := rewr(c_i(d, e_i), [d := s])$ \\
\qquad \qquad $\If condition = false  \Space \Then \Continue$ \\
\qquad \qquad $E := \{ e \mid rewr(condition, [e_i := e]) = true \}$ \\
\qquad \qquad $\For e \in E  \Space \Do$ \\
\qquad \qquad \qquad $a := a_i(rewr(f_i(d,e_i), [d:=s,e_i:=t]))$ \\
\qquad \qquad \qquad $s' := \colorbox{lightgray}{$\textsc{FindRepresentative}(rewr(g_i(d,e_i), [d:=s,e_i:=t]))$}$ \\
\qquad \qquad \qquad $\If s' \notin discovered \Space \Then$ \\
\qquad \qquad \qquad \qquad $todo := todo \cup \{ s' \}$ \\
\qquad \qquad \qquad \qquad $discovered := discovered \cup \{s'\}$ \\
\qquad \qquad \qquad \qquad $\textsf{discover\_state}(s')$ \\
\qquad \qquad \qquad $\textsf{examine\_transition}(s, a, s')$ \\
\qquad $\textsf{finish\_state}(s)$
\end{tabular}
\]

As suggested in \cite{DBLP:conf/cav/BlomP02} Tarjan's strongly connected component (SCC) algorithm (see \cite{Tarjan72depthfirst}) can be used to compute a unique representative.

\subsection{Tarjan's SCC algorithm}
A recursive implementation of Tarjan uses four global variables $stack$, $low$, $disc$ and $result$. The function
\textsc{TarjanRecursive} computes the connected component reachable from node $u$. In this function it is assumed that the function call $successors(u)$ returns the successor states of $u$ in a deterministic order.

\[
\begin{tabular}{l}
$stack := \emptylist$ \\
$low := \emptymap$ \\
$disc := \emptymap$ \\
$result := \emptylist$ \\
\\
\textsc{TarjanRecursive}$(u)$ \\
$k := |disc|$ \\
$disc[u] := k$ \\
$low[u] := k$ \\
$stack := stack \concat [u]$ \\
\For $v \in successors(u)$ \Do \\
\qquad \If $v \notin low$ \Then \\
\qquad \qquad \textsc{TarjanRecursive}$(u)$ \\
\qquad \qquad $low[u] := min(low[u], low[v])$ \\
\qquad \Else \If $v \in stack$ \Then \\
\qquad \qquad low[u] := min(low[u], disc[v]) \\
\qquad \If $low[u] = disc[u]$ \Then \\
\qquad \qquad $component := \emptylist$ \\
\qquad \qquad \While $true$ \Do \\
\qquad \qquad \qquad $v := stack[|stack| - 1]$ \\
\qquad \qquad \qquad $stack := stack[0:|stack| - 1]$ \\
\qquad \qquad \qquad $component := component \concat [v]$ \\
\qquad \qquad \qquad \If $v == u$ \Then \Break \\
\qquad \qquad $result := result \concat [component]$ \\
\end{tabular}
\]
A side effect of a call \textsc{TarjanRecursive}$(u)$ is that $result$ contains the connected components that have been found.

\subsection{FindRepresentative}
Due to properties of confluent $\tau$-summands, there is always only one terminal strongly connected component, i.e. a strongly connected component without outgoing edges. Furthermore, the first strongly connected component reported by Tarjan's algorithm is always terminating. For our implementation of \textsc{FindRepresentative} we prefer to use an iterative version of Tarjan's SCC algorithm. The reason for this is that an iterative version can be more easily interrupted once the first SCC has been found. The algorithm description in \cite{TarjanIterative} has been used as a model for our solution.

\[
\begin{tabular}{l}
\textsc{FindRepresentative}$(u)$ \\
$stack := \emptylist$ \\
$low := \emptymap$ \\
$disc := \emptymap$ \\
$work := [(u, 0)]$ \\
$\While work \neq \emptylist \Space \Do$ \\
\qquad $(u, i) := work[|work| - 1]$ \\
\qquad $work := work[0 : |work| - 1]$ \\
\qquad $\If i = 0 \Space \Then$ \\
\qquad \qquad $k := |disc|$ \\
\qquad \qquad $disc[u] := k$ \\
\qquad \qquad $low[u] := k$ \\
\qquad \qquad $stack := stack \concat [u]$ \\
\qquad $recurse := false$ \\
\qquad $\For j \in [i, \ldots, |successors(u)|]$ \Do \\
\qquad \qquad $v := successors(u)[j]$ \\
\qquad \qquad $\If v \notin disc$ \Then \\
\qquad \qquad \qquad $work := work \concat [(u, j+1)]$ \\
\qquad \qquad \qquad $work := work \concat [(v, 0)]$ \\
\qquad \qquad \qquad $recurse := true$ \\
\qquad \qquad \qquad \Break \\
\qquad \qquad \Else \If $v \in stack$ \Then \\
\qquad \qquad \qquad low[u] := min(low[u], disc[v]) \\
\qquad $\If recurse$ \Then \Continue \\
\qquad \If $low[u] = disc[u]$ \Then \\
\qquad \qquad $result := u$ \\
\qquad \qquad \While $true$ \Do \\
\qquad \qquad \qquad $v := stack[|stack| - 1]$ \\
\qquad \qquad \qquad $stack := stack[0:|stack| - 1]$ \\
\qquad \qquad \qquad \If $v == u$ \Then \Break \\
\qquad \qquad \qquad $\If v < result \Space \Then result := v$ \\
\qquad \qquad $\Return result$ \\
\qquad $\If work \neq \emptylist \Space \Then$ \\
\qquad \qquad $v := u$ \\
\qquad \qquad $(u, z) := work[|work| - 1]$\\
\qquad \qquad $low[u] := min(low[u], low[v])$ \\
\end{tabular}
\]

\newpage
\bibliographystyle{plain}
\bibliography{state-space-exploration}
\end{document}
