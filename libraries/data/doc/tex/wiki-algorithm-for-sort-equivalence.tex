\documentclass{article}
\usepackage{a4wide}
\usepackage{latexsym}
\usepackage{times}
%\usepackage{here}
\usepackage{theorem}
\usepackage{url}
\usepackage{pstricks,pst-node,pst-grad,pst-plot,pst-coil}
\usepackage[english]{babel}
\usepackage[final]{graphics}
\usepackage{amsmath,amssymb}
\usepackage{amsfonts}
\usepackage{array}
\usepackage{calc}
\usepackage{xspace}
\usepackage{color}
\usepackage{epsfig}
\usepackage{subfigure}
\usepackage{float}
\usepackage{stmaryrd}
\pagestyle{plain}

\newcommand{\Bool}{{\mathbb B}}
\newcommand{\Real}{\mathbb{R}}
\newcommand{\Nat}{\mathbb{N}}
\newcommand{\Pos}{\mathbb{N^{+}}}
\newcommand{\Int}{\mathbb{Z}}

\newcolumntype{L}{>{$}l<{$}}%stopzone%stopzone%stopzone
\newcolumntype{C}{>{$}c<{$}}%stopzone%stopzone%stopzone
\newcolumntype{R}{>{$}r<{$}}%stopzone%stopzone%stopzone

\newenvironment{mcrl2}%
{\begin{trivlist}
\item\begin{tabular}{@{}>{\bf}p{2.3em}L@{\ }L@{\ }L@{\ }L@{\ }L@{\ }L@{\ }L@{\ }L}}%
{\end{tabular}\end{trivlist}}

\title{An algorithm to find a representant for sorts in the context of sort aliases and
recursive sorts}
\author{Jan Friso Groote}
\date{May 4, 2010}
\begin{document}
\maketitle
%\section{Type equivalence}
\noindent%
In mCRL2 it is possible to define sort aliases, which have the form
$A=B$. This means that sort $A$ and sort $B$ are considered the same,
and are completely exchangeable. 

Typical examples of sort aliases are
\begin{mcrl2}
sort&{\it Time}=\Nat;\\
    &F=C\rightarrow D;\\
    &L=List(C);\\
    &{\it Complex}={\it Bag}(A{\rightarrow}{\it Set}(B)); 
\end{mcrl2}
It is also possible to define
structured sorts that can be recursive (contrary to function sorts, lists, sets, 
and bags above, which cannot be recursive). 

A structured sort has the shape:
\begin{mcrl2}
sort & A={\bf struct}
    &c_{1}({\it pr}_{1,1}: A_{1,1}, & \ldots & , {\it pr}_{1,k_{1}}: A_{1,k_{1}})
      ?{\it isC_{1}}\\
&\hfill |
    &c_{2}({\it pr}_{2,1}: A_{2,1}, & \ldots & , {\it pr}_{2,k_{2}}: A_{2,k_{2}})
      ?{\it isC_{2}}\\
&                                 & \multicolumn{1}{c}{\vdots}\\
&\hfill |
    &c_{n}({\it pr}_{n,1}: A_{n,1}, & \ldots & , {\it pr}_{n,k_{n}}: A_{n,k_{n}})
      ?{\it isC_{n}}\\
\end{mcrl2}
This declares sort $A$ to consists of 
$n$ constructors $c_i$, projection functions ${\it pr}_{i,j}$ and
recognisers ${\it isC_i}$. All the $A_{i,j}$ are sorts. The $A_{i,j}$ 
can be equal to $A$, in which case $A$ is a recursive structured sort.

A very well known example is that of a tree data structure in which natural 
numbers can be stored.
\begin{mcrl2}
sort& {\it Tree}=
{\bf struct}~{\it node}({\it left}:{\it Tree},{\it right}:{\it Tree})~|~{\it leave}(\Nat)?
{\it is\_leave};
\end{mcrl2}

By combining aliases and structured sorts, 
it is possible to have very different looking sort expressions that denote
the same sorts. Two such expressions are equal if by folding and unfolding the definitions in
sort aliases and in structured sorts, 
the sort expressions can be rewritten to each other. 
When manipulating terms it is inconvenient
to be forced to perform folding and unfolding to determine equivalences of sorts. Therefore, it
is useful to replace all equivalent sorts by a single unique representation, reducing the check
for equivalency of sorts to checking whether the sorts are syntactically equal. This process
is called {\it sort normalisation}. Note that normalisation depends on a data specification.
Adding one sort alias or one structured sort can change the outcome of the 
normalisation procedure. 

Below we give an algorithm to perform normalisation which is used in the mCRL2 tool suite.
The essential idea is that all 
the definitions of structured sorts are interpreted from right to left, whereas all other 
rules are interpreted as rewrite rules from left to right. So, in the example above,
${\it Time}$ is rewritten to $\Nat$, $F$ is rewritten to $C\rightarrow D$, etc. Because,
ordinary sort aliases rules cannot be recursive, and structured sorts shrink with every
rewrite step, this rewrite system is terminating.

But as the rewrite system is not confluent, unique normal forms are not guaranteed. 
The following example shows the problem.
\begin{mcrl2}
sort & A={\bf struct}~f(\Nat);\\
     & C={\bf struct}~f(\Nat);
\end{mcrl2}
A sort of the shape ${\bf struct}~f(\Nat)$ can be normalised to sort $A$ and sort $B$. 
In order to deal with this problem, we apply Knuth-Bendix completion, to guarantee that
all normal forms are unique.

The algorithm is performed in three steps.
First, the set of aliases is checked for recursive definitions
in all sorts except the structured sorts. If such a loop in the sort aliases is
detected, an exception is thrown. The algorithm consists of
a simple depth first search.
 
Second, the set of aliases is stored from 
left to right in a map $\it sort\_aliases\_to\_be\_investigated$, except structured
sorts which are stored from right to left.
So, if $A=B$ is declared, then ${\it sort\_aliases\_to\_be\_investigated}(B)=A$.
The name of this map uses the word `{\it investigated}' because it
must be investigated whether these rewrite rules are confluent.

As a third step the sort aliases are taken as rewrite rules, and 
a form of Knuth-Bendix completion is applied to them, to transform them
into a confluent term rewriting system, guaranteeing unique representations.

So, if there are two overlapping left hand sides in the rewrite system, this
means that one term is a subterm of the other. So, we have a rule 
$C(g(t))\rightarrow u_1$ and a rule $g(t)\rightarrow u_2$ where $C$ represents a
possibly empty context. So, the term
$C(g(t))$ can rewrite to both $u_1$ and $C(u_2)$. In this case we simply add
a rewrite rule $C(u_2)\rightarrow u_1$. 

An important observation is that the rules always have one of the following 
shapes:
\[\begin{array}{rcl}
{\bf struct}\ldots  &\rightarrow& A,\\
B&\rightarrow & {\it Exp}
\end{array}\]
where $A$ and $B$ are basic sorts and ${\it Exp}$ is a sort expression which
can be a basic sort, but can also contain all other type forming constructs.
There are the following invariants on the rules. 
For each basic sort $B$ there is at most one rule of the form
$B\rightarrow \cdots$. Furthermore, a basic sort $A$ occurring at the right
of a struct rule can never occur as the left hand side of a rewrite rule also. 
 
So, when one left hand side of a rule overlaps with another left hand side, one of the rules
must have the shape ${\bf struct}\ldots \rightarrow\ldots$, whereas the other
can contain a struct or a basic term at the left hand side. As the rule with
a struct rewrites to a basic sort $A$, the newly added rewrite rule has $A$ at
its right hand side. 

The number of newly added rules in this way is bounded. When both left hand sides contain 
structs,
the newly added rule has a strictly smaller number of structs in its right hand side
than one of its originals. Moreover, no new basic sort is introduced that can 
act as the lhs of a new rule. When a rule of the shape $A\rightarrow {\it Exp}$ contains overlap,
a rule is obtained where an occurrence of $A$ is replaced by an occurrence of ${\it Exp}$.
But as these rules are acyclic, this can only be performed a finite number of times.

In more detail, we have two sets of rewrite rules. One that is definitive
${\it m\_normalised\_sort\_aliases}$ and $\it sort\_aliases\_to\_be\_investigated$ 
that contains sort rewrite rules
still to be investigated. Initially, all rules are in 
$\it sort\_aliases\_to\_be\_investigated$. Each rewrite rule $t_1\rightarrow u_1$
in $\it sort\_aliases\_to\_be\_investigated$ is checked with
each rule $t_2\rightarrow u_2$ in ${\it m\_normalised\_sort\_aliases}$.
If $t_1$ is a subterm of $t_2$ (i.e.\ $t_2=C(t_1)$) and 
$u_2$ and $C(u_1)$ do not have the same normal forms, then a rule $C(u_1)\rightarrow
u_2$ is added to $\it sort\_aliases\_to\_be\_investigated$.
If $t_2$ is a subterm of $t_1$ a symmetric sequence of steps is done.
After all rewrite rules $t_1\rightarrow u_1$ in ${\it m\_normalised\_sort\_aliases}$
have been investigated, $t_2\rightarrow u_2$ is added to 
${\it m\_normalised\_sort\_aliases}$.

The resulting rewrite system is terminating, provided that the original rewrite system
was terminating. Each new rule that is added has the shape $C = a$, where $C$ is
a basic or complex type, and $a$ is a basic sort, which is a normal form in the
rewrite system. The only way that there is non termination, is when there
is an infinite sequence of basic sorts $a_1$, $a_2,\ldots$, such that $a_i$ rewrites
to $a_{i+1}$. This loop came into existence by adding some rewrite rule $a=a'$ at
some moment in time, where $a'$ was not a normal form. But this cannot happen,
because by construction $a'$ is a normal form.

After constructing the normal forms, the content of ${\it m\_normalised\_sort\_aliases}$
is copied into {\it m\_normalised\_aliases}, where every right hand side is normalised,
to speed up rewriting when applied to concrete sorts.

Normalisation of concrete sorts is now very simple. Every sort which equals a
left hand side of a sort alias is replaced by the right hand side. This is repeated
until no such substitution can be applied. This can be done using a simple 
innermost rewriting procedure. This rewriter has been implemented in 
{\tt normalize\_sorts\_function}. 
\end{document}
